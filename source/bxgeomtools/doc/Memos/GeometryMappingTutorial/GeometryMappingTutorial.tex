%% GeometryMappingTutorial.tex
%%
%%
\documentclass[a4paper,12pt]{article}

\usepackage[T1]{fontenc} 
\usepackage{ucs} 
\usepackage[utf8x]{inputenc} 
%%french%%\usepackage[frenchb]{babel}
\usepackage{amsmath}
\usepackage{amssymb}
\usepackage{latexsym}
\usepackage{verbatim}
\usepackage{moreverb}
\usepackage{fancyvrb}
\usepackage{alltt} 
\usepackage{eurosym} 
\usepackage{hyperref}
\usepackage{colortbl}
\usepackage{epsfig}
\usepackage{graphicx}
\usepackage{pgf}

\addtolength{\textwidth}{+2cm}
\addtolength{\textheight}{+3cm}
\addtolength{\topmargin}{-1.5cm}
\addtolength{\oddsidemargin}{-1cm}

\newcommand{\basepath}{.}
\newcommand{\imagepath}{\basepath/images}
\newcommand{\codingpath}{\basepath/coding}
\newcommand{\pdftextpath}{\basepath/pdftex_t}
\newcommand{\pdftexpath}{\basepath/pdftex}

%% declare_verbatim.tex
%% -*- mode: latex;-*-
%%
\DefineVerbatimEnvironment%
{ShellVerbatim}{Verbatim}%
{fontsize=\small,%
frame=single,%
framesep=2mm,%
framerule=0.25mm,%
labelposition=topline,%
numbers=none%
}

\DefineVerbatimEnvironment%
  {PathVerbatim}{Verbatim}%
{fontsize=\small,%
frame=single,%
framesep=1mm,%
framerule=0.15mm,%
numbers=none%
}

\DefineVerbatimEnvironment%
{CppVerbatim}{Verbatim}%
{commandchars=\\\{\},%
fontsize=\small,%
frame=single,%
framesep=2mm,%
framerule=0.25mm,%
labelposition=topline,
numbers=left,%
numbersep=2pt%
}

%% end of declare_verbatim.tex


%% Example :
%% \VerbatimInput[frame=single,
%% numbers=left,
%% numbersep=2pt,
%% firstline=1,
%% fontsize=\small,
%% showspaces=false]{\codingpath/code_snippet.cxx}
%% \caption{Some code snippet.}\label{fig:code_snippet}

\newcommand{\pn}{\par\noindent}
\newcommand{\TT}[1]{"\texttt{#1}"}

\title{Geometry mapping tutorial\\%
{\small{(Software/geomtools/GeometryMappingTutorial -- version 0.1)}}}
\author{F. Mauger <\texttt{mauger@lpccaen.in2p3.fr}>}
\date{2011-11-22}

%%%%%%%%%%%%%%%%%%%%%%%%%%%%%%%%%%%%%%%%%%%%%%%%%%%%%%%%%%%%%%%%
\begin{document}

\maketitle

\begin{abstract}
Abstract...
\end{abstract}

\tableofcontents

\section{Introduction}

\pn Introduction\dots

\pagebreak

%% gid.tex

\section{Geometry identifiers}

\subsection{Presentation of the concept}

\pn A \emph{geometry  identifier}, also known as  \emph{GID} or \emph{geom
  ID}, is an unique identifier associated to a geometry volume that is
part of a  geometry hierarchy. The figure \ref{fig:0}  shows a virtual
geometry setup made  of a collection of dictinct volumes  (A, B, C, D,
E, F, G) placed in a  reference frame (doted rectangle).  Here some of
the volumes  (F, G) are  included in another  one (E); this  implies a
natural hierarchy relationship between these last 3 volumes : volume E
is the mother of volumes F an  G; volumes F and G are the daughters of
volume E.

\begin{figure}[h]
\begin{center}
\scalebox{0.75}{\input{\pdftextpath/fig_0.pdftex_t}}
\end{center}
\caption{A simple geometry setup.}\label{fig:0}
\end{figure}

%\clearpage
\pn In  an application  that depends  on  and manipulates  such a  virtual
geometry setup,  it may  be interesting to  have access to  an unified
identification scheme for all or  part of the volumes.  In the present
case, volumes  labelled A,  B, E, F  and G  are each associated  to an
unique \emph{geometry identifier} (blue rounded boxes);  
on the other hand volumes  C and D
do not  benefit of  such association,  because it may be  not needed  
in the specific context of the application.

\pn Within the  \texttt{geomtools} program  library, the association  of a
geometry  volume placed in  the virtual  setup and  its GID  is called
\emph{geometry  mapping}.   Using such  a  concept,  it is  possible to
implement some techniques/algorithms that enable for example to:

\begin{itemize}

\item retrieve  the properties of a  volume given its  GID : placement
  (position  and rotation  matrix)  in the  reference  frame (or  some
  arbitrary  relative frame),  shape,  color, material  or any  useful
  property in the context of the application.
\begin{center}
\scalebox{1.0}{\input{\pdftextpath/fig_1.pdftex_t}}
\end{center}

\item  find/compute the  GID associated  to a  volume that  contains a
  given position P within the geometry setup :
\begin{center}
\scalebox{1.0}{\input{\pdftextpath/fig_2.pdftex_t}}
\end{center}

\end{itemize}

\pn These techniques can be implemented as \emph{locator} algorithms.

\pn But wait ! The following figure shows the case where the point Q
is inside both volumes E \textbf{and} G ;
\begin{center}
\scalebox{1.0}{\input{\pdftextpath/fig_3.pdftex_t}}
\end{center}

\pn It means that, in general,  the determination of the GID associated to
the volume  where a point lies  in can be ambiguous.  The final result
depends on what the user/application expects or at which depth of the
hierarchy the search is performed. In the present case, the ambiguity 
can be removed if one gives an additional information to the search
\emph{locator} algorithm. Such an information could be:
\begin{itemize}
\item the maximum depth of the volume hierarchy,
\item a specific depth of the volume hierarchy,
\item the \emph{category} of the object which is expected to 
  contain the given position $Q$.
\end{itemize}

\subsection{Design of the \texttt{geomtools::geom\_id} class}
 
\pn The above considerations give us some hints to design the implementation of
a \emph{geometry identifier} class. The minimal useful embedded information
to uniquely describe a volume could imply:
\begin{itemize}

\item  the specification  of the  \emph{category} of  object  a volume
  belongs   to:  a  physical   volume  is   thus  interpreted   as  an
  \emph{instance} of the \emph{geometry cateogory},

\item some informations that  describe the full hierarchical path from
  the top level  mother volume that contains the  considered volume to
  the effective depth of the volume in this hierarchy.

\end{itemize}

\pn This approach is used in \texttt{geomtools} to implement
the \texttt{geomtools::geom\_id} class.

\pn A GID is thus composed of the following attributes :
\begin{itemize}

\item the  \emph{geometry category}  is stored as  an \texttt{integer}
  value for  storage optimization and fast  manipulation. This integer
  value is named the \emph{type}  of the GID (or \emph{geometry type}). 
  Conventionnaly the value
  \texttt{-1}  means that  the type  is  not defined  (or not  valid).
  Value  \texttt{0} is  reserved  to  the top  level  category of  the
  geometry  setup \footnote{in the terminology of the
    GEANT4 program library,  it  corresponds  to  the
  unique \emph{world volume}}. All values  greater than \texttt{0} can be used
  to  represent   the  \emph{type}  of   some  volume.  \\It   is  the
  responsability of  the application  to manage the  classification of
  different categories of volumes hosted in the setup and associate an
  unique type integer value to any given category.

\item  a  collection of  successive  \emph{addresses}  that allows  to
  identify/traverse the  different levels  of the hierarchy  that host
  the  corresponding  volume. These  addresses  are  implemented as  a
  vector of integers. The size  of this vector points out the position
  depth  of  the  volume  in  some  hierarchical  relationship.   Each
  \emph{address}  in  the  vector  is  an  integer  value  with  value
  \texttt{-1} if  not defined  (not valid) and  \texttt{0} or  more if
  valid. \\It is  the responsability of the application  to allocate a
  meaning to each  address in the collection. It  is expected that the
  ordering of  addresses in the collection  reflects some hierarchical
  placement relationship between the volume  and its mother volume, grand-mother
  volume (if any)\dots this logics is managed by the application.
  
\end{itemize}

\pn    So,    within     \texttt{geomtools},    instances    of    the
\texttt{geomtools::geom\_id} class are represented using the following
format :
\begin{center}
\verb+[TYPE:ADDR0.ADDR1.+$\cdots$\verb+.ADDRN]+
\end{center}
\pn where \verb+TYPE+ is the value of the type (geometry category) and
\verb+ADDRN+ are  the values of each  address at successive  depths of the
geometry hierarchy.

\pn Examples : \verb+[6:0]+,  \verb+[6:1]+, \verb+[6:2]+
\verb+[12:2.0]+, \verb+[12:2.1]+, \verb+[18:2.0.3]+, \verb+[18:2.0.4]+.

\pn  It  should be  mentionned  that the  \texttt{geomtools::geom\_id}
class  is  a  low-level  class  used to  store  the  raw  informations
corresponding to  the geometry  identifier associated to  some volume.
The  choice  for  using  integers   as  the  basic  support  to  store
information  has been made  for convenience  and performance  both for
storage and  manipulation.  All  the \emph{intelligence} that  gives some
meaning to  the values used to  store the \emph{type}, as  well as the
interpretation of  the vector of \emph{addresses}, must  be managed at
higher  level.  The  \texttt{geomtools::id\_mgr} class  is responsible
for such features.

%% end of gid.tex


\pagebreak
%% gid_concepts.tex

\section{More concepts about the use of GIDs}

\subsection{A simple use case}

\pn  In order  to illustrate  the basic  concepts used  to  create and
manipulate  GIDs  with  the  help of  the  \texttt{geomtools}  program
library,  we  present  here  a  simple virtual  domestic  setup  which
accomodates several kinds of  objects with some simple (and realistic)
mother-to-daughter relationship between them.

\pn  We start first  by the  definition of  the set  of \emph{geometry
  categories}  the objects belongs  to.  For  now, we  have 10  of such
\emph{type} of objects :

\begin{itemize}
  
\item the \emph{category} \TT{house} is associated to the \emph{type}
  value \texttt{1},
  
\item the \emph{category} \TT{floor} is associated to the \emph{type}
  value \texttt{2},
  
\item the \emph{category} \TT{room} is associated to the \emph{type}
  value \texttt{3},
 
\item the \emph{category} \TT{table} is associated to the \emph{type}
  value \texttt{4},
  
\item the \emph{category} \TT{chair} is associated to the \emph{type}
  value \texttt{6},
   
\item the \emph{category} \TT{bed} is associated to the \emph{type}
  value \texttt{9},
  
\item the \emph{category} \TT{cupboard} is associated to the \emph{type}
  value \texttt{12}.
  
\item the \emph{category} \TT{small\_drawer} is associated to the \emph{type}
  value \texttt{34}.
  
\item the \emph{category} \TT{large\_drawer} is associated to the \emph{type}
  value \texttt{35}.
  
\item the \emph{category} \TT{blanket} is associated to the \emph{type}
  value \texttt{74}.
  
\end{itemize}

\begin{table}[h]
\begin{center}
  \begin{tabular}{|c|c|}
    \hline
    Category  & Type  \\
    \hline
    \hline
    \TT{world}  & \texttt{0} \\
    \hline
    \TT{house}  & \texttt{1}\\
    \hline
    \TT{floor}  & \texttt{2}\\
    \hline
    \TT{room}  & \texttt{3}\\
    \hline
    \TT{table}  & \texttt{4}\\
    \hline
    \TT{chair}  & \texttt{6}\\
    \hline
    \TT{bed}  & \texttt{9}\\
    \hline
    \TT{cupboard}  & \texttt{12}\\
    \hline
    \TT{small\_drawer}  & \texttt{34}\\
    \hline
    \TT{large\_drawer}  & \texttt{35}\\
    \hline
    \TT{blanket}  & \texttt{74}\\
    \hline
  \end{tabular}
  \end{center}
  \caption{The lookup table for \emph{geometry categories} and associated \emph{types}.}
  \label{tab:id_map:0}
\end{table}

\pn  The  only  constraint   here  is  that  both  \emph{type}  values
(implemented as  integers) and \emph{category}  labels (implemented as
character strings) are  unique for a given GID  manager. We needs here
some kind  of look-up table with unique  \emph{key/value} pairs (table
\ref{tab:id_map:0}).

\pn  Then   we  give  the   rules  that  describes   the  hierarchical
relationships between different categories  of objects. In our present
domestic  example, we  can  make  explicit the  rules  of our  virtual
domestic world :

\begin{itemize}
  
\item the  whole setup (the  top-level volume) contains one  or more
  objects  of  the  \TT{house}  category  (\emph{houses})  (here  we
  exclude the case  of a top level volume without any  house in it :
  it is indeed of no interest !)
  
\item a \emph{house} contains at least one or several \emph{floors},
  
\item a \emph{floor} contains at least one or several \emph{rooms},
  
\item a \emph{room} contains zero or more objects of the following 
  categories : \TT{chair}, \TT{table}, \TT{bed}, \TT{cupboard},
  
\item a \emph{table} can have zero or only one \emph{small drawer},
  
\item a \emph{cupboard} can have zero or up to 4 \emph{large drawer},
  
\item a \emph{bed} can host zero or more \emph{blankets},
  
\item  a  \emph{blanket}, a  \emph{small  drawer}  or a  \emph{large
  drawer} cannot contains anything;  there are the terminal leafs of
  the hierarchy (they cannot be considered as \emph{containers}).
  
\end{itemize}

\pn The figure \ref{fig:house:1} shows these various categories of objects.

\begin{figure}[h]
\begin{center}
\scalebox{1.}{\input{\pdftextpath/fig_house_1.pdftex_t}}
\end{center}
\caption{Various categories of objects in a domestic world.}\label{fig:house:1}
\end{figure}

\subsection{Different kinds of hierarchical relationships}

\pn This domestic example is a good start to investigate the different
placement  relationships that can  be identified  in such  a virtual
model.  Let's consider the \emph{world} in figure \ref{fig:house:2}.
\begin{figure}[h]
\begin{center}
\scalebox{1.}{\input{\pdftextpath/fig_house_2.pdftex_t}}
\end{center}
\caption{A simple domestic world.}\label{fig:house:2}
\end{figure}
\pn We  have here  only one  house with two  floors. The  ground floor
contains two  rooms, the first floor  has one single  room. The large
room  at ground floor  contains three  chairs and  one table  with one
unique small  drawer.  The small room  at ground floor  has one single
chair.  The unique room at first floor contains one chair and one bed
with one blanket on it. Looks like your place isn't it ?

\pn Now we can make an exhaustive inventory of all the objects
that belongs to this world. We can associate to each of them an unique
\emph{geometry identifier} :

\begin{itemize}

\item the GID of the unique  house here has type value \texttt{1}.  As
  our world can  in principle host several houses,  we must allocate a
  \emph{house  number} to  this particuliar  house. Let's  chose house
  number \texttt{666} (the  Devil's house !). So the  GID of the house
  can be read : ''I'm an  object in category \TT{house} and my address
  is fully defined by the \TT{house\_number}=666''. This should lead to
  the following  format: \verb+[1:666]+.   The depth of  the addresses
  path of the  GID (\texttt{666}) is only 1 because only one  integer value is enough
  to distinguish this \emph{Devil's house} from some possible other houses
  we could add in this virtual world (\emph{Mary's place}, \emph{The Red Lantern}\dots).

\item Let's now consider the ground and first floors. Of course these
  objects share the same \emph{category}, both are \emph{floors}.  The
  only  way to  distinguish  them  is to  allocate  to them  different
  addresses.  The  addresses path  still has to  reflect to  fact that
  they    both   belong    to   the    \emph{Devil's    house}   (with
  \TT{house\_number}=666).   So  thay  must  share  this  hierarchical
  information. Finally  the minimal information  needed to distinguish
  both floor is to  allocate and additionnal \emph{floor number} (thus
  another integer number).
  \begin{itemize}
 
  \item  The ground  floor, or  floor with  number \texttt{0},  can be
    provided with  the following GID: \verb+[2:666.0]+,  which reads :
    ''I'm an object of type  \texttt{2} (so my category is \TT{floor})
    and  I  belong  to   the  house  of  which  \TT{house\_number}  is
    \texttt{666}.  My \TT{floor\_number} is \texttt{0}''

  \item  Using  a  similar  scheme,  the first  floor  is  given  the
    following GID : \verb+[2:666.1]+, which reads : ''I'm an object of
    type \texttt{2} (so my category is \TT{floor}) and I belong to the
    house   of   which   \TT{house\_number}   is   \texttt{666}.    My
    \TT{floor\_number} is \texttt{1}''
  \end{itemize}
\pn So the  depth of the addresses associated to  a \emph{floor} is 2.
In the  geometry mapping terminology in use  in \texttt{geomtools}, we
say  that  the   \TT{floor}  category  \emph{extends}  the  \TT{house}
category   (and   its   \TT{house\_number}  address)   \emph{by}   the
additionnal  \TT{floor\_number} address, leading  to a  two-levels (or
depth) addressing scheme.

\item  Now we  can play  the  same game  for the  \emph{rooms}. It  is
  obvious that  all 3 rooms in  the Devil's house share  the same type
  (set  conventionnaly at \texttt{3})  and belongs  to the  same house
  (\TT{house\_number}=\texttt{666}).     However     thay    can    be
  distinguished     by    two     different     informations:    their
  \TT{floor\_number}   and    a   new   level    of   address:   their
  \TT{room\_number}  with  respect to  the  floor  they  lie in.  This
  enables us to allocate a unique GID to each of them:
  \begin{itemize}
 
  \item Large room at ground floor: GID is \verb+[2:666.0.0]+ where we
    chose  to  start  the  numbering  of  rooms  at  this  floor  with
    \TT{room\_number}=0
 
  \item Small room at ground floor: GID is \verb+[2:666.0.1]+ where we
    chose  the next available  \TT{room\_number}=1 because  the former
    room already used \TT{room\_number}=0.

  \item Unique  room at first floor: GID  is \verb+[2:666.1.0]+ where
    we chose  to start  the numbering  of rooms at  this a  floor with
    \TT{room\_number}=0

  \end{itemize}
So the depth of the addresses  associated to a \emph{room} is 3.  Here
we  say  that the  \TT{room}  category  \emph{extends} the  \TT{floor}
category (and its \TT{house\_number} and \TT{floor\_number} addresses)
\emph{by}  the additionnal  \TT{floor\_number} address,  leading  to a
three-levels (or depth) addressing scheme.

\item  What about  chairs, beds  and tables  ? Again  as one  room can
  contains several of  these objects, we will have  to add additionnal
  number along  the addresses  path to uniquely  identify them  with a
  GID. This gives the following collection of GID:
\begin{itemize}
\item[] \verb+[6:666.0.0.0]+ : the first chair in the large room at ground floor of the Devil's house
  where a \TT{chair\_number}=\texttt{0} value has been appended to the addresses path of the mother
  room with GID \verb+[6:666.0.0]+,
\item[]\verb+[6:666.0.0.1]+ : the second chair in the large room at ground floor of the Devil's house
(appended \TT{chair\_number}=\texttt{1}),
\item[]\verb+[6:666.0.0.2]+ : the third chair in the large room at ground floor of the Devil's house
(appended \TT{chair\_number}=\texttt{2}),
\item[]\verb+[4:666.0.0.0]+ : the unique table in the large room at ground floor of the Devil's house
(appended \TT{table\_number}=\texttt{0}),
\item[]\verb+[6:666.0.1.0]+ : the unique chair in the small room at ground floor of the Devil's house
(appended \TT{chair\_number}=\texttt{0}),
\item[]\verb+[6:666.1.0.0]+ : the unique chair in the unique room at first floor of the Devil's house
(appended \TT{chair\_number}=\texttt{0}),
\item[]\verb+[9:666.1.0.0]+ : the unique bed in the unique room at first floor of the Devil's house
(appended \TT{bed\_number}=\texttt{0}),
\end{itemize}
\end{itemize}

\pn Should we add  another chair in the room at first  floor and also a cupboard
in the large room at ground floor (figure \ref{fig:house:3}) ?  
\begin{figure}[h]
\begin{center}
\scalebox{1.0}{\input{\pdftextpath/fig_house_3.pdftex_t}}
\end{center}
\caption{More objects in the domestic world.}\label{fig:house:3}
\end{figure}
\pn No problem ! We allocate to this  new chair the GID :
\begin{center}
\verb+[6:666.1.0.1]+
\end{center} 
\pn where we have appended the \TT{chair\_number}=\texttt{1}. 
We also allocate to the  new cupboard the GID :
\begin{center}
\verb+[12:666.0.0.0]+
\end{center} 
\pn where we use the  type \texttt{12} that corresponds to 
the \TT{cupboard} category and
extends   the  GID   of  its   mother  room   (\verb+666.0.0+)   by  a
\TT{cupboard\_number}. The \TT{cupboard\_number} arbitrarily starts  at \texttt{0}  (in  case we
would add more and more cupboards in this room : \TT{cupboard\_number}=\texttt{1}, \texttt{2}\dots).

\pn  We have  clearly identified  one kind  of  hierarchy relationship
between  different  \emph{categories}  of   objects  in  term  of  the
\emph{extension}  of the  addresses path  by some  additionnal address
value(s)  (integer   numbers).  These  appended   numbers  enable  the
distinction between different objects of  the same type located in the
same  container object.  This concept  is closed  to  the \emph{mother
  volume} concept in  a hierachical geometry setup and  is indeed well
adapted to represent such geometry placement relationship.

\pn  Now let's  consider another  situation !  As the  placement rules
above  state it  : ''a  table object  can have  one and  only  one small
drawer''. What  does it mean  ? It means  that to absolutely  identify a
specific  drawer, we  only need  to know  to which  table  it belongs.
There is no need for  additionnal information to \emph{locate} it.  We
say in this case  that the \TT{small\_drawer} category \emph{inherits}
the addressing scheme of the  \TT{table} category.  So, in the case of
the drawer plugged  in the ''unique table in the  large room at ground
floor of the Devil's house'', we can build its GID :
\begin{center}
\verb+[34:666.0.0.0]+ 
\end{center}
\pn which shares  the same 4-levels addresses path  of its host/mother
table,  but  differs only  by  the  type  identifier (\texttt{34}  for
the  \TT{small\_drawer} category in place of \texttt{4} for the
\TT{table} category).

\pn This  has to  be compared with  the way  we should treat  the four
\TT{large\_drawer} objects that belong  to the newly added cupboard in
the  same  room.  In  this   case  we  must  use  a  \emph{extend  by}
relationship because  an additionnal \TT{drawer\_number}  is needed to
distinguish the 4 possible drawers  in a cupboard. So, in this virtual
world,  the GID of  a \TT{small\_drawer}  has some  4-levels addresses
path but the  GID of a \TT{large\_drawer} has  some 5-levels addresses
path. This  cleary shows that in  this approach, the  structure of the
numbering scheme attached to an object is not an intrinsic property of
the (geoemtrical)  nature of the  object but an intrinsic  property of
its relationship  with its environment (the  way it is  placed in some
mother object).  We can imagine  that there is no  physical difference
(shape, color, dimensions) between the model of the drawer inserted in
a table and  the model of the drawer inserted in  a cupboard. They can
have the  same physical  properties and thus  share a  common physical
description. However, the way \emph{we  use} a generic drawer 
defines another level of properties:
\begin{itemize}
\item a drawer for a table  must belong to an object of the \TT{table}
  category and can contains  objects of the \TT{pencil} category; this
  defines the \TT{small\_drawer} category,
\item  a  drawer for  a  cupboard  must belong  to  an  object of  the
  \TT{cupboard} category  and can  contains objects of  the \TT{shirt}
  or \TT{tights} category; this defines the \TT{large\_drawer} category.
\end{itemize}

\subsection{Numbering scheme is an arbitrary choice within an application}

\pn  The above  considerations  enlight the  fact  that the  so-called
\emph{geometry  category} does not  reflect the  nature of  a physical
object with its physical properties  but the way it is \emph{inserted}
and/or \emph{filled} in its environment made of other objects.

\pn This can be seen on figures below where we can attribute different
numbering  schemes  to  the  same  virtual world,  depending  on  some
arbitrary way of thinking :

\begin{itemize}
\item Case 1 :
   Here  all  box  objects  are  considered  to  belong  to  the
    \TT{any\_box} category (with  \texttt{type=0}). 

    \begin{tabular}{|c|c|}
      \hline
      Category  & Type  \\
      \hline
      \hline
      \TT{world}  & \texttt{0} \\
      \hline
      \TT{any\_box}  & \texttt{1}\\
      \hline
    \end{tabular}
    
    \pn We  don't take  into account the  mother/daughter relationship
    between the  red and the green  boxes, then between  the green and
    the  blue box. This  corresponds to  a \emph{flat}  geometry model
    where we don't need to  know anything about the internal hierarchy
    of  the geometry  setup.  So  we don't  need to  set  any specific
    \emph{hierarchy rules}  but one  : the top-level  volume (world)
    contains some objects of category \TT{any\_box}.

    \pn We get 3 GIDs with the same \emph{type} and some distinct
    \TT{object\_number} (running from \texttt{0} to \texttt{2}) :
    \begin{center}
      \scalebox{1.0}{\input{\pdftextpath/fig_alt_0.pdftex_t}}
    \end{center}

    \pn Note  that in this case,  the physical structure  of the setup
    does rely on several level of nested volumes, but we simply do not
    consider this fact while choosing the abstract numbering scheme
    and building the GIDs.


\item Case 2 :
 Here  we take into account the hierarchical relationships
    between the three  box  objects. The lookup table now defines
    3 different \emph{categories} with 3 associated \emph{types} :

    \begin{tabular}{|c|c|}
      \hline
      Category  & Type  \\
      \hline
      \hline
      \TT{world}  & \texttt{0} \\
      \hline
      \TT{big\_box}  & \texttt{10}\\
      \hline
      \TT{medium\_box}  & \texttt{11}\\
      \hline
      \TT{small\_box}  & \texttt{12}\\
      \hline
    \end{tabular}

    \pn Some \emph{hierarchy rules} are also given :

    \begin{itemize}

    \item  ''the  top-level volume (world)  contains some
    objects of category \TT{big\_box}'',

    \item  ''an object  of category \TT{big\_box}  contains some
    objects of category \TT{medium\_box}'',

    \item  ''an object  of category \TT{medium\_box}  contains some
    objects of category \TT{small\_box}''.

    \item  ''an object  of category \TT{small\_box} cannot contain anything.

    \end{itemize}

    \pn The resulting GIDs associated to the 3 boxes is now :
    
    \begin{center}
      \scalebox{1.0}{\input{\pdftextpath/fig_alt_1.pdftex_t}}
    \end{center}

    \pn GIDs are not only distinct by their \emph{type} but also
    by the depth of their addresses path, with reflect the hierachical
    nature of thier respective placements. Of course, this way of doing 
    is richer than the \emph{flat} model.
\end{itemize}

\pn Finally,  it turns  out that the  choice for the  numbering scheme
strategy  (the  \emph{mapping})  is  the  affair of  the  user  and/or
application  and  is   not  fixed  by  the  physical   nature  of  the
objects/components  of  the  virtual  model, though  it  is  generally
related to it.

\subsection{Limitations of this approach}

\pn Although  the above hierarchical  approach is rather  powerful and
can be  naturally implemented to  design the numbering scheme  of many
different virtual geometry setups -- usually designed/interpreted with
the  hierarchical   (mother/daughter)  approach  --   there  are  some
(realistic)  cases that  cannot be  addressed  or with  at least  some
difficulties or limitations.

\pn The first of these case is the \emph{door}. Let's consider again a
domestic case with two adjacent rooms on the same floor of some house.

\begin{center}
  \scalebox{1.0}{\input{\pdftextpath/fig_house_4.pdftex_t}}
\end{center}


\pn In the real world, we generally use doors to move from one room to
the other.  In the  physical world,  a door can  be considered  as yet
another kind of  object in the domectic environment.   For example, it
is made of wood, has generally a rectangular shape, some dimensions, a
mass, can  be painted with  some arbitrary colour\dots exactly  like a
chair.  However   the  way  a   door  is  \emph{inserted}   within  the
\emph{geometry} of  our world is not  obvious on the point  of view of
the  natural  \emph{house/floor/room/furniture}  hierarchy.  The  main
question is : ''Of which room  a specific door belongs to ?''.  It was
easy to  answer this question  for a chair\dots  So we meet  here some
issues to use  our hierarchy model and clearly we  will have to invent
new concepts to handle such a situation.

\pn Another interesting case has to do with overlapping and/or complex
assembly of objects. The figure below shows two of these special cases
:
\begin{center}
  \scalebox{1.0}{\input{\pdftextpath/fig_house_5.pdftex_t}}
\end{center}

\pn Here  we have a  red screw  that has some  parts of its  volume in
different rooms,  another part is  inserted in the grey  region (house
structure);  its  peak  is even  inserted  in  the  brown box  on  the
left.  The other  special case  concerns the  yellow  region (labelled
''crime scene'' that  is splitted in two distinct  rooms. Because of
the overlapping placements of these  objects, it is again not possible
to address the problem in term of a simple hierarchical relationship.

\pn  In software applications  we use  for geometry  modelling (GEANT4
simulation, GDML  modelization, visualization software),  such complex
situations are generally forbiden  by the underlying geometry modeling
engine, despite they are not rare in the real world.

%% end of gid_concepts.tex


\pagebreak
%% id_mgr.tex

\section{The GID manager}

\subsection{Requirements to setup a numbering scheme policy}

\pn Within \texttt{geomtools},  a class named \verb+geomtools::id_mgr+
has been designed to activate  a numbering scheme policy for a virtual
geometry setup and enable the  interpretation of the GID associated to
some volumes  in this setup. Following the concepts we have investigated
in the previous section, a \verb+geomtools::id_mgr+
instance is initialized with some special directives that specify :

\begin{itemize}

\item  a  list  of  \emph{geometry  categories}  with  their  uniquely
  associated \emph{type}  values and also a human  readable label (the
  \emph{category} name or  label). 

\item for each \emph{geometry category}, some rules that describes the
  hierarchical relationships (\emph{inherit}/\emph{extend to}) 
  with the other categories of objects.

\end{itemize}

\pn  In the current  implementation, these  configuration informations
are   stored  in  a   \\  \texttt{datatools::utils::multi\_properties}
container  \footnote{seed  the  ''\emph{Using  container  objects  in}
  \texttt{datatool}''  tutorial  from  the \texttt{datatools}  program
  library.}  and used to create an internal lookup table (based on the
\texttt{std::map}  class).   Each   section  of  this  multi-container
concerns a specific \emph{geometry  category} of which the \emph{name}
is the primary  key to access useful informations.   The section stores
some configuration parameters  that reflect the \emph{hierarchy rules}
between categories.


\subsection{Configuration file format}

\pn   Practically,   the  \texttt{datatools::utils::multi\_properties}
configuration  container is  saved as  an ASCII  file using a human
friendly readable format.  Let's  consider the domestic world explored
in the previous sections to illustrate the syntax of this file !

First of all, the header of the file must contain the following
meta comments:

\begin{ShellVerbatim}
#@description The description of geometry category in the domestic model
#@key_label   "category"
#@meta_label  "type"
\end{ShellVerbatim}
%%$
\pn Note that the description line is optionnal and the \emph{key} and
\emph{label} ones are mandatory.

Then we must add the default top-level category, namely \TT{world} which
implements a only-one level addresses path with a value labelled \TT{world}.
The \texttt{type} is set at \texttt{0} by convention of the library :
\begin{ShellVerbatim}
[category="world" type="0"]
addresses : string[1] = "world"
\end{ShellVerbatim}
%%$

This design could allow in the future to run simultaneously
some \emph{parallel} virtual world, each having a different 
\TT{world} number, but sharing the same \TT{world} category.

Now it's time to enter the description of the categories
of domestic objects.
We start with the \TT{house} category to which we allocate
the type \texttt{1} and a 1-level addresses path with a single 
\TT{house\_number} value:
\begin{ShellVerbatim}
[category="house" type="1"]
addresses : string[1] = "house_number"
\end{ShellVerbatim}
%%$
Note that the category label \TT{house}, as well as
the address label \TT{house\_number}, will be usable by the user through
a human friendly interface. This will allow people not to memorize
all the integer numbers used in this system and will provide simple methods
to retrieve useful geometry informations through character strings.

Now we are done with houses, let's define the rules for the \TT{floor} category
with type \texttt{2} :
\begin{ShellVerbatim}
[category="floor" type="2"]
extends : string    = "house"
by      : string[1] = "floor_number"
\end{ShellVerbatim}
%%$
which tells (\texttt{extends}) that any object of the \TT{floor} category 
is contained by another object of the \TT{house} category.
More, as any house can contains several floors, we need some additionnal 
one-level depth appended to the full addresses path. The \texttt{by}
directive thus defines the additional \TT{floor\_number}.
Given this rule, the \texttt{id\_mgr} instance will
automatically use a two-level addresses path for any floor object:
the first integer value is the   \TT{house\_number} inherited from
the mother house object, followed by
a second integer which corresponds to the \TT{floor\_number}.
Example: \verb+[2:666.9]+ means the floor number \texttt{9} 
in house number \texttt{666}.

The numbering scheme for the \TT{room}, \TT{table}, \TT{chair}, \TT{bed} 
and \TT{cupboard}  categories is similar:
\begin{ShellVerbatim}
[category="room" type="3"]
extends : string    = "floor"
by      : string[1] = "room_number"
\end{ShellVerbatim}
%%$
this gives GID like:  \verb+[3:666.9.7]+ means the room number \texttt{7}
on floor number \texttt{9} 
in house number \texttt{666}.

\begin{ShellVerbatim}
[category="table" type="4"]
extends : string    = "room"
by      : string[1] = "table_number"

[category="chair" type="6"]
extends : string    = "room"
by      : string[1] = "chair_number"

[category="bed" type="9"]
extends : string    = "room"
by      : string[1] = "bed_number"

[category="cupboard" type="12"]
extends : string    = "room"
by      : string[1] = "cupboard_number"
\end{ShellVerbatim}
%%$
leading respectively to GIDs like:
\verb+[4:666.9.7.0]+,
\verb+[6:666.9.7.2]+,
\verb+[9:666.9.7.1]+ and \verb+[12:666.9.7.0]+.

The case of the \TT{small\_drawer} is special because
there can be only one of such object that belongs to a given table.
This is specified with:
\begin{ShellVerbatim}
[category="small_drawer" type="34"]
inherits : string    = "table"
\end{ShellVerbatim}
%%$
and gives GIDs like: \verb+[34:666.9.7.0]+ for the unique
drawer of table \verb+[4:666.9.7.0]+.

Drawer for cupboard use the \emph{extend by} technique:
\begin{ShellVerbatim}
[category="large_drawer" type="35"]
extends : string    = "cupboard"
by      : string[1] = "drawer_number"
\end{ShellVerbatim}
%%$
and gives GIDs like: \verb+[35:666.9.7.0.2]+ for the
drawer numbered \texttt{2} in the \verb+[34:666.9.7.0]+ cupboard.

Of course more \emph{category records} can be added to the file
if it is needed:
\begin{ShellVerbatim}
[category="fork" type="105"]
extends : string    = "small_drawer"
by      : string[1] = "fork_number"

[category="spoon" type="106"]
extends : string    = "small_drawer"
by      : string[1] = "spoon_number"

[category="knife" type="107"]
extends : string    = "small_drawer"
by      : string[1] = "knife_number"
\end{ShellVerbatim}

There is also a special useful case. Suppose we are allowed
to place a special type of jewelry box in any drawer of a cupboard.
We then write a new hierarchy rule :
\begin{ShellVerbatim}
[category="jewelry_box" type="110"]
extends : string    = "large_drawer"
by      : string[1] = "box_number"
\end{ShellVerbatim}
So far so good. Assume now that each jewelry box can contains
up to 4 jewels that can only be placed in one of the
four available compartments :

    \begin{center}
      \scalebox{1.0}{\input{\pdftextpath/fig_chest_0.pdftex_t}}
    \end{center}
As it can be seen on the above figure,
each compartment can be naturally located in a abstract
two-dimensional space using the value of its row number
\textbf{and} the value of its column number.
The addresses path of a given jewel will thus be determined 
first by the addresses path from the box it lies in, then
with the information of both row and column numbers.
Such a rule can be written with : 
\begin{ShellVerbatim}
[category="jewel" type="111"]
extends : string    = "jewelry_box"
by      : string[2] = "row_number" "column_number"
\end{ShellVerbatim}
\pn where  two additionnal numbers (with their  human readable labels)
have been appended (one-shot extension) to the address path. Of course
the  choice   of  ordering   the  row  and   column  numbers   is  set
conventionally. At  least we know how  to build the GID  of any jewel.
Example: \verb+[111:666.9.7.0.2.0.1.0]+ reads : \\
 ''I'm a  (beautiful) jewel
hidden in the top/left (\TT{row}=\texttt{1}, \TT{column}=\texttt{0})
compartment  of the  unique jewelry  box stored  in the  drawer number
\texttt{2} of the  only cupboard in the room  number \texttt{7} on the
9th floor of the Devil's house''. Now you have its GID, 
I guess you are able to steal the treasure !

\pagebreak
\subsection{Code snippets}

\subsubsection{Initializing a GID manager object}

\pn The  following sample program illustrates the  initialization of a
\emph{GID     manager},      using     an     instance      of     the
\texttt{geomtools::id\_mgr}          class          using          the
\TT{domestic\_categories.lis} configuration file.

\VerbatimInput[frame=single,
numbers=left,
numbersep=2pt,
firstline=1,
fontsize=\small,
showspaces=false]{\codingpath/domestic_1.cxx}

\pagebreak
The \TT{domestic\_categories.lis} file contains the description
of the domestic categories we have proposed in the previous section :

\VerbatimInput[frame=single,
numbers=left,
numbersep=2pt,
firstline=1,
fontsize=\small,
showspaces=false]{\codingpath/domestic_categories.lis}

\pagebreak
The program  first parses the  file. It constructs an  internal lookup
table  that  stores  all  the  informations  needed  to  describe  the
hierarchical  relationships between  all kind  of objects.  Finally it
prints the contents of the categories lookup table :
\VerbatimInput[frame=single,
numbers=left,
numbersep=2pt,
firstline=1,
fontsize=\small,
showspaces=false]{\codingpath/domestic_1.out}

\pagebreak
\subsubsection{Creating some GIDs following the hierarchy rules of a GID manager object}

\pn     This    new     sample    program     uses     the    previous
\TT{domestic\_categories.lis}   configuration  file.   It   creates  a
specific GID in a  given \emph{category} (\TT{room}) through the human
friendly  interface of the  \texttt{geomtools::id\_mgr} class.  Here a
first GID is  created from scratch then the GID of  a parent object is
automatically extracted and finally the  GID of a daughter object in a
given category is created :

\VerbatimInput[frame=single,
numbers=left,
numbersep=2pt,
firstline=1,
fontsize=\small,
showspaces=false]{\codingpath/domestic_2.cxx}

\pn The program prints :
\VerbatimInput[frame=single,
numbers=left,
numbersep=2pt,
firstline=1,
fontsize=\small,
showspaces=false]{\codingpath/domestic_2.out}
\pagebreak

\subsubsection{Creating a large number of GIDs with optimized techniques}

\pn   The    next   sample    program   still   uses    the   previous
\TT{domestic\_categories.lis} configuration file to initialize the GID
manager.   It  accesses  to  the  database  of  categories  through  a
\texttt{geomtools::id\_mgr::category\_info}  object  fetched from  the
GID  manager.   It then  uses  the  available  informations about  the
hierarchy rules to manipulate GID  objects at very low-level, i.e.  by
direct manipulation of the type value and the values/subaddress of the
addresses path at  any level :

\VerbatimInput[frame=single,
numbers=left,
numbersep=2pt,
firstline=1,
fontsize=\small,
showspaces=false]{\codingpath/domestic_3.cxx}

\pn The program prints :
\VerbatimInput[frame=single,
numbers=left,
numbersep=2pt,
firstline=1,
fontsize=\small,
showspaces=false]{\codingpath/domestic_3.out}
 
\pn Such a technique should be  favored when one needs to manipulate a
large number of  GIDs or even a few GIDs  very frequently.  Indeed the
human-friendly    methods     provided    by    the     GID    manager
(\texttt{geomtools::id\_mgr}) class are  based on searching algorithms
in various associative containers keyed by \texttt{string} objects. If
these methods are often used, some performance issues are expected. It
is  thus  more  efficient  to  directly use  the  integer  values  for
\emph{types} and \emph{addresses}  indexes, rather than human friendly
string labels.  Here the human-friendly  methods are used once  at the
beginning of  the program  to retrieve useful  addressing informations
stored as  integer values  (category \emph{types} and  address index);
then it is straightforward to reuse this addressing parameters a large
number of  times, without further  request through the  human friendly
interface of the GID manager.


%% end of id_mgr.tex


\pagebreak

%% mapping.tex

\section{Geometry mapping and associated tools}

\pn In the previous section, we have presented the basic concepts (GID, 
hierarchy rules, GID manager) used in the \texttt{geomtools} library.
Now it is time to introduce some high-level functionnalities that can be
implemented on top of these low-level concepts : \emph{geometry mapping}
and \emph{locators}.

\subsection{Geometry mapping}

The key concept of \emph{geometry mapping} is to allow the users to benefit
of some automated (or semi-automated) database of all (or part of) the 
objects that belongs to a geometry hierarchical setup. Such a database
will naturally use the object's GID as the primary keys for accessing some 
meta-data associated to an object.

As seen  in the previous sections,  it is possible to  define some non
ambiguous  \emph{hierarchy  rules}   to  reflect  the  mother/daughter
relationships between objects  of a virtual geometry setup.  This is a
task  for  the   GID  manager  object,  which  is   available  in  the
library. 

However,  when designing the  numbering scheme,  there are  still many
arbitrary choices that have to  be made by the architect/developper of
the geometry  model :  ''Do we start  the values of  subaddresses from
\texttt{0} or \texttt{1} when  several replicates of some category are
placed  in  a  mother volume  ?'',  ''What  value  is chosen  for  the
subaddress  of the  left part  of  this tracking  chamber ?'',  ''What
integer  values are  associated to  the  North, South,  West and  East
directions ?''\dots  So we need some additional  conventions and rules
to finalize the  addressing scheme. Then we will  be able to establish
and build the full list of GIDs that makes sense in our application.

If we consider the above \emph{domestic} virtual world, we have 
implicitely used such rules on top of the hierarchy rules. We now need
some tools to automatically inform the geometry model of these rules.
Let's build such a virtual setup with the tools provided
by geoemtools. Then we will see how to enrich this model
with special \emph{mapping directives}. 

\subsection{A toy model}

%% end of mapping.tex


%% %% Material specific to memos in some "Software" category :
%% \begin{ShellVerbatim}
%% bash-3.2$ ls -l
%% total 20
%% -rw-r--r-- 1 mauger mauger    0 2011-11-16 10:16 ChangeLog
%% drwxr-xr-x 2 mauger mauger 4096 2011-11-16 10:24 doc
%% drwxr-xr-x 2 mauger mauger 4096 2011-11-16 15:25 lib
%% -rw-r--r-- 1 mauger mauger 1180 2011-11-16 10:24 README
%% drwxr-xr-x 2 mauger mauger 4096 2011-11-16 16:06 resources
%% drwxr-xr-x 2 mauger mauger 4096 2011-11-16 15:42 scripts
%% \end{ShellVerbatim}
%% %%$

%% \pn \texttt{MakeMemo} path is:
%% \begin{PathVerbatim}
%% /sps/nemo/scratch/mauger/sw/Memos/MakeMemo/trunk
%% \end{PathVerbatim}
%% %%$

%% \pn Code snippet can be included (see figure \ref{fig:code_snippet}).

%% \begin{figure}
%% \VerbatimInput[frame=single,
%% numbers=left,
%% numbersep=2pt,
%% firstline=1,
%% fontsize=\small,
%% showspaces=false]{\codingpath/code_snippet.cxx}
%% \caption{Some code snippet.}\label{fig:code_snippet}
%% \end{figure}

%% \pn Also we can print verbatim code on-the-fly with special references 
%% to lines \ref{vrb:important} and \ref{vrb:alt} :
%% \begin{CppVerbatim}
%% double pi = 3.14159;
%% double x = 11.3 * CLHEP::mm; // important \label{vrb:important}
%% double y = 2.3 * CLHEP::cm;
%% double t = 1.2 * CLHEP::ns;  // also important \label{vrb:alt}
%% \end{CppVerbatim}
%% %%$

%% \pn See also how to include a combined PDF/LaTeX figure from
%% XFig (see figure \ref{fig:sketch_package})

%% \begin{figure}
%% \begin{center}
%% \scalebox{0.75}{\input{\pdftextpath/sketch_package.pdftex_t}}
%% \end{center}
%% \caption{The figure caption.}\label{fig:sketch_package}
%% \end{figure}

\section{Conclusion}

\pn The conclusion\dots

%% \pn Here is the SuperNEMO logo as a plain PDF image :
%% \begin{center}
%% \includegraphics[width=0.25\linewidth]{\imagepath/snemo_logo.pdf}
%% \end{center}

\pn This is the end.

\end{document}
%%%%%%%%%%%%%%%%%%%%%%%%%%%%%%%%%%%%%%%%%%%%%%%%%%%%%%%%%%%%%%%%

%% end of GeometryMappingTutorial.tex
