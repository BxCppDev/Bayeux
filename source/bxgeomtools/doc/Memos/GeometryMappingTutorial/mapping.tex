
%% mapping.tex

\section{Geometry mapping and associated tools}

\pn In the previous section, we have presented the basic concepts (GID, 
hierarchy rules, GID manager) used in the \texttt{geomtools} library.
Now it is time to introduce some high-level functionnalities that can be
implemented on top of these low-level concepts : \emph{geometry mapping}
and \emph{locators}.

\subsection{Geometry mapping}

The key concept of \emph{geometry mapping} is to allow the users to benefit
of some automated (or semi-automated) database of all (or part of) the 
objects that belongs to a geometry hierarchical setup. Such a database
will naturally use the object's GID as the primary keys for accessing some 
meta-data associated to an object.

As seen  in the previous sections,  it is possible to  define some non
ambiguous  \emph{hierarchy  rules}   to  reflect  the  mother/daughter
relationships between objects  of a virtual geometry setup.  This is a
task  for  the   GID  manager  object,  which  is   available  in  the
library. 

However,  when designing the  numbering scheme,  there are  still many
arbitrary choices that have to  be made by the architect/developper of
the geometry  model :  ''Do we start  the values of  subaddresses from
\texttt{0} or \texttt{1} when  several replicates of some category are
placed  in  a  mother volume  ?'',  ''What  value  is chosen  for  the
subaddress  of the  left part  of  this tracking  chamber ?'',  ''What
integer  values are  associated to  the  North, South,  West and  East
directions ?''\dots  So we need some additional  conventions and rules
to finalize the  addressing scheme. Then we will  be able to establish
and build the full list of GIDs that makes sense in our application.

If we consider the above \emph{domestic} virtual world, we have 
implicitely used such rules on top of the hierarchy rules. We now need
some tools to automatically inform the geometry model of these rules.
Let's build such a virtual setup with the tools provided
by geoemtools. Then we will see how to enrich this model
with special \emph{mapping directives}. 

\subsection{A toy model}

%% end of mapping.tex
