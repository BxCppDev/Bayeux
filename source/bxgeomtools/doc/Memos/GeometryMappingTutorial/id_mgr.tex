%% id_mgr.tex

\section{The GID manager}

\subsection{Requirements to setup a numbering scheme policy}

\pn Within \texttt{geomtools},  a class named \verb+geomtools::id_mgr+
has been designed to activate  a numbering scheme policy for a virtual
geometry setup and enable the  interpretation of the GID associated to
some volumes  in this setup. Following the concepts we have investigated
in the previous section, a \verb+geomtools::id_mgr+
instance is initialized with some special directives that specify :

\begin{itemize}

\item  a  list  of  \emph{geometry  categories}  with  their  uniquely
  associated \emph{type}  values and also a human  readable label (the
  \emph{category} name or  label). 

\item for each \emph{geometry category}, some rules that describes the
  hierarchical relationships (\emph{inherit}/\emph{extend to}) 
  with the other categories of objects.

\end{itemize}

\pn  In the current  implementation, these  configuration informations
are   stored  in  a   \\  \texttt{datatools::utils::multi\_properties}
container  \footnote{seed  the  ''\emph{Using  container  objects  in}
  \texttt{datatool}''  tutorial  from  the \texttt{datatools}  program
  library.}  and used to create an internal lookup table (based on the
\texttt{std::map}  class).   Each   section  of  this  multi-container
concerns a specific \emph{geometry  category} of which the \emph{name}
is the primary  key to access useful informations.   The section stores
some configuration parameters  that reflect the \emph{hierarchy rules}
between categories.


\subsection{Configuration file format}

\pn   Practically,   the  \texttt{datatools::utils::multi\_properties}
configuration  container is  saved as  an ASCII  file using a human
friendly readable format.  Let's  consider the domestic world explored
in the previous sections to illustrate the syntax of this file !

First of all, the header of the file must contain the following
meta comments:

\begin{ShellVerbatim}
#@description The description of geometry category in the domestic model
#@key_label   "category"
#@meta_label  "type"
\end{ShellVerbatim}
%%$
\pn Note that the description line is optionnal and the \emph{key} and
\emph{label} ones are mandatory.

Then we must add the default top-level category, namely \TT{world} which
implements a only-one level addresses path with a value labelled \TT{world}.
The \texttt{type} is set at \texttt{0} by convention of the library :
\begin{ShellVerbatim}
[category="world" type="0"]
addresses : string[1] = "world"
\end{ShellVerbatim}
%%$

This design could allow in the future to run simultaneously
some \emph{parallel} virtual world, each having a different 
\TT{world} number, but sharing the same \TT{world} category.

Now it's time to enter the description of the categories
of domestic objects.
We start with the \TT{house} category to which we allocate
the type \texttt{1} and a 1-level addresses path with a single 
\TT{house\_number} value:
\begin{ShellVerbatim}
[category="house" type="1"]
addresses : string[1] = "house_number"
\end{ShellVerbatim}
%%$
Note that the category label \TT{house}, as well as
the address label \TT{house\_number}, will be usable by the user through
a human friendly interface. This will allow people not to memorize
all the integer numbers used in this system and will provide simple methods
to retrieve useful geometry informations through character strings.

Now we are done with houses, let's define the rules for the \TT{floor} category
with type \texttt{2} :
\begin{ShellVerbatim}
[category="floor" type="2"]
extends : string    = "house"
by      : string[1] = "floor_number"
\end{ShellVerbatim}
%%$
which tells (\texttt{extends}) that any object of the \TT{floor} category 
is contained by another object of the \TT{house} category.
More, as any house can contains several floors, we need some additionnal 
one-level depth appended to the full addresses path. The \texttt{by}
directive thus defines the additional \TT{floor\_number}.
Given this rule, the \texttt{id\_mgr} instance will
automatically use a two-level addresses path for any floor object:
the first integer value is the   \TT{house\_number} inherited from
the mother house object, followed by
a second integer which corresponds to the \TT{floor\_number}.
Example: \verb+[2:666.9]+ means the floor number \texttt{9} 
in house number \texttt{666}.

The numbering scheme for the \TT{room}, \TT{table}, \TT{chair}, \TT{bed} 
and \TT{cupboard}  categories is similar:
\begin{ShellVerbatim}
[category="room" type="3"]
extends : string    = "floor"
by      : string[1] = "room_number"
\end{ShellVerbatim}
%%$
this gives GID like:  \verb+[3:666.9.7]+ means the room number \texttt{7}
on floor number \texttt{9} 
in house number \texttt{666}.

\begin{ShellVerbatim}
[category="table" type="4"]
extends : string    = "room"
by      : string[1] = "table_number"

[category="chair" type="6"]
extends : string    = "room"
by      : string[1] = "chair_number"

[category="bed" type="9"]
extends : string    = "room"
by      : string[1] = "bed_number"

[category="cupboard" type="12"]
extends : string    = "room"
by      : string[1] = "cupboard_number"
\end{ShellVerbatim}
%%$
leading respectively to GIDs like:
\verb+[4:666.9.7.0]+,
\verb+[6:666.9.7.2]+,
\verb+[9:666.9.7.1]+ and \verb+[12:666.9.7.0]+.

The case of the \TT{small\_drawer} is special because
there can be only one of such object that belongs to a given table.
This is specified with:
\begin{ShellVerbatim}
[category="small_drawer" type="34"]
inherits : string    = "table"
\end{ShellVerbatim}
%%$
and gives GIDs like: \verb+[34:666.9.7.0]+ for the unique
drawer of table \verb+[4:666.9.7.0]+.

Drawer for cupboard use the \emph{extend by} technique:
\begin{ShellVerbatim}
[category="large_drawer" type="35"]
extends : string    = "cupboard"
by      : string[1] = "drawer_number"
\end{ShellVerbatim}
%%$
and gives GIDs like: \verb+[35:666.9.7.0.2]+ for the
drawer numbered \texttt{2} in the \verb+[34:666.9.7.0]+ cupboard.

Of course more \emph{category records} can be added to the file
if it is needed:
\begin{ShellVerbatim}
[category="fork" type="105"]
extends : string    = "small_drawer"
by      : string[1] = "fork_number"

[category="spoon" type="106"]
extends : string    = "small_drawer"
by      : string[1] = "spoon_number"

[category="knife" type="107"]
extends : string    = "small_drawer"
by      : string[1] = "knife_number"
\end{ShellVerbatim}

There is also a special useful case. Suppose we are allowed
to place a special type of jewelry box in any drawer of a cupboard.
We then write a new hierarchy rule :
\begin{ShellVerbatim}
[category="jewelry_box" type="110"]
extends : string    = "large_drawer"
by      : string[1] = "box_number"
\end{ShellVerbatim}
So far so good. Assume now that each jewelry box can contains
up to 4 jewels that can only be placed in one of the
four available compartments :

    \begin{center}
      \scalebox{1.0}{\input{\pdftextpath/fig_chest_0.pdftex_t}}
    \end{center}
As it can be seen on the above figure,
each compartment can be naturally located in a abstract
two-dimensional space using the value of its row number
\textbf{and} the value of its column number.
The addresses path of a given jewel will thus be determined 
first by the addresses path from the box it lies in, then
with the information of both row and column numbers.
Such a rule can be written with : 
\begin{ShellVerbatim}
[category="jewel" type="111"]
extends : string    = "jewelry_box"
by      : string[2] = "row_number" "column_number"
\end{ShellVerbatim}
\pn where  two additionnal numbers (with their  human readable labels)
have been appended (one-shot extension) to the address path. Of course
the  choice   of  ordering   the  row  and   column  numbers   is  set
conventionally. At  least we know how  to build the GID  of any jewel.
Example: \verb+[111:666.9.7.0.2.0.1.0]+ reads : \\
 ''I'm a  (beautiful) jewel
hidden in the top/left (\TT{row}=\texttt{1}, \TT{column}=\texttt{0})
compartment  of the  unique jewelry  box stored  in the  drawer number
\texttt{2} of the  only cupboard in the room  number \texttt{7} on the
9th floor of the Devil's house''. Now you have its GID, 
I guess you are able to steal the treasure !

\pagebreak
\subsection{Code snippets}

\subsubsection{Initializing a GID manager object}

\pn The  following sample program illustrates the  initialization of a
\emph{GID     manager},      using     an     instance      of     the
\texttt{geomtools::id\_mgr}          class          using          the
\TT{domestic\_categories.lis} configuration file.

\VerbatimInput[frame=single,
numbers=left,
numbersep=2pt,
firstline=1,
fontsize=\small,
showspaces=false]{\codingpath/domestic_1.cxx}

\pagebreak
The \TT{domestic\_categories.lis} file contains the description
of the domestic categories we have proposed in the previous section :

\VerbatimInput[frame=single,
numbers=left,
numbersep=2pt,
firstline=1,
fontsize=\small,
showspaces=false]{\codingpath/domestic_categories.lis}

\pagebreak
The program  first parses the  file. It constructs an  internal lookup
table  that  stores  all  the  informations  needed  to  describe  the
hierarchical  relationships between  all kind  of objects.  Finally it
prints the contents of the categories lookup table :
\VerbatimInput[frame=single,
numbers=left,
numbersep=2pt,
firstline=1,
fontsize=\small,
showspaces=false]{\codingpath/domestic_1.out}

\pagebreak
\subsubsection{Creating some GIDs following the hierarchy rules of a GID manager object}

\pn     This    new     sample    program     uses     the    previous
\TT{domestic\_categories.lis}   configuration  file.   It   creates  a
specific GID in a  given \emph{category} (\TT{room}) through the human
friendly  interface of the  \texttt{geomtools::id\_mgr} class.  Here a
first GID is  created from scratch then the GID of  a parent object is
automatically extracted and finally the  GID of a daughter object in a
given category is created :

\VerbatimInput[frame=single,
numbers=left,
numbersep=2pt,
firstline=1,
fontsize=\small,
showspaces=false]{\codingpath/domestic_2.cxx}

\pn The program prints :
\VerbatimInput[frame=single,
numbers=left,
numbersep=2pt,
firstline=1,
fontsize=\small,
showspaces=false]{\codingpath/domestic_2.out}
\pagebreak

\subsubsection{Creating a large number of GIDs with optimized techniques}

\pn   The    next   sample    program   still   uses    the   previous
\TT{domestic\_categories.lis} configuration file to initialize the GID
manager.   It  accesses  to  the  database  of  categories  through  a
\texttt{geomtools::id\_mgr::category\_info}  object  fetched from  the
GID  manager.   It then  uses  the  available  informations about  the
hierarchy rules to manipulate GID  objects at very low-level, i.e.  by
direct manipulation of the type value and the values/subaddress of the
addresses path at  any level :

\VerbatimInput[frame=single,
numbers=left,
numbersep=2pt,
firstline=1,
fontsize=\small,
showspaces=false]{\codingpath/domestic_3.cxx}

\pn The program prints :
\VerbatimInput[frame=single,
numbers=left,
numbersep=2pt,
firstline=1,
fontsize=\small,
showspaces=false]{\codingpath/domestic_3.out}
 
\pn Such a technique should be  favored when one needs to manipulate a
large number of  GIDs or even a few GIDs  very frequently.  Indeed the
human-friendly    methods     provided    by    the     GID    manager
(\texttt{geomtools::id\_mgr}) class are  based on searching algorithms
in various associative containers keyed by \texttt{string} objects. If
these methods are often used, some performance issues are expected. It
is  thus  more  efficient  to  directly use  the  integer  values  for
\emph{types} and \emph{addresses}  indexes, rather than human friendly
string labels.  Here the human-friendly  methods are used once  at the
beginning of  the program  to retrieve useful  addressing informations
stored as  integer values  (category \emph{types} and  address index);
then it is straightforward to reuse this addressing parameters a large
number of  times, without further  request through the  human friendly
interface of the GID manager.


%% end of id_mgr.tex
