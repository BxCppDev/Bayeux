%% gid.tex

\section{Geometry identifiers}

\subsection{Presentation of the concept}

\pn A \emph{geometry  identifier}, also known as  \emph{GID} or \emph{geom
  ID}, is an unique identifier associated to a geometry volume that is
part of a  geometry hierarchy. The figure \ref{fig:0}  shows a virtual
geometry setup made  of a collection of dictinct volumes  (A, B, C, D,
E, F, G) placed in a  reference frame (doted rectangle).  Here some of
the volumes  (F, G) are  included in another  one (E); this  implies a
natural hierarchy relationship between these last 3 volumes : volume E
is the mother of volumes F an  G; volumes F and G are the daughters of
volume E.

\begin{figure}[h]
\begin{center}
\scalebox{0.75}{\input{\pdftextpath/fig_0.pdftex_t}}
\end{center}
\caption{A simple geometry setup.}\label{fig:0}
\end{figure}

%\clearpage
\pn In  an application  that depends  on  and manipulates  such a  virtual
geometry setup,  it may  be interesting to  have access to  an unified
identification scheme for all or  part of the volumes.  In the present
case, volumes  labelled A,  B, E, F  and G  are each associated  to an
unique \emph{geometry identifier} (blue rounded boxes);  
on the other hand volumes  C and D
do not  benefit of  such association,  because it may be  not needed  
in the specific context of the application.

\pn Within the  \texttt{geomtools} program  library, the association  of a
geometry  volume placed in  the virtual  setup and  its GID  is called
\emph{geometry  mapping}.   Using such  a  concept,  it is  possible to
implement some techniques/algorithms that enable for example to:

\begin{itemize}

\item retrieve  the properties of a  volume given its  GID : placement
  (position  and rotation  matrix)  in the  reference  frame (or  some
  arbitrary  relative frame),  shape,  color, material  or any  useful
  property in the context of the application.
\begin{center}
\scalebox{1.0}{\input{\pdftextpath/fig_1.pdftex_t}}
\end{center}

\item  find/compute the  GID associated  to a  volume that  contains a
  given position P within the geometry setup :
\begin{center}
\scalebox{1.0}{\input{\pdftextpath/fig_2.pdftex_t}}
\end{center}

\end{itemize}

\pn These techniques can be implemented as \emph{locator} algorithms.

\pn But wait ! The following figure shows the case where the point Q
is inside both volumes E \textbf{and} G ;
\begin{center}
\scalebox{1.0}{\input{\pdftextpath/fig_3.pdftex_t}}
\end{center}

\pn It means that, in general,  the determination of the GID associated to
the volume  where a point lies  in can be ambiguous.  The final result
depends on what the user/application expects or at which depth of the
hierarchy the search is performed. In the present case, the ambiguity 
can be removed if one gives an additional information to the search
\emph{locator} algorithm. Such an information could be:
\begin{itemize}
\item the maximum depth of the volume hierarchy,
\item a specific depth of the volume hierarchy,
\item the \emph{category} of the object which is expected to 
  contain the given position $Q$.
\end{itemize}

\subsection{Design of the \texttt{geomtools::geom\_id} class}
 
\pn The above considerations give us some hints to design the implementation of
a \emph{geometry identifier} class. The minimal useful embedded information
to uniquely describe a volume could imply:
\begin{itemize}

\item  the specification  of the  \emph{category} of  object  a volume
  belongs   to:  a  physical   volume  is   thus  interpreted   as  an
  \emph{instance} of the \emph{geometry cateogory},

\item some informations that  describe the full hierarchical path from
  the top level  mother volume that contains the  considered volume to
  the effective depth of the volume in this hierarchy.

\end{itemize}

\pn This approach is used in \texttt{geomtools} to implement
the \texttt{geomtools::geom\_id} class.

\pn A GID is thus composed of the following attributes :
\begin{itemize}

\item the  \emph{geometry category}  is stored as  an \texttt{integer}
  value for  storage optimization and fast  manipulation. This integer
  value is named the \emph{type}  of the GID (or \emph{geometry type}). 
  Conventionnaly the value
  \texttt{-1}  means that  the type  is  not defined  (or not  valid).
  Value  \texttt{0} is  reserved  to  the top  level  category of  the
  geometry  setup \footnote{in the terminology of the
    GEANT4 program library,  it  corresponds  to  the
  unique \emph{world volume}}. All values  greater than \texttt{0} can be used
  to  represent   the  \emph{type}  of   some  volume.  \\It   is  the
  responsability of  the application  to manage the  classification of
  different categories of volumes hosted in the setup and associate an
  unique type integer value to any given category.

\item  a  collection of  successive  \emph{addresses}  that allows  to
  identify/traverse the  different levels  of the hierarchy  that host
  the  corresponding  volume. These  addresses  are  implemented as  a
  vector of integers. The size  of this vector points out the position
  depth  of  the  volume  in  some  hierarchical  relationship.   Each
  \emph{address}  in  the  vector  is  an  integer  value  with  value
  \texttt{-1} if  not defined  (not valid) and  \texttt{0} or  more if
  valid. \\It is  the responsability of the application  to allocate a
  meaning to each  address in the collection. It  is expected that the
  ordering of  addresses in the collection  reflects some hierarchical
  placement relationship between the volume  and its mother volume, grand-mother
  volume (if any)\dots this logics is managed by the application.
  
\end{itemize}

\pn    So,    within     \texttt{geomtools},    instances    of    the
\texttt{geomtools::geom\_id} class are represented using the following
format :
\begin{center}
\verb+[TYPE:ADDR0.ADDR1.+$\cdots$\verb+.ADDRN]+
\end{center}
\pn where \verb+TYPE+ is the value of the type (geometry category) and
\verb+ADDRN+ are  the values of each  address at successive  depths of the
geometry hierarchy.

\pn Examples : \verb+[6:0]+,  \verb+[6:1]+, \verb+[6:2]+
\verb+[12:2.0]+, \verb+[12:2.1]+, \verb+[18:2.0.3]+, \verb+[18:2.0.4]+.

\pn  It  should be  mentionned  that the  \texttt{geomtools::geom\_id}
class  is  a  low-level  class  used to  store  the  raw  informations
corresponding to  the geometry  identifier associated to  some volume.
The  choice  for  using  integers   as  the  basic  support  to  store
information  has been made  for convenience  and performance  both for
storage and  manipulation.  All  the \emph{intelligence} that  gives some
meaning to  the values used to  store the \emph{type}, as  well as the
interpretation of  the vector of \emph{addresses}, must  be managed at
higher  level.  The  \texttt{geomtools::id\_mgr} class  is responsible
for such features.

%% end of gid.tex
