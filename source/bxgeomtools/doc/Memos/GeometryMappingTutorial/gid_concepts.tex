%% gid_concepts.tex

\section{More concepts about the use of GIDs}

\subsection{A simple use case}

\pn  In order  to illustrate  the basic  concepts used  to  create and
manipulate  GIDs  with  the  help of  the  \texttt{geomtools}  program
library,  we  present  here  a  simple virtual  domestic  setup  which
accomodates several kinds of  objects with some simple (and realistic)
mother-to-daughter relationship between them.

\pn  We start first  by the  definition of  the set  of \emph{geometry
  categories}  the objects belongs  to.  For  now, we  have 10  of such
\emph{type} of objects :

\begin{itemize}
  
\item the \emph{category} \TT{house} is associated to the \emph{type}
  value \texttt{1},
  
\item the \emph{category} \TT{floor} is associated to the \emph{type}
  value \texttt{2},
  
\item the \emph{category} \TT{room} is associated to the \emph{type}
  value \texttt{3},
 
\item the \emph{category} \TT{table} is associated to the \emph{type}
  value \texttt{4},
  
\item the \emph{category} \TT{chair} is associated to the \emph{type}
  value \texttt{6},
   
\item the \emph{category} \TT{bed} is associated to the \emph{type}
  value \texttt{9},
  
\item the \emph{category} \TT{cupboard} is associated to the \emph{type}
  value \texttt{12}.
  
\item the \emph{category} \TT{small\_drawer} is associated to the \emph{type}
  value \texttt{34}.
  
\item the \emph{category} \TT{large\_drawer} is associated to the \emph{type}
  value \texttt{35}.
  
\item the \emph{category} \TT{blanket} is associated to the \emph{type}
  value \texttt{74}.
  
\end{itemize}

\begin{table}[h]
\begin{center}
  \begin{tabular}{|c|c|}
    \hline
    Category  & Type  \\
    \hline
    \hline
    \TT{world}  & \texttt{0} \\
    \hline
    \TT{house}  & \texttt{1}\\
    \hline
    \TT{floor}  & \texttt{2}\\
    \hline
    \TT{room}  & \texttt{3}\\
    \hline
    \TT{table}  & \texttt{4}\\
    \hline
    \TT{chair}  & \texttt{6}\\
    \hline
    \TT{bed}  & \texttt{9}\\
    \hline
    \TT{cupboard}  & \texttt{12}\\
    \hline
    \TT{small\_drawer}  & \texttt{34}\\
    \hline
    \TT{large\_drawer}  & \texttt{35}\\
    \hline
    \TT{blanket}  & \texttt{74}\\
    \hline
  \end{tabular}
  \end{center}
  \caption{The lookup table for \emph{geometry categories} and associated \emph{types}.}
  \label{tab:id_map:0}
\end{table}

\pn  The  only  constraint   here  is  that  both  \emph{type}  values
(implemented as  integers) and \emph{category}  labels (implemented as
character strings) are  unique for a given GID  manager. We needs here
some kind  of look-up table with unique  \emph{key/value} pairs (table
\ref{tab:id_map:0}).

\pn  Then   we  give  the   rules  that  describes   the  hierarchical
relationships between different categories  of objects. In our present
domestic  example, we  can  make  explicit the  rules  of our  virtual
domestic world :

\begin{itemize}
  
\item the  whole setup (the  top-level volume) contains one  or more
  objects  of  the  \TT{house}  category  (\emph{houses})  (here  we
  exclude the case  of a top level volume without any  house in it :
  it is indeed of no interest !)
  
\item a \emph{house} contains at least one or several \emph{floors},
  
\item a \emph{floor} contains at least one or several \emph{rooms},
  
\item a \emph{room} contains zero or more objects of the following 
  categories : \TT{chair}, \TT{table}, \TT{bed}, \TT{cupboard},
  
\item a \emph{table} can have zero or only one \emph{small drawer},
  
\item a \emph{cupboard} can have zero or up to 4 \emph{large drawer},
  
\item a \emph{bed} can host zero or more \emph{blankets},
  
\item  a  \emph{blanket}, a  \emph{small  drawer}  or a  \emph{large
  drawer} cannot contains anything;  there are the terminal leafs of
  the hierarchy (they cannot be considered as \emph{containers}).
  
\end{itemize}

\pn The figure \ref{fig:house:1} shows these various categories of objects.

\begin{figure}[h]
\begin{center}
\scalebox{1.}{\input{\pdftextpath/fig_house_1.pdftex_t}}
\end{center}
\caption{Various categories of objects in a domestic world.}\label{fig:house:1}
\end{figure}

\subsection{Different kinds of hierarchical relationships}

\pn This domestic example is a good start to investigate the different
placement  relationships that can  be identified  in such  a virtual
model.  Let's consider the \emph{world} in figure \ref{fig:house:2}.
\begin{figure}[h]
\begin{center}
\scalebox{1.}{\input{\pdftextpath/fig_house_2.pdftex_t}}
\end{center}
\caption{A simple domestic world.}\label{fig:house:2}
\end{figure}
\pn We  have here  only one  house with two  floors. The  ground floor
contains two  rooms, the first floor  has one single  room. The large
room  at ground floor  contains three  chairs and  one table  with one
unique small  drawer.  The small room  at ground floor  has one single
chair.  The unique room at first floor contains one chair and one bed
with one blanket on it. Looks like your place isn't it ?

\pn Now we can make an exhaustive inventory of all the objects
that belongs to this world. We can associate to each of them an unique
\emph{geometry identifier} :

\begin{itemize}

\item the GID of the unique  house here has type value \texttt{1}.  As
  our world can  in principle host several houses,  we must allocate a
  \emph{house  number} to  this particuliar  house. Let's  chose house
  number \texttt{666} (the  Devil's house !). So the  GID of the house
  can be read : ''I'm an  object in category \TT{house} and my address
  is fully defined by the \TT{house\_number}=666''. This should lead to
  the following  format: \verb+[1:666]+.   The depth of  the addresses
  path of the  GID (\texttt{666}) is only 1 because only one  integer value is enough
  to distinguish this \emph{Devil's house} from some possible other houses
  we could add in this virtual world (\emph{Mary's place}, \emph{The Red Lantern}\dots).

\item Let's now consider the ground and first floors. Of course these
  objects share the same \emph{category}, both are \emph{floors}.  The
  only  way to  distinguish  them  is to  allocate  to them  different
  addresses.  The  addresses path  still has to  reflect to  fact that
  they    both   belong    to   the    \emph{Devil's    house}   (with
  \TT{house\_number}=666).   So  thay  must  share  this  hierarchical
  information. Finally  the minimal information  needed to distinguish
  both floor is to  allocate and additionnal \emph{floor number} (thus
  another integer number).
  \begin{itemize}
 
  \item  The ground  floor, or  floor with  number \texttt{0},  can be
    provided with  the following GID: \verb+[2:666.0]+,  which reads :
    ''I'm an object of type  \texttt{2} (so my category is \TT{floor})
    and  I  belong  to   the  house  of  which  \TT{house\_number}  is
    \texttt{666}.  My \TT{floor\_number} is \texttt{0}''

  \item  Using  a  similar  scheme,  the first  floor  is  given  the
    following GID : \verb+[2:666.1]+, which reads : ''I'm an object of
    type \texttt{2} (so my category is \TT{floor}) and I belong to the
    house   of   which   \TT{house\_number}   is   \texttt{666}.    My
    \TT{floor\_number} is \texttt{1}''
  \end{itemize}
\pn So the  depth of the addresses associated to  a \emph{floor} is 2.
In the  geometry mapping terminology in use  in \texttt{geomtools}, we
say  that  the   \TT{floor}  category  \emph{extends}  the  \TT{house}
category   (and   its   \TT{house\_number}  address)   \emph{by}   the
additionnal  \TT{floor\_number} address, leading  to a  two-levels (or
depth) addressing scheme.

\item  Now we  can play  the  same game  for the  \emph{rooms}. It  is
  obvious that  all 3 rooms in  the Devil's house share  the same type
  (set  conventionnaly at \texttt{3})  and belongs  to the  same house
  (\TT{house\_number}=\texttt{666}).     However     thay    can    be
  distinguished     by    two     different     informations:    their
  \TT{floor\_number}   and    a   new   level    of   address:   their
  \TT{room\_number}  with  respect to  the  floor  they  lie in.  This
  enables us to allocate a unique GID to each of them:
  \begin{itemize}
 
  \item Large room at ground floor: GID is \verb+[2:666.0.0]+ where we
    chose  to  start  the  numbering  of  rooms  at  this  floor  with
    \TT{room\_number}=0
 
  \item Small room at ground floor: GID is \verb+[2:666.0.1]+ where we
    chose  the next available  \TT{room\_number}=1 because  the former
    room already used \TT{room\_number}=0.

  \item Unique  room at first floor: GID  is \verb+[2:666.1.0]+ where
    we chose  to start  the numbering  of rooms at  this a  floor with
    \TT{room\_number}=0

  \end{itemize}
So the depth of the addresses  associated to a \emph{room} is 3.  Here
we  say  that the  \TT{room}  category  \emph{extends} the  \TT{floor}
category (and its \TT{house\_number} and \TT{floor\_number} addresses)
\emph{by}  the additionnal  \TT{floor\_number} address,  leading  to a
three-levels (or depth) addressing scheme.

\item  What about  chairs, beds  and tables  ? Again  as one  room can
  contains several of  these objects, we will have  to add additionnal
  number along  the addresses  path to uniquely  identify them  with a
  GID. This gives the following collection of GID:
\begin{itemize}
\item[] \verb+[6:666.0.0.0]+ : the first chair in the large room at ground floor of the Devil's house
  where a \TT{chair\_number}=\texttt{0} value has been appended to the addresses path of the mother
  room with GID \verb+[6:666.0.0]+,
\item[]\verb+[6:666.0.0.1]+ : the second chair in the large room at ground floor of the Devil's house
(appended \TT{chair\_number}=\texttt{1}),
\item[]\verb+[6:666.0.0.2]+ : the third chair in the large room at ground floor of the Devil's house
(appended \TT{chair\_number}=\texttt{2}),
\item[]\verb+[4:666.0.0.0]+ : the unique table in the large room at ground floor of the Devil's house
(appended \TT{table\_number}=\texttt{0}),
\item[]\verb+[6:666.0.1.0]+ : the unique chair in the small room at ground floor of the Devil's house
(appended \TT{chair\_number}=\texttt{0}),
\item[]\verb+[6:666.1.0.0]+ : the unique chair in the unique room at first floor of the Devil's house
(appended \TT{chair\_number}=\texttt{0}),
\item[]\verb+[9:666.1.0.0]+ : the unique bed in the unique room at first floor of the Devil's house
(appended \TT{bed\_number}=\texttt{0}),
\end{itemize}
\end{itemize}

\pn Should we add  another chair in the room at first  floor and also a cupboard
in the large room at ground floor (figure \ref{fig:house:3}) ?  
\begin{figure}[h]
\begin{center}
\scalebox{1.0}{\input{\pdftextpath/fig_house_3.pdftex_t}}
\end{center}
\caption{More objects in the domestic world.}\label{fig:house:3}
\end{figure}
\pn No problem ! We allocate to this  new chair the GID :
\begin{center}
\verb+[6:666.1.0.1]+
\end{center} 
\pn where we have appended the \TT{chair\_number}=\texttt{1}. 
We also allocate to the  new cupboard the GID :
\begin{center}
\verb+[12:666.0.0.0]+
\end{center} 
\pn where we use the  type \texttt{12} that corresponds to 
the \TT{cupboard} category and
extends   the  GID   of  its   mother  room   (\verb+666.0.0+)   by  a
\TT{cupboard\_number}. The \TT{cupboard\_number} arbitrarily starts  at \texttt{0}  (in  case we
would add more and more cupboards in this room : \TT{cupboard\_number}=\texttt{1}, \texttt{2}\dots).

\pn  We have  clearly identified  one kind  of  hierarchy relationship
between  different  \emph{categories}  of   objects  in  term  of  the
\emph{extension}  of the  addresses path  by some  additionnal address
value(s)  (integer   numbers).  These  appended   numbers  enable  the
distinction between different objects of  the same type located in the
same  container object.  This concept  is closed  to  the \emph{mother
  volume} concept in  a hierachical geometry setup and  is indeed well
adapted to represent such geometry placement relationship.

\pn  Now let's  consider another  situation !  As the  placement rules
above  state it  : ''a  table object  can have  one and  only  one small
drawer''. What  does it mean  ? It means  that to absolutely  identify a
specific  drawer, we  only need  to know  to which  table  it belongs.
There is no need for  additionnal information to \emph{locate} it.  We
say in this case  that the \TT{small\_drawer} category \emph{inherits}
the addressing scheme of the  \TT{table} category.  So, in the case of
the drawer plugged  in the ''unique table in the  large room at ground
floor of the Devil's house'', we can build its GID :
\begin{center}
\verb+[34:666.0.0.0]+ 
\end{center}
\pn which shares  the same 4-levels addresses path  of its host/mother
table,  but  differs only  by  the  type  identifier (\texttt{34}  for
the  \TT{small\_drawer} category in place of \texttt{4} for the
\TT{table} category).

\pn This  has to  be compared with  the way  we should treat  the four
\TT{large\_drawer} objects that belong  to the newly added cupboard in
the  same  room.  In  this   case  we  must  use  a  \emph{extend  by}
relationship because  an additionnal \TT{drawer\_number}  is needed to
distinguish the 4 possible drawers  in a cupboard. So, in this virtual
world,  the GID of  a \TT{small\_drawer}  has some  4-levels addresses
path but the  GID of a \TT{large\_drawer} has  some 5-levels addresses
path. This  cleary shows that in  this approach, the  structure of the
numbering scheme attached to an object is not an intrinsic property of
the (geoemtrical)  nature of the  object but an intrinsic  property of
its relationship  with its environment (the  way it is  placed in some
mother object).  We can imagine  that there is no  physical difference
(shape, color, dimensions) between the model of the drawer inserted in
a table and  the model of the drawer inserted in  a cupboard. They can
have the  same physical  properties and thus  share a  common physical
description. However, the way \emph{we  use} a generic drawer 
defines another level of properties:
\begin{itemize}
\item a drawer for a table  must belong to an object of the \TT{table}
  category and can contains  objects of the \TT{pencil} category; this
  defines the \TT{small\_drawer} category,
\item  a  drawer for  a  cupboard  must belong  to  an  object of  the
  \TT{cupboard} category  and can  contains objects of  the \TT{shirt}
  or \TT{tights} category; this defines the \TT{large\_drawer} category.
\end{itemize}

\subsection{Numbering scheme is an arbitrary choice within an application}

\pn  The above  considerations  enlight the  fact  that the  so-called
\emph{geometry  category} does not  reflect the  nature of  a physical
object with its physical properties  but the way it is \emph{inserted}
and/or \emph{filled} in its environment made of other objects.

\pn This can be seen on figures below where we can attribute different
numbering  schemes  to  the  same  virtual world,  depending  on  some
arbitrary way of thinking :

\begin{itemize}
\item Case 1 :
   Here  all  box  objects  are  considered  to  belong  to  the
    \TT{any\_box} category (with  \texttt{type=0}). 

    \begin{tabular}{|c|c|}
      \hline
      Category  & Type  \\
      \hline
      \hline
      \TT{world}  & \texttt{0} \\
      \hline
      \TT{any\_box}  & \texttt{1}\\
      \hline
    \end{tabular}
    
    \pn We  don't take  into account the  mother/daughter relationship
    between the  red and the green  boxes, then between  the green and
    the  blue box. This  corresponds to  a \emph{flat}  geometry model
    where we don't need to  know anything about the internal hierarchy
    of  the geometry  setup.  So  we don't  need to  set  any specific
    \emph{hierarchy rules}  but one  : the top-level  volume (world)
    contains some objects of category \TT{any\_box}.

    \pn We get 3 GIDs with the same \emph{type} and some distinct
    \TT{object\_number} (running from \texttt{0} to \texttt{2}) :
    \begin{center}
      \scalebox{1.0}{\input{\pdftextpath/fig_alt_0.pdftex_t}}
    \end{center}

    \pn Note  that in this case,  the physical structure  of the setup
    does rely on several level of nested volumes, but we simply do not
    consider this fact while choosing the abstract numbering scheme
    and building the GIDs.


\item Case 2 :
 Here  we take into account the hierarchical relationships
    between the three  box  objects. The lookup table now defines
    3 different \emph{categories} with 3 associated \emph{types} :

    \begin{tabular}{|c|c|}
      \hline
      Category  & Type  \\
      \hline
      \hline
      \TT{world}  & \texttt{0} \\
      \hline
      \TT{big\_box}  & \texttt{10}\\
      \hline
      \TT{medium\_box}  & \texttt{11}\\
      \hline
      \TT{small\_box}  & \texttt{12}\\
      \hline
    \end{tabular}

    \pn Some \emph{hierarchy rules} are also given :

    \begin{itemize}

    \item  ''the  top-level volume (world)  contains some
    objects of category \TT{big\_box}'',

    \item  ''an object  of category \TT{big\_box}  contains some
    objects of category \TT{medium\_box}'',

    \item  ''an object  of category \TT{medium\_box}  contains some
    objects of category \TT{small\_box}''.

    \item  ''an object  of category \TT{small\_box} cannot contain anything.

    \end{itemize}

    \pn The resulting GIDs associated to the 3 boxes is now :
    
    \begin{center}
      \scalebox{1.0}{\input{\pdftextpath/fig_alt_1.pdftex_t}}
    \end{center}

    \pn GIDs are not only distinct by their \emph{type} but also
    by the depth of their addresses path, with reflect the hierachical
    nature of thier respective placements. Of course, this way of doing 
    is richer than the \emph{flat} model.
\end{itemize}

\pn Finally,  it turns  out that the  choice for the  numbering scheme
strategy  (the  \emph{mapping})  is  the  affair of  the  user  and/or
application  and  is   not  fixed  by  the  physical   nature  of  the
objects/components  of  the  virtual  model, though  it  is  generally
related to it.

\subsection{Limitations of this approach}

\pn Although  the above hierarchical  approach is rather  powerful and
can be  naturally implemented to  design the numbering scheme  of many
different virtual geometry setups -- usually designed/interpreted with
the  hierarchical   (mother/daughter)  approach  --   there  are  some
(realistic)  cases that  cannot be  addressed  or with  at least  some
difficulties or limitations.

\pn The first of these case is the \emph{door}. Let's consider again a
domestic case with two adjacent rooms on the same floor of some house.

\begin{center}
  \scalebox{1.0}{\input{\pdftextpath/fig_house_4.pdftex_t}}
\end{center}


\pn In the real world, we generally use doors to move from one room to
the other.  In the  physical world,  a door can  be considered  as yet
another kind of  object in the domectic environment.   For example, it
is made of wood, has generally a rectangular shape, some dimensions, a
mass, can  be painted with  some arbitrary colour\dots exactly  like a
chair.  However   the  way  a   door  is  \emph{inserted}   within  the
\emph{geometry} of  our world is not  obvious on the point  of view of
the  natural  \emph{house/floor/room/furniture}  hierarchy.  The  main
question is : ''Of which room  a specific door belongs to ?''.  It was
easy to  answer this question  for a chair\dots  So we meet  here some
issues to use  our hierarchy model and clearly we  will have to invent
new concepts to handle such a situation.

\pn Another interesting case has to do with overlapping and/or complex
assembly of objects. The figure below shows two of these special cases
:
\begin{center}
  \scalebox{1.0}{\input{\pdftextpath/fig_house_5.pdftex_t}}
\end{center}

\pn Here  we have a  red screw  that has some  parts of its  volume in
different rooms,  another part is  inserted in the grey  region (house
structure);  its  peak  is even  inserted  in  the  brown box  on  the
left.  The other  special case  concerns the  yellow  region (labelled
''crime scene'' that  is splitted in two distinct  rooms. Because of
the overlapping placements of these  objects, it is again not possible
to address the problem in term of a simple hierarchical relationship.

\pn  In software applications  we use  for geometry  modelling (GEANT4
simulation, GDML  modelization, visualization software),  such complex
situations are generally forbiden  by the underlying geometry modeling
engine, despite they are not rare in the real world.

%% end of gid_concepts.tex
