%% GeometryModellingTutorial.tex
%%
%%
\documentclass[a4paper,12pt]{article}

\usepackage[T1]{fontenc} 
\usepackage{ucs} 
\usepackage[utf8x]{inputenc} 
%%french%%\usepackage[frenchb]{babel}
\usepackage{amsmath}
\usepackage{amssymb}
\usepackage{latexsym}
\usepackage{verbatim}
\usepackage{moreverb}
\usepackage{fancyvrb}
\usepackage{alltt} 
\usepackage{eurosym} 
\usepackage{hyperref}
\usepackage{colortbl}
\usepackage{epsfig}
\usepackage{graphicx}
\usepackage{pgf}
\usepackage{float} 

\addtolength{\textwidth}{+2cm}
\addtolength{\textheight}{+3cm}
\addtolength{\topmargin}{-1.5cm}
\addtolength{\oddsidemargin}{-1cm}

\newcommand{\basepath}{.}
\newcommand{\imagepath}{\basepath/images}
\newcommand{\codingpath}{\basepath/coding}
\newcommand{\pdftextpath}{\basepath/pdftex_t}
\newcommand{\pdftexpath}{\basepath/pdftex}

%% declare_verbatim.tex
%% -*- mode: latex;-*-
%%
\DefineVerbatimEnvironment%
{ShellVerbatim}{Verbatim}%
{fontsize=\small,%
frame=single,%
framesep=2mm,%
framerule=0.25mm,%
labelposition=topline,%
numbers=none%
}

\DefineVerbatimEnvironment%
  {PathVerbatim}{Verbatim}%
{fontsize=\small,%
frame=single,%
framesep=1mm,%
framerule=0.15mm,%
numbers=none%
}

\DefineVerbatimEnvironment%
{CppVerbatim}{Verbatim}%
{commandchars=\\\{\},%
fontsize=\small,%
frame=single,%
framesep=2mm,%
framerule=0.25mm,%
labelposition=topline,
numbers=left,%
numbersep=2pt%
}

%% end of declare_verbatim.tex


\newcommand{\pn}{\par\noindent}
\newcommand{\TT}[1]{"\texttt{#1}"}

\title{Geometry modelling with \texttt{geomtools}\\%
{\small{(Software/geomtools/GeometryModellingTutorial -- version 0.1)}}}
\author{F. Mauger <\texttt{mauger@lpccaen.in2p3.fr}>}
\date{2011-12-01}

%%%%%%%%%%%%%%%%%%%%%%%%%%%%%%%%%%%%%%%%%%%%%%%%%%%%%%%%%%%%%%%%
\begin{document}

\maketitle

\begin{abstract}
In this note, we explain the principles and tools
for modelling a virtual geometry setup within \texttt{geomtools}.
\end{abstract}

\tableofcontents

\section{Introduction}

\pn  The \texttt{geomtools} program  library provides  some high-level
general  purpose  utility classes  and  embedded  algorithms to  build
virtual geometry models in a  somewhat easy way.  It has been designed
as  the main  tool for  feeding  external programs  with a  standalone
geometry description, using some automated exporting/filtering tools to
interface with these third-party programs and libraries. In the fields
of experimental  high energy and  nuclear physics, those  programs are
typically in charge of :
\begin{itemize}
\item the simulation of particle tracks through the
virtual  geometry model  of  a  detector  (GEANT4),  
\item reconstruction  algorithm,
\item visualization software,
\item various step of the data analysis. 
\end{itemize}
Unfortunately, there  is not one  single generic and standard  tool or
even  paradigm  to achieve  such  functionnalities. Usually,  software
libraries  use their  own geometry  modelling scheme,  with  their own
strategy, optimization and various policies  that can make them non or
hardly interoperable.   Particularly, the choice  for such a  tool can
prevent the user of client application to use in parallel another tool
with different modeling techniques.

\pn  Because of  this,  we have  implemented  some geometry  modelling
interface (GMI)  and tools  that, we hope,  are supposed to  allow the
making  of new  interfaces  compatible with  other geometry  modelling
implementations.   Of  course,  there   is  no   hope  to   cover  all
functionnalities  you  could   dream  about.   However,  the  geometry
modelling tools  in \texttt{geomtools}  try to address  most practical
cases  in the  framework  of our  experimental  activities :  detector
simulation, visualization, reconstruction and data analysis.

\pn  The  \texttt{geomtools} GMI  relies  on  some standard  modelling
approach that  have intrinsic limitations. Any  virtual geometry model
managed by  \texttt{geomtools} uses a hierarchical  system to describe
the  physical 3D-objects  in a  geometry  setup.  It  deals only  with
placement of daughter objects in their mother volume at several levels
of  a possibly  large hierarchy.  There  is no  possibility to  handle
overlapping   volumes  or  complex   nested  geometries   with  cyclic
mother/daughter  relationship. This  apporach  is shared  by the  GDML
language, the GEANT4 and ROOT libraries which are compatible with this
paradigm.  So  it is  expected that \texttt{geomtools}  will interface
naturally with these implementations.

\vskip 5mm
\pn
Subversion repository:\\
\texttt{https://nemo.lpc-caen.in2p3.fr/svn/geomtools/}
\pn
DocDB reference: \texttt{NemoDocDB-doc-1995}
\pn
References: see also
\textit{Geometry mapping with \texttt{geomtools}} (\texttt{NemoDocDB-doc-1996})
\pagebreak

%% basic_concepts.tex

\section{Basic concepts}

\subsection{Logical volumes and physical volumes}

\pn  First of  all, the  \texttt{geomtools} GMI  provides  an abstract
interface that allows the description of the hierarchical relationship
between physical objects in a geometry setup.

\pn The key  concept is the \emph{logical volume}  which describes the
fundamental  geometry properties  of  a  3D-object in  a  way that  is
independant of its  placement in the whole setup.  This is illustrated
on figure  \ref{fig:lv:0} where  several \emph{objects} (copies)  of a
the same \emph{type} are placed within a setup.

It  this crucial to  distinguish the  \emph{logical description}  of a
chair  from its \emph{physical  implementations} (placements))  in the
  virtual world :
  \begin{itemize}

  \item the \emph{logical volume} concept implement the description of
    intrinsic geoemtry properties (the \emph{type}),

  \item the  \emph{physical volume} concept  implement the description
    of the placement (position/rotation matrix) of an instance of some
    \emph{logical  volume}  in the  geometry  setup (the  instantiated
    \emph{object}).

  \end{itemize}

\begin{figure}[h]
\begin{center}
\scalebox{0.75}{\input{\pdftextpath/fig_lv_0.pdftex_t}}
\end{center}
\caption{Several copies of a same type of object can be placed in the virtual 
geometry setup.}\label{fig:lv:0}
\end{figure}

So what is a \emph{logical  volume} ? Following the approach of the GDML langage
and GEANT4 modelling interface, we can list the informations that fully describe
an instance of logical volume :

\begin{itemize}

\item   a  unique  \emph{name}   that  will   allow  to   address  the
  \emph{logical volume}  object non-ambiguously in a  database of many
  \emph{logical  volumes}.\\  \pn  Examples:  \TT{chair},  \TT{table},
  \TT{desk},     \TT{bedroom},     \TT{bathroom},    \TT{city\_house},
  \TT{cottage}\dots
  
\item  a 3D  shape that  will  define the  physical bounds  in the  3D
  virtual   space.\\  \pn   Examples:  a   \emph{box}   of  dimensions
  3$\times$2$\times$1.3 m$^3$,  a \emph{cylinder} of  radius $r$=25 cm
  and height $h$=75 cm\dots

\item An optional list of daughter volumes that are fully contained in
  the  bounding limits  (the 3D  shape)  of the  logical volume.  This
  implies that we know where  to place these daughter volumes. We thus
  need to provide  not only their own \emph{logical  volumes} but also
  their position  and orientation  (placement) in the  current logical
  volume that is called the \emph{mother} volume.

\end{itemize}

Figure \ref{fig:lv:1} shows two  different descriptions of some simple
volumes.  Despite  their external envelopes  share the same  shape and
dimensions,  the  left  and   right  ''boxed''  logical  volumes  have
significant differences.   They do  not have the  same colour  (may be
because  they are  made with  different  materials) and  the blue  one
contains  some  daughter  box  objects  while  the  left  one  has  no
daughters. As the blue logical volume contains two tiny red boxes, we must
provide the coordinates (position/rotation) for the placement of these
daughter volumes. More it is obvious that the red boxes are copies 
(\emph{physical volumes} that share the same description; we must
also provide the description of this third \emph{logical volume}.

\begin{figure}[h]
\begin{center}
\scalebox{0.75}{\input{\pdftextpath/fig_lv_1.pdftex_t}}
\end{center}
\caption{Two  different  simple  logical  volumes.}\label{fig:lv:1}
\end{figure}

We see  here that  the full  description of a  geometry setup  will be
given by a  more or less complex hierarchy  of physical volumes nested
in logical  volumes, in turn nested  in physical volume  at the parent
level and so on\dots

Typically the geometry description looks like a large tree
of an arbitrary depth which depends on the number of nested
hierarchy levels:
\begin{ShellVerbatim}
"world.log"(top-level of the hierarchy)
|-- Material: "air"
|-- Colour: "transparent"
|-- Shape: "box" with x,y,z=(30,20,10) m 
`-- Daughters:
    |-- Physical: "house_0.phys"
    |    |-- Position/rotation = (3,-2,0) m / R(Oz, 90°)
    |    `-- Logical: "house.log"
    |        |-- Material: "air"
    |        |-- Colour: "transparent"
    |        |-- Shape: "box" with x,y,z=(30,20,10) m 
    |        `-- Daughters:
    |            |-- Physical: "ground_floor.phys"
    |            |   |-- Position/rotation = (1,2,0) m / R(Oz,0°)
    |            |   `-- Logical: "ground_floor.log"
    |            |       |-- Material: "concrete"
    |            |       |-- Colour: "gray"
    |            |       |-- Shape: "box" with x,y,z=(...) m 
    |            |       `-- Daughters:
    |            |           |-- Physical: "kitchen.phys"
    |            |           |    |-- Position...
    |            |           |    `-- Logical: "kitchen.log"
    :            :           :
    |-- Physical: "house_1.phys"
    |    |-- Position/rotation = (3,-2,0) m / R(Oz, 90°)
    |    `-- Logical: "house.log"
    |        |-- Material: "air"
    |        |-- Colour: "transparent"
    |        |-- Shape: "box" with x,y,z=(30,20,10) m 
    |        `-- Daughters:
    |            |-- Physical: "ground_floor.phys"
    :            :   
    |            `-- Physical: "last_floor.phys"
    :
\end{ShellVerbatim}
%%$

\pn Practically, the \texttt{geomtools} API provides
two classes:
\begin{itemize}
\item the \texttt{geomtools::logical\_volume} class,
\item the \texttt{geomtools::physical\_volume} class.
\end{itemize}
\pn Basically its  API is a clone  of what can be found  in the GEANT4
program library. This is natural as both APIs follow the GDML geometry
modelling approach. However \texttt{geomtools} allows to associate arbitrary
properties to any logical or physical object. This allows
client applications to benefit of some tools for storage and/or fetching
high-level meta-data. This mechanism is used to pass visualization informations
to some 3D-display program (visibility, colour\dots), material information
for physics simulation programs (GEANT4), directives to the numbering scheme
manager (\emph{mapping})\dots Some naming conventions are of course needed
to ease the extraction of arbitrary informations by topic. This
feature relies on the \texttt{datatools::utils::properties} container class.
By essence, the availability of this functionnality make this mechanism
extensible to other applications. 

\subsection{Geometry models}

\pn If the \texttt{geomtools} API had proposes only a rewritting
of the GEANT4 interface and/or the GDML modelling approach, even with a few
more features added in it, it would have been of limited interest.
Indeed this API is \emph{based} on a similar interface to GEANT4/GDML, but
implements higher level functionnalities.

The key  idea here is  to obtain a  very compact and efficient  way to
describe  a geometry  setup without  entering the  guts of  the nested
logical/physical  volume   hierarchy  and  the   expertise  needed  to
manipulate and navigate through this hierarchy. More, we would like to
implement some tools to:
\begin{itemize}

\item automate the construction of a transcient virtual geometry model
  using  a \emph{geometry  engine} that  uses  only a  limited set  of
  configuration  parameters  to  build   a    hierarchy  of
  3D-volumes of arbitrary complexity,

\item benefit of a collection  of generic objects that represents very
  often  used geometry patterns  : volumes  with very  familiar shapes
  (box, cylinder\dots), stacked volumes, replicated volumes, composite
  volumes (union, intersection, differences)\dots,

\item  automate   a  simple   3D-rendering  for  fast   debugging  and
  developpement,

\item automate the  conversion of any geometry model  in the format of
  another API (GDML, GEANT4, ROOT\dots),

\item automated the management of a numbering scheme policy,

\item enable some arbitrary meta-data to be attached to any node
  of the hierarchy tree,

\item  enable  extension  with  possible new  geometry  primitives  or
  hard-coded descriptions of new  logical volumes if generixity cannot
  be achieved,

\item interoperability between generic components and hard-coded components,

\item be human-friendly with ASCII file based configurations,

\item hide  the complex memory  management of the  transcient geometry
  model\footnote{You may  have a look on  GEANT4 detector construction
    and all the pointers users have to play with !}.

\end{itemize}

Within  GDML,  part  of  this features  (flexibility,  human  readable
configuration  files,  some  filters  to  GEANT4  and  ROOT\dots)  are
addressed   through  the   grammar  and   syntax  of   this  XML-based
language. However, the manipulation of XML files turns to be difficult
as  the  complexity of  the  geometry  system  increases. Despite  the
possibility  to use  \emph{parametrized  logical volumes},  on-the-fly
computed  positionning  is  limited.   More  it  does  not  provide  a
standalone  transcient  geometry  model  :  you  have  to  choose  the
transcient model from the GEANT4  or ROOT library but it is difficult 
to used both systems in  cooperation  within  the same  program  
or  in  the context  of  a
third-party application that has its own modelling scheme and approach
(your data  analysis and event reconstruction  for example).  However,
GDML files are a  good interface medium and we will use this technique
 as the core of the interface with GEANT4.

As  the  handling of  many  logical/physical  volumes  is complex  and
request some programming expertise (memory management, pointers, knowledge
of some specific API\dots),
\texttt{geomtools} proposes a special concept to automated and hide
most parts of  this low-level techniques, still providing  the user or
external application a way  to manipulate the concepts of hierarchical
geoemtry  modelling.  A  new  interface  has been  implemented  :  the
\emph{geometry model}.

A  \emph{geometry  model}  has  the  responsability  to  describe  the
characteristics  of a  given \emph{logical  volume}, in  such a  way a
logical volume is always  instantiated through its associated geometry
model. However, a  geometry model is for the logical  volume what is a
class for  an object.  Thus  a geometry model may  have parametrization
facilities that enable to  instantiate different kinds of \emph{logical
  volumes}; such logical volumes will be rather similar because 
they are managed/created by the same \emph{geometry driver} and 
they will \emph{behaves} in the same way, .

The difference between the \emph{geometry model} and the \emph{logical
  volume} concepts is not obvious for very simple 3D objects (a simple
box or cylinder without any  daughter volumes): the geometry model for
a simple box-shaped  volume made of copper will  be responsible of the
instantiation of the logical volume  made of a box shape associated to
copper     material.     The      figure     below     represent     a
\texttt{simple\_boxed\_model}   geoemtry  model   that   is  able   to
instantiate a logical volume  named \TT{my\_green\_box}.  The user (on
the    left)    just    has    to    send    a    request    to    the
\texttt{simple\_boxed\_model} object that behaves like a \emph{logical
  volume     factory}    and     instantiates     automatically    the
\TT{my\_box}  logical  volume  given some  specific  parameters
($w$, $h$, $d$, colour, material\dots) that are passed when the user's
request is submitted.

\begin{center}
\scalebox{0.75}{\input{\pdftextpath/fig_gm_0.pdftex_t}}
\end{center}


\pn Introducing  the
\emph{geometry model} layer just adds  an intermediate step but does not
add functionnalities.  We could have directly create the logical 
volume by hand using low-level functionnalities of the API :

\begin{center}
\scalebox{0.75}{\input{\pdftextpath/fig_gm_1.pdftex_t}}
\end{center}

However, as soon  as we want to manipulate some
complex logical  volumes, the  difference is fondamental.   Suppose we
want to  build a logical volume  that stacks three  boxes of different
dimensions along  an arbitrary axis. To  get this in  GEANT4, you will
have to  instantiate first the three logical  volumes corresponding to
each box. Then you will create a logical volume, choose a shape for it
(say  a  box), compute  its  dimensions  from  the dimensions  of  the
internal boxes you want to  stack, position the daughter boxes (yes we
speak about daughter  physical volumes) along the given  axis, check that there
is no overlapping volumes and that daughters are fully contained in the
mother box :

\begin{center}
\scalebox{0.75}{\input{\pdftextpath/fig_gm_2.pdftex_t}}
\end{center}

This is some work ! Each time you will  have to stack objects in this way, you
will have to reproduce this algorithm, manage the memory associated to
the volumes, compute all  requested geometry parameters. In real life,
it appears  that stacking  volume is a  frequent operation :  in fact
this is a current  \emph{geometry modelling pattern}.  This is exactly
the moment for a dedicated  \emph{geometry model} to enter the scene !
It is possible  to implement a generic algorithm of  which the task is
to  perform all  the  complex operations  needed  to obtain  a set  of
stacked  volumes enclosed in  some mother  volume.  Not  only the geometry
model for constructing such stacked volumes  
will use this algorithm, it  will also
manage al  the internals  : memory stuff,  mother/daughter relationships,
conventionnal naming of the internal objects/volumes :

\begin{center}
\scalebox{0.75}{\input{\pdftextpath/fig_gm_3.pdftex_t}}
\end{center}

Practically,  the designer  of  a virtual  geometry  setup will  never
directly manipulate logical and/or physical volumes. He will deal only
with geometry models.  This  approach defers the complex technical
part of the modelling to the \texttt{geomtools} embedded geometry engine.
The user can concentrate only on the building of the fondamental
objects that enter the composition of the setup and their relative 
relationships in terms of hierarchy.

Of  course, logical  and physical  volumes  still exist  and are  used
within the  internals of the \texttt{geomtools}  engine. These objects
are the  core of the  hierarchy geometry tree.  So it is  natural that
developpers  will still have  to handle  logical and  physical volumes
when thay design  new geometry model classes and  propose extension to
the library of already available models.

A \emph{model factory} class has been implemented as the main engine
responsible of the construction of a geometry hierarchy. Given some
geometry configuration files, it automates the allocation of requested 
geometry models, checks (partially) the coherence of the system, generates the 
associated logical and physical volumes and makes a full transcient
geometry model available. The class is \texttt{geomtools::model\_factory}.

More all generic and primitive models available from the core library
benefit of an automated class registration mechanism based on some
internal lookup table. The instantiation
of geometry models is thus completely transparent, as well as memory
management issues. This mechanism also extends to the geometry model classes
implemented by developpers to handle special cases where no
combination of the existing pre-registred models can cover users' needs.
This makes the system rather generic and extensible.

%% end of basic_concepts.tex


\clearpage
%% use_cases.tex

\section{Use cases}

Now it is time to play  with geometry models and the geometry factory.
We will start  by a very simple case and will  introduce more and more
complexity progressively.

\subsection{A box in a virtual world}

We  want to achieve  the construction  of the  virtual setup  shown on
figure \ref{fig:mf:0}.   We have here  a simple top level  large green
box (the \emph{world}) with  given dimensions and auxiliary properties
(color,  material\dots).   This \emph{world}  box  contains a  single
smaller red  box with its  own geometry and auxiliary  properties. The
placement of the red box inside  the mother green box is determined by
the position  of the center  $O'$ of the  red box with respect  to the
center $O$  of the green box.   The orientation of the  red box inside
the mother green box can  be determined by the Euler angles associated
to  the rotation matrix  that transform  the ($0xyz$)  world reference
frame axis into the ($0'x'y'z'$) daughter frame axis.


\begin{figure}[h]
\begin{center}
\scalebox{0.75}{\input{\pdftextpath/fig_mf_0.pdftex_t}}
\end{center}
\caption{A very simple world.}\label{fig:mf:0}
\end{figure}

With  \texttt{geomtools}, the  easiest way  to construct  this virtual
geometry setup is to use two available geometry model classes :
\begin{itemize} 

\item  the \texttt{geomtools::simple\_world\_model} class  is designed
  to model a  top-level box with arbitrary dimensions.  It can contain
  one and  only one daughter  volume at any position  and orientation:
  the \TT{setup} volume.   This daughter volume can be  modeled by any
  other geometry model available at run-time in the library.

\item the  \texttt{geomtools::simple\_shaped\_model} class is designed
  to address very  usual cases : volume with a  simple shape like box,
  cylinder,  sphere\dots.  Optionally  it  can contain  some  daughters
  volumes.

\end{itemize}

Both            \texttt{geomtools::simple\_world\_model}           and
\texttt{geomtools::simple\_shaped\_model} are registered by default in
the \texttt{geomtools}' \emph{model factory}.

What we  need now is to  provide some configuration file  with all the
directives that reflect the  layout seen on figure \ref{fig:mf:0}. The
file will use by convention the \texttt{.geom} extension, note however
it  is  not  mandatory.  The  format  is  the  ASCII encoding  of  the
\texttt{datatools::utils::multi\_properties}  class. The  principle is
to provide one  section of properties per geometry  model described in
the setup. Here we will have two section : one for the top-level world
model, the  second for the daughter  box in it. Note  that blank lines
are ignored as  well as line starting with the  \# character. There is
an exception  with special meta-comments starting  with \verb+#@+ that
are        part        of        the       syntax        of        the
\texttt{datatools::utils::multi\_properties}        and       embedded
\texttt{datatools::utils::properties} objects.  This meta-comments are
\verb+#@description+ and \verb+#@config+ (see the example below).

We first  create a file named \TT{simple\_world\_1.geom}  and we write
its header as shown on sample \ref{sample:header:1}.

\begin{sample}
\VerbatimInput[frame=single,
numbers=left,
numbersep=2pt,
firstline=1,
lastline=3,
fontsize=\footnotesize,
showspaces=false]{\codingpath/simple_world_1.geom}
\caption{The header of the  \TT{simple\_world\_1.geom} file.}
\label{sample:header:1}
\end{sample}


\pn This meta information,  stored as meta-comments, inform the parser
for  the \\  \texttt{datatools::utils::multi\_properties}  object that
each section  will have  a main key  labeled with \verb+name+  and an
additional tag labeled \verb+type+.  The \emph{name} will be used as
the primary key to access  each geometry model from an internal look-up
table.   The \emph{type}  is a  character string  that  identifies the
unique  geometry model  class that  must  be used  to instantiate  the
corresponding geometry model and  its associated logical volume; it is
thus used by the internal model factory.

\pn Now we are done with the logistics of \emph{multi\_properties} and
\emph{factory} objects, we can describe  the two models we need.  Here
again there is an important constraint.  As the small red box is to be
inserted in the lard world green  box. The geometry model of the green
box will \emph{depend on} the existence  of the small red box.  On the
other side, in this hierarchy, the small red box does not need to know
anything about the large green box. In principle, we could have chosen
to put it in another  cylindrical purple universe.  So to reflect this
mother/daughter  dependency,   we  \textbf{must}  first   declare  the
geometry model  for the small red  box. Then only we  will provide the
definition of the model that will instantiate the green world.

\pn The  small red box  use a very  simple shape. More, as  a terminal
leaf of  the hierarchy, it contains  no daughter. This  simple case is
addressed  by  the   \texttt{simple\_shaped\_model}  provided  by  the
\texttt{geomtools} API.

\pn The file sample \ref{sample:section_srb:1} shows 
the section for the \emph{small red box}.

\begin{sample}[h]
\VerbatimInput[frame=single,
numbers=left,
numbersep=2pt,
firstline=10,
lastline=30,
fontsize=\footnotesize,
showspaces=false]{\codingpath/simple_world_1.geom}
\caption{The \emph{small red box}
  section of the  \TT{simple\_world\_1.geom} file.}
\label{sample:section_srb:1}
\end{sample}

\pn Note that the mandatory parameters are:
\begin{itemize}

\item \texttt{shape\_type}, \texttt{x}, \texttt{y}, \texttt{z} : for a
  full geometry description of the volume,

\item \texttt{material.ref}  for this  is a crucial  physical property
  for client application.\\ Remark:  this behavior should be change in
  the  future to support,  for debugging  purpose, a  default material
  when this property is missing in the file.
\end{itemize}

\pn       Visibility      parameters      (\texttt{visibility.hidden},
\texttt{visibility.color}) are  optional.  However, because  there are
used  by the  Gnuplot  based fast  visualization  program provided  in
\texttt{geomtools}, we recommend to use them. They will also be passed
to the GEANT4 Open-GL visualization driver.

\pn Believe it or not, these  few lines will generate all the software
machinery to  handle the  3D-box object, set  its dimension,  add some
properties in it, and allocate  the associated logical volume. This is
transparent to the  user. At the end of the  processing of these lines
by the geometry model factory, a new object named \TT{small\_red\_box}
is inserted  in a  dynamic internal database  for further  usage. This
object now just has to wait to be used by some other (mother) geometry
model. Be patient, the \emph{world} is coming !

\pn So  what about the \emph{world}  volume ? As  mentioned above, the
\texttt{geomtools::simple\_world\_model} has  been implemented in this
purpose : hosting a single 3D object in a boxed universe.

\pn  Let's   write  the  \emph{world}   section  !  The   file  sample
\ref{sample:section_world:1}  stores  the   directives  that  must  be
written \textbf{after} the \TT{small\_red\_box} section :
\begin{sample}
\VerbatimInput[frame=single,
numbers=left,
numbersep=2pt,
firstline=37,
lastline=88,
fontsize=\footnotesize,
showspaces=false]{\codingpath/simple_world_1.geom}
\caption{The \emph{world}
  section of the  \TT{simple\_world\_1.geom} file.}
\label{sample:section_world:1}
\end{sample}

\pn It can  be seen here that this model needs  more information to be
properly described.  Not only it requires the dimensions, material and
visibility parameters,  but it  also needs some  geometry informations
for the placement  of the setup volume it  contains. The configuration
properties are rather self explanatory.

\pn  The  C++   program  \ref{program:section_world:1}  illustrates  a
minimal  use   of  the  \texttt{geomtools::model\_factory}   class  to
construct  the   virtual  geometry  model  that   corresponds  to  the
directives  stored in the  \TT{simple\_world\_1.geom} file.   Once the
factory object has loaded the file and has been locked, the transcient
geometry hierarchy model is built and the program prints its structure
on the terminal.
  

\begin{program}[h]
\VerbatimInput[frame=single,
numbers=left,
numbersep=2pt,
firstline=1,
%%lastline=88,
fontsize=\footnotesize,
showspaces=false]{\codingpath/simple_world_1.cxx}
\caption{A program for the  construction of the virtual geometry setup
  described in the \TT{simple\_world\_1.geom} file.}
\label{program:section_world:1}
\end{program}


\pn The following  C++ program \ref{program:section_world:2} shows how
to pass the geometry hierarchy model built by the \emph{geometry model
  factory} to some special driver : a Gnuplot renderer (visualization)
and  a GDML  filter.  Figure  \ref{fig:mf:simple_world_1:a}  shows the
Gnuplot  based 3D-display originated  from the  program. We  have been
able  to   achieve  the   goal  in  figure   \label{fig:mf:0}.  Sample
\ref{sample:gdml:1}  shows  the contents  of  the  GDML  file that  is
generated by the GDML export driver object.


\begin{program}[hp]
\VerbatimInput[frame=single,
numbers=left,
numbersep=2pt,
firstline=1,
%%lastline=88,
fontsize=\footnotesize,
showspaces=false]{\codingpath/simple_world_2.cxx}
\caption{A  program  that  extends  the  functionnalities  of  program
  \ref{program:section_world:1} and  display a  simple 3D view  of the
  geometry. It produces  also a GDML file that  describes the geometry
  setup.}
\label{program:section_world:2}
\end{program}

\begin{figure}[h]
\begin{center}
\includegraphics[width=0.75\linewidth]{\imagepath/simple_world_1.jpeg}
\end{center}
\caption{The Gnuplot  display of the simple  virtual world constructed
  from   the   file   \texttt{simple\_world\_1.geom}  built   by   the
  \ref{program:section_world:2}
  program.}\label{fig:mf:simple_world_1:a}
\end{figure}

\begin{sample}[hp]
\VerbatimInput[frame=single,
numbers=left,
numbersep=2pt,
firstline=1,
%%lastline=88,
fontsize=\tiny,
showspaces=false]{\codingpath/simple_world_2.gdml.save}
\caption{The  GDML  file  generated  by  \ref{program:section_world:2}
  program from  the setup described  in the \TT{simple\_world\_1.geom}
  file. Here a default list of  materials is added by the driver. In a
  practical  case, a  list of  materials  is inserted  by an  external
  software agent.}
\label{sample:gdml:1}
\end{sample}

\clearpage

\subsection{A virtual world with two identical boxes in it}

We now address  a somewhat more complex setup :  we want two identical
small  red boxes to  be placed  within the  world volume  at different
(no overlapping) positions and orientations.

We  thus create  a new  \TT{setup\_2.geom} file  to describe  this new
geometry  layout.  Here  the description  of  the \TT{small\_red\_box}
geometry model is unchanged with regards to the previous setup (sample
\ref{sample:section_srb:1}). But now, as  we want to place two objects
within   the   \emph{world}  volume,   we   cannot   use  the   former
\texttt{geomtools::simple\_world\_model} geoemtry model.  In place, we
will  use a  \texttt{geomtools::simple\_shaped\_model}  model (with  a
\TT{box}  shape) but  we will  add some  directives in  the \TT{world}
section  to  inform  the  \emph{world}  volume that  it  contains  two
boxes.  To  achieve  this   we  use  special  directives  prefixed  by
\texttt{internal\_item}.

\pn The file  sample \ref{sample:section_world:2} shows the directives
that   must   be  written   in   the   \TT{world}   section  of   file
\TT{setup\_2.geom} in order to describe the new world volume :
\begin{sample}
\VerbatimInput[frame=single,
numbers=left,
numbersep=2pt,
firstline=37,
lastline=68,
fontsize=\footnotesize,
showspaces=false]{\codingpath/setup_2.geom}
\caption{The \emph{world}
  section of the  \TT{setup\_2.geom} file.}
\label{sample:section_world:2}
\end{sample}

Using   the   \texttt{setup\_construct\_and\_view.cxx}  program   (see
program  \ref{program:scav:0}),  we   obtain  the  display  in  figure
\ref{fig:setup_2:0}. 

\begin{program}[hp]
\VerbatimInput[frame=single,
numbers=left,
numbersep=2pt,
firstline=1,
fontsize=\footnotesize,
showspaces=false]{\codingpath/setup_construct_and_view.cxx}
\caption{The \texttt{setup\_construct\_and\_view.cxx} program.}
\label{program:scav:0}
\end{program}

\begin{figure}[h]
\begin{center}
\includegraphics[width=0.75\linewidth]{\imagepath/setup_2.jpeg}
\end{center}
\caption{The Gnuplot  display of the  virtual world constructed
  from   the   file   \texttt{setup\_2.geom}  built   by   the
  \ref{program:scav:0} program.}\label{fig:setup_2:0}
\end{figure}


\clearpage

\subsection{A virtual world with several objects of different types\dots}

Adding more  boxes is trivial, we  just have to  extend the \TT{world}
section's   \texttt{internal\_item.labels}   list   of   labels   with
additional      names      and      provide     the      corresponding
\texttt{internal\_item.model.XXX}                                   and
\texttt{internal\_item.placement.XXX}  rules.  But we  can also  get a
mix  of different  kind of  objects as  daughters of  the \emph{world}
volume.  

In  a new  \TT{setup\_3.geom}  file, let's  introduce,  after the  now
well-known  \TT{small\_red\_box} section,  a new  geometry  model that
represents a  long blue cylinder.  This  is done with  the file sample
\ref{sample:setup_3:0} that shows the  parameters of a new section for
the cylindric object.

\begin{sample}[h]
\VerbatimInput[frame=single,
numbers=left,
numbersep=2pt,
firstline=26,
lastline=34,
fontsize=\footnotesize,
showspaces=false]{\codingpath/setup_3.geom}
\caption{The \emph{long blue cylinder}
  section of the  \TT{setup\_3.geom} file.}
\label{sample:setup_3:0}
\end{sample}

The \TT{world} section now write like in sample \ref{sample:setup_3:1}.
\begin{sample}[h]
\VerbatimInput[frame=single,
numbers=left,
numbersep=2pt,
firstline=41,
lastline=60,
fontsize=\footnotesize,
showspaces=false]{\codingpath/setup_3.geom}
\caption{The \emph{world} section of the \TT{setup\_3.geom} file.}
\label{sample:setup_3:1}
\end{sample}

Now the \texttt{setup\_construct\_and\_view.cxx} program (program source 
\ref{program:scav:0}) displays the figure \ref{fig:setup_3:0}.
Easy isn't it ?

\begin{figure}[h]
\begin{center}
\includegraphics[width=0.75\linewidth]{\imagepath/setup_3.jpeg}
\end{center}
\caption{The Gnuplot  display of the  virtual world constructed
  from   the   file   \texttt{setup\_3.geom}  built   by   the
  \ref{program:scav:0} program.}\label{fig:setup_3:0}
\end{figure}

\clearpage

\subsection{A setup with more hierarchy levels}

This section describes a more complex setup. Now we will implement a
world volume that contains not only a long blue cylinder but also
a huge magenta cube that contains in turn three small red boxes
with arbitrary placements. 

File samples \ref{sample:setup_4:1} and \ref{sample:setup_4:2}
show the associated sections of a new file \TT{setup\_4.geom}.
The display of this three-level hierarchy setup can be sen on figure \ref{fig:setup_4:0}.

\begin{sample}[h]
\VerbatimInput[frame=single,
numbers=left,
numbersep=2pt,
firstline=41,
lastline=58,
fontsize=\footnotesize,
showspaces=false]{\codingpath/setup_4.geom}
\caption{The \emph{huge magenta cube}
  section of the  \TT{setup\_4.geom} file.}
\label{sample:setup_4:1}
\end{sample}

\begin{sample}[h]
\VerbatimInput[frame=single,
numbers=left,
numbersep=2pt,
firstline=65,
lastline=82,
fontsize=\footnotesize,
showspaces=false]{\codingpath/setup_4.geom}
\caption{The \emph{world} section of the \TT{setup\_4.geom} file.}
\label{sample:setup_4:2}
\end{sample}

\begin{figure}[h]
\begin{center}
\includegraphics[width=0.75\linewidth]{\imagepath/setup_4.jpeg}
\end{center}
\caption{The Gnuplot  display of the  virtual world constructed
  from   the   file   \texttt{setup\_4.geom}.}\label{fig:setup_4:0}
\end{figure}

\clearpage


%% end of use_cases.tex


\clearpage

\section{Conclusion}

\pn To be done.

\end{document}
%%%%%%%%%%%%%%%%%%%%%%%%%%%%%%%%%%%%%%%%%%%%%%%%%%%%%%%%%%%%%%%%

%% end of GeometryModellingTutorial.tex
