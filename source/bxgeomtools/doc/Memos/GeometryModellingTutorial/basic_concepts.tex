%% basic_concepts.tex

\section{Basic concepts}

\subsection{Logical volumes and physical volumes}

\pn  First of  all, the  \texttt{geomtools} GMI  provides  an abstract
interface that allows the description of the hierarchical relationship
between physical objects in a geometry setup.

\pn The key  concept is the \emph{logical volume}  which describes the
fundamental  geometry properties  of  a  3D-object in  a  way that  is
independant of its  placement in the whole setup.  This is illustrated
on figure  \ref{fig:lv:0} where  several \emph{objects} (instances or copies)  of a
the same \emph{type} are placed within a setup.

It  this crucial to  distinguish the  \emph{logical description}  of a
chair  from its \emph{physical  implementations} (placements)  in the
  virtual world :
  \begin{itemize}

  \item the \emph{logical volume} (the \emph{type}) concept implements the description of
    intrinsic properties od the geometry volume , including the shape, the material and other
    arbitrary traits that could be useful for client applications,

  \item the  \emph{physical volume} concept (the \emph{copy}) implements the description
    of the placement (position/rotation matrix) of an instance of some
    \emph{logical  volume}  in the  geometry  setup (the  instantiated
    \emph{object}) with respect of a mother logical volume.

  \end{itemize}

\begin{figure}[h]
\begin{center}
\scalebox{0.75}{\input{\pdftextpath/fig_lv_0.pdftex_t}}
\end{center}
\caption{Several copies of a same type of object can be placed in the virtual
geometry setup.}\label{fig:lv:0}
\end{figure}

So what is a \emph{logical  volume} ? Following the approach of the GDML langage
and GEANT4 modelling interface, we can list the informations that fully describe
an instance of logical volume :

\begin{itemize}

\item   a  unique  \emph{name}   that  will   allow  to   address  the
  \emph{logical volume}  object non-ambiguously in a  database that way contains
  many \emph{logical  volumes}.\\  \pn  Examples:  \TT{chair},  \TT{table},
  \TT{desk},     \TT{bedroom},     \TT{bathroom},    \TT{city\_house},
  \TT{cottage}\dots

\item  a 3D  shape that  will  define the  physical bounds  in the  3D
  virtual   space.\\  \pn   Examples:  a   \emph{box}   of  dimensions
  3$\times$2$\times$1.3 m$^3$,  a \emph{cylinder} of  radius $r$=25 cm
  and height $h$=75 cm\dots

\item An optional list of daughter volumes that are fully contained in
  the  bounding limits  (the 3D  shape)  of the  logical volume.  This
  implies that we know where  to place these daughter volumes. We thus
  need to provide  not only their own \emph{logical  volumes} but also
  their position  and orientation  (placement) in the  current logical
  volume that is called the \emph{mother} volume. The daughter volumes
  are \emph{physical volumes}.

\end{itemize}

Figure \ref{fig:lv:1} shows two  different descriptions of some simple
volumes.  Despite  their external envelopes  share the same  shape and
dimensions,  the  left  and   right  ''boxed''  logical  volumes  have
significant differences.   They do  not have the  same colour  (may be
because  they are  made with  different  materials) and  the blue  one
contains  some  daughter  box  objects  while  the  left  one  has  no
daughters. As the blue logical volume contains two tiny red boxes, we must
provide the coordinates (position/rotation) for the placement of these
daughter volumes. More, it is obvious that the red boxes are copies
(\emph{physical volumes} that share the same description; we must
also provide the description of this third \emph{logical volume}.

\begin{figure}[h]
\begin{center}
\scalebox{0.75}{\input{\pdftextpath/fig_lv_1.pdftex_t}}
\end{center}
\caption{Two  different  simple  logical  volumes.}\label{fig:lv:1}
\end{figure}

We see  here that  the full  description of a  geometry setup  will be
given by a  more or less complex hierarchy  of physical volumes nested
in logical  volumes, in turn nested  in physical volume  at the parent
level and so on\dots

Typically the geometry description looks like a large tree
of an arbitrary depth which depends on the number of nested
hierarchy levels:
\begin{ShellVerbatim}
Logical volume: "world.log" (top-level of the hierarchy)
|-- Material: "air"
|-- Colour: "transparent"
|-- Shape: "box" with x,y,z=(30,20,10) m
`-- Daughters:
    |-- Physical volume: "house_0.phys"
    |    |-- Position/rotation = (3,-2,0) m / R(Oz, 90°)
    |    `-- Logical volume: "house.log"
    |        |-- Material: "air"
    |        |-- Colour: "transparent"
    |        |-- Shape: "box" with x,y,z=(30,20,10) m
    |        `-- Daughters:
    |            |-- Physical: "ground_floor.phys"
    |            |   |-- Position/rotation = (1,2,0) m / R(Oz,0°)
    |            |   `-- Logical volume: "ground_floor.log"
    |            |       |-- Material: "concrete"
    |            |       |-- Colour: "gray"
    |            |       |-- Shape: "box" with x,y,z=(...) m
    |            |       `-- Daughters:
    |            |           |-- Physical volume: "kitchen.phys"
    |            |           |    |-- Position...
    |            |           |    `-- Logica volumel: "kitchen.log"
    :            :           :
    |-- Physical volume: "house_1.phys"
    |    |-- Position/rotation = (3,-2,0) m / R(Oz, 90°)
    |    `-- Logical volume: "house.log"
    |        |-- Material: "air"
    |        |-- Colour: "transparent"
    |        |-- Shape: "box" with x,y,z=(30,20,10) m
    |        `-- Daughters:
    |            |-- Physical volume: "ground_floor.phys"
    :            :
    |            `-- Physical volume: "last_floor.phys"
    :
\end{ShellVerbatim}
%%$

\pn Practically, the \texttt{geomtools} API provides
two classes:
\begin{itemize}
\item the \texttt{geomtools::logical\_volume} class,
\item the \texttt{geomtools::physical\_volume} class.
\end{itemize}
\pn Basically its  API is a clone  of what can be found  in the GEANT4
program library. This is natural as both APIs follow the GDML geometry
modelling approach. However \texttt{geomtools} allows to associate arbitrary
properties to any logical or physical object. This allows
client applications to benefit of some tools for storage and/or fetching
high-level meta-data. This mechanism is used to pass visualization informations
to some 3D-display program (visibility, colour\dots), material informations
for physics simulation programs (GEANT4), directives to the numbering scheme
manager (\emph{geometry mapping})\dots\ Some naming conventions are of course needed
to ease the extraction of arbitrary informations by topic. This
feature relies on the \texttt{datatools::properties} container class.
By essence, the availability of this functionnality makes this mechanism
extensible to other applications.

\subsection{Geometry models}

\pn If the \texttt{geomtools} API had proposes only a rewritting
of the GEANT4 interface and/or the GDML modelling approach, even with a few
more features added in it, it would have been of limited interest.
Indeed this API is \emph{based} on a similar interface to GEANT4/GDML, but
implements higher level functionalities.

The key  idea here is  to obtain a  very compact and efficient  way to
describe  a geometry  setup without  entering the  guts of  the nested
logical/physical  volume   hierarchy  and  the   expertise  needed  to
manipulate and navigate through this hierarchy. More, we would like to
implement some tools to:
\begin{itemize}

\item automate the construction of a transcient virtual geometry model
  using  a \emph{geometry  engine} that  uses  only a  limited set  of
  configuration  parameters  to  build   a    hierarchy  of
  3D-volumes of arbitrary complexity,

\item benefit of a collection  of generic objects that represents very
  often  used geometry patterns  : volumes  with very  familiar shapes
  (box, cylinder\dots), stacked volumes, replicated volumes, composite
  volumes (union, intersection, differences)\dots,

\item  automate a  simple  3D-rendering tool  for  fast debugging  and
  developpement,

\item automate the  conversion of any geometry model in  the format of
  another API (GDML/GEANT4, ROOT\dots),

\item automate the management of a numbering scheme for volumes in the
  hierarchy (\emph{geometry mapping}),

\item enable some arbitrary meta-data to be attached to any node
  of the hierarchy tree,

\item  enable  extensions  with  possible new  geometry  primitives  or
  hard-coded descriptions of new  logical volumes if genericity cannot
  be achieved,

\item interoperability between generic components and hard-coded components,

\item be human-friendly with ASCII file based configurations (don't even think about XML!),

\item hide  the complex memory  management of the  transcient geometry
  model\footnote{You may  have a look on  GEANT4 detector construction
    and all the pointers users have to play with !}.

\end{itemize}

Within  GDML,  part  of  this features  (flexibility,  human  readable
configuration  files,  some  filters  to  GEANT4  and  ROOT\dots)  are
addressed   through  the   grammar  and   syntax  of   this  XML-based
language. However, the manipulation of XML files turns to be difficult
as  the  complexity of  the  geometry  system  increases. Despite  the
possibility  to use  \emph{parametrized  logical volumes},  on-the-fly
computed  positionning  is  limited.   More  it  does  not  provide  a
standalone  transcient  geometry  model  :  you  have  to  choose  the
transcient model from the GEANT4  or the ROOT library but it is difficult
to used both systems in  cooperation  within  the same  program
or  in  the context  of  a
third-party application that has its own modelling scheme and approach
(your data  analysis and event reconstruction  for example).  However,
GDML files are a  good interface medium and we will use this technique
 as the core of the interface between geomtools and GEANT4.

As  the  handling of  many  logical/physical  volumes  is complex  and
request some programming expertise (memory management, pointers, knowledge
of some specific API\dots),
\texttt{geomtools} proposes a special concept to automate and hide
most parts of  this low-level techniques, still providing  the user or
an external application a way  to manipulate the concepts of hierarchical
geometry  modelling.  A  new  interface  has been  implemented  :  the
\emph{geometry model}.

A  \emph{geometry  model}  has  the  responsability  to  describe  the
characteristics  of a  given \emph{logical  volume}, in  such a  way a
logical volume is always  instantiated through its associated geometry
model. However, a  geometry model is for the logical  volume what is a
class for  an object.  Thus  a geometry model may  have parametrization
facilities that enable to  instantiate different kinds of \emph{logical
  volumes}; such logical volumes will be rather similar because
they are managed/created by the same \emph{geometry driver} and
they will \emph{behaves} in the same way.

The difference between the \emph{geometry model} and the \emph{logical
  volume} concepts is not obvious for very simple 3D objects (a simple
box or cylinder without any  daughter volumes): the geometry model for
a simple box-shaped  volume made of copper will  be responsible of the
instantiation of the logical volume  made of a box shape associated to
copper     material.     The      figure     \ref{fig:gm:0}     represents     a
\texttt{simple\_boxed\_model}   geometry  model   which   is  able   to
instantiate a logical volume  named \TT{my\_green\_box}.  The user (on
the    left)    just    has    to    send    a    request    to    the
\texttt{simple\_boxed\_model} object that behaves like a \emph{logical
  volume     factory}    and     instantiates     automatically    the
\TT{my\_box}  logical  volume  given some  specific  parameters
($w$, $h$, $d$, colour, material\dots) that are passed when the user's
request is submitted.

\begin{figure}[h]
\begin{center}
\scalebox{0.75}{\input{\pdftextpath/fig_gm_0.pdftex_t}}
\end{center}
\caption{Instantiation of  a logical  volume through a  geoemtry model
  driver.}\label{fig:gm:0}
\end{figure}


\pn  Introducing   the  \emph{geometry  model}  layer   just  adds  an
intermediate step  but does not  add functionnalities.  We  could have
directly   create  the   logical  volume   by  hand   using  low-level
functionnalities of the API (figure \ref{fig:gm:1}).

\begin{figure}[h]
\begin{center}
\scalebox{0.75}{\input{\pdftextpath/fig_gm_1.pdftex_t}}
\end{center}
\caption{Instantiation of  a logical  volume through the native
API.}\label{fig:gm:1}
\end{figure}

However, as soon  as we want to manipulate some
complex logical  volumes, the  difference is fondamental.   Suppose we
want to  build a logical volume  that stacks three  boxes of different
dimensions along  an arbitrary axis. To  get this in  GEANT4, you will
have to  instantiate first the three logical  volumes corresponding to
each box. Then you will create a logical volume, choose a shape for it
(say  a  box), compute  its  dimensions  from  the dimensions  of  the
internal boxes you want to  stack, position the daughter boxes (yes we
speak about daughter  physical volumes) along an arbitrary  axis, check that there
is no overlapping volumes and that daughters are fully contained in the
mother box. Such case is illustrated on figure \ref{fig:gm:2}.


\begin{figure}[h]
\begin{center}
\scalebox{0.75}{\input{\pdftextpath/fig_gm_2.pdftex_t}}
\end{center}
\caption{Instantiation of  a complex logical  volume through the native
API.}\label{fig:gm:2}
\end{figure}

This is some work ! Each time you will  have to stack objects in this way, you
will have to reproduce this algorithm, manage the memory and the pointers associated to
the volumes, compute all  requested geometry parameters. In real life,
it appears  that stacking  volume is a  frequent operation :  in fact
this is a current  \emph{geometry modelling pattern}.  This is exactly
the moment for a dedicated  \emph{geometry model} to enter the scene !
It is possible  to implement a generic algorithm of  which the task is
to  perform all  the  complex operations  needed  to obtain  a set  of
stacked  volumes enclosed in  some mother  volume.  Not  only the geometry
model for constructing such stacked volumes
will use this algorithm, it  will also
manage all  the internals  : memory stuff (pointers, memory allocation),
mother/daughter relationships,
conventionnal naming of the internal objects/volumes (figure \ref{fig:gm:3}).

\begin{figure}[h]
\begin{center}
\scalebox{0.75}{\input{\pdftextpath/fig_gm_3.pdftex_t}}
\end{center}
\caption{Instantiation of  a complex logical  volume through a smart
geometry model driver.}\label{fig:gm:3}
\end{figure}

Practically,  the designer  of  a virtual  geometry  setup will  never
directly manipulate logical and/or physical volumes. He will deal only
with geometry models.  This approach defers the complex technical part
of the  modelling to the \texttt{geomtools}  embedded geometry engine.
The  user can  concentrate only  on  the building  of the  fondamental
objects that  enter the  composition of the  setup and  their relative
relationships in terms of hierarchy.

Of  course, logical  and physical  volumes  still exist  and are  used
within the  internals of the \texttt{geomtools}  engine. These objects
are the  core of the hierarchy  geometry tree.  So it  is natural that
developpers  will still have  to handle  logical and  physical volumes
when  they  design  new  geometry  model/driver  classes  and  propose
extension to the base library with its already available models.

A \emph{model factory}  class has been implemented as  the main engine
responsible of  the construction of a geometry  models.  Given some
geometry configuration ASCII files, it automates the allocation of requested
geometry  models,  checks (partially)  the  coherence  of the  system,
generates the associated logical and physical volumes and makes a full
transcient hierarchical geometry model  available to the user.  The  class is named
\texttt{geomtools::model\_factory}.

More  all generic  and primitive  geometry models  available  from the
library benefit of an  automated class registration mechanism based on
some internal  lookup table (a static singleton). The  instantiation of geometry  models is
thus  completely transparent,  as  well as  memory management  issues.
This mechanism also extends  to the geometry model classes implemented
by developpers  to handle  special cases where  no combination  of the
existing pre-registred models can  cover new needs.  This makes the
system rather generic and extensible.

%% end of basic_concepts.tex
