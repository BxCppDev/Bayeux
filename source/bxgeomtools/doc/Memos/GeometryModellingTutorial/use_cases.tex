%% use_cases.tex

\section{Use cases}

Now it is time to play  with geometry models and the geometry factory.
We will start  by a very simple case and will  introduce more and more
complexity progressively.

\subsection{A box in a virtual world}

We  want to achieve  the construction  of the  virtual setup  shown on
figure \ref{fig:mf:0}.   We have here  a simple top level  large green
box (the \emph{world}) with  given dimensions and auxiliary properties
(color,  material\dots).   This \emph{world}  box  contains a  single
smaller red  box with its  own geometry and auxiliary  properties. The
placement of the red box inside  the mother green box is determined by
the position  of the center  $O'$ of the  red box with respect  to the
center $O$  of the green box.   The orientation of the  red box inside
the mother green box can  be determined by the Euler angles associated
to  the rotation matrix  that transform  the ($0xyz$)  world reference
frame axis into the ($0'x'y'z'$) daughter frame axis.


\begin{figure}[h]
\begin{center}
\scalebox{0.75}{\input{\pdftextpath/fig_mf_0.pdftex_t}}
\end{center}
\caption{A very simple world.}\label{fig:mf:0}
\end{figure}

With  \texttt{geomtools}, the  easiest way  to construct  this virtual
geometry setup is to use two available geometry model classes :
\begin{itemize} 

\item  the \texttt{geomtools::simple\_world\_model} class  is designed
  to model a  top-level box with arbitrary dimensions.  It can contain
  one and  only one daughter  volume at any position  and orientation:
  the \TT{setup} volume.   This daughter volume can be  modeled by any
  other geometry model available at run-time in the library.

\item the  \texttt{geomtools::simple\_shaped\_model} class is designed
  to address very  usual cases : volume with a  simple shape like box,
  cylinder,  sphere\dots.  Optionally  it  can contain  some  daughters
  volumes.

\end{itemize}

Both            \texttt{geomtools::simple\_world\_model}           and
\texttt{geomtools::simple\_shaped\_model} are registered by default in
the \texttt{geomtools}' \emph{model factory}.

What we  need now is to  provide some configuration file  with all the
directives that reflect the  layout seen on figure \ref{fig:mf:0}. The
file will use by convention the \texttt{.geom} extension, note however
it  is  not  mandatory.  The  format  is  the  ASCII encoding  of  the
\texttt{datatools::utils::multi\_properties}  class. The  principle is
to provide one  section of properties per geometry  model described in
the setup. Here we will have two section : one for the top-level world
model, the  second for the daughter  box in it. Note  that blank lines
are ignored as  well as line starting with the  \# character. There is
an exception  with special meta-comments starting  with \verb+#@+ that
are        part        of        the       syntax        of        the
\texttt{datatools::utils::multi\_properties}        and       embedded
\texttt{datatools::utils::properties} objects.  This meta-comments are
\verb+#@description+ and \verb+#@config+ (see the example below).

We first  create a file named \TT{simple\_world\_1.geom}  and we write
its header as shown on sample \ref{sample:header:1}.

\begin{sample}
\VerbatimInput[frame=single,
numbers=left,
numbersep=2pt,
firstline=1,
lastline=3,
fontsize=\footnotesize,
showspaces=false]{\codingpath/simple_world_1.geom}
\caption{The header of the  \TT{simple\_world\_1.geom} file.}
\label{sample:header:1}
\end{sample}


\pn This meta information,  stored as meta-comments, inform the parser
for  the \\  \texttt{datatools::utils::multi\_properties}  object that
each section  will have  a main key  labeled with \verb+name+  and an
additional tag labeled \verb+type+.  The \emph{name} will be used as
the primary key to access  each geometry model from an internal look-up
table.   The \emph{type}  is a  character string  that  identifies the
unique  geometry model  class that  must  be used  to instantiate  the
corresponding geometry model and  its associated logical volume; it is
thus used by the internal model factory.

\pn Now we are done with the logistics of \emph{multi\_properties} and
\emph{factory} objects, we can describe  the two models we need.  Here
again there is an important constraint.  As the small red box is to be
inserted in the lard world green  box. The geometry model of the green
box will \emph{depend on} the existence  of the small red box.  On the
other side, in this hierarchy, the small red box does not need to know
anything about the large green box. In principle, we could have chosen
to put it in another  cylindrical purple universe.  So to reflect this
mother/daughter  dependency,   we  \textbf{must}  first   declare  the
geometry model  for the small red  box. Then only we  will provide the
definition of the model that will instantiate the green world.

\pn The  small red box  use a very  simple shape. More, as  a terminal
leaf of  the hierarchy, it contains  no daughter. This  simple case is
addressed  by  the   \texttt{simple\_shaped\_model}  provided  by  the
\texttt{geomtools} API.

\pn The file sample \ref{sample:section_srb:1} shows 
the section for the \emph{small red box}.

\begin{sample}[h]
\VerbatimInput[frame=single,
numbers=left,
numbersep=2pt,
firstline=10,
lastline=30,
fontsize=\footnotesize,
showspaces=false]{\codingpath/simple_world_1.geom}
\caption{The \emph{small red box}
  section of the  \TT{simple\_world\_1.geom} file.}
\label{sample:section_srb:1}
\end{sample}

\pn Note that the mandatory parameters are:
\begin{itemize}

\item \texttt{shape\_type}, \texttt{x}, \texttt{y}, \texttt{z} : for a
  full geometry description of the volume,

\item \texttt{material.ref}  for this  is a crucial  physical property
  for client application.\\ Remark:  this behavior should be change in
  the  future to support,  for debugging  purpose, a  default material
  when this property is missing in the file.
\end{itemize}

\pn       Visibility      parameters      (\texttt{visibility.hidden},
\texttt{visibility.color}) are  optional.  However, because  there are
used  by the  Gnuplot  based fast  visualization  program provided  in
\texttt{geomtools}, we recommend to use them. They will also be passed
to the GEANT4 Open-GL visualization driver.

\pn Believe it or not, these  few lines will generate all the software
machinery to  handle the  3D-box object, set  its dimension,  add some
properties in it, and allocate  the associated logical volume. This is
transparent to the  user. At the end of the  processing of these lines
by the geometry model factory, a new object named \TT{small\_red\_box}
is inserted  in a  dynamic internal database  for further  usage. This
object now just has to wait to be used by some other (mother) geometry
model. Be patient, the \emph{world} is coming !

\pn So  what about the \emph{world}  volume ? As  mentioned above, the
\texttt{geomtools::simple\_world\_model} has  been implemented in this
purpose : hosting a single 3D object in a boxed universe.

\pn  Let's   write  the  \emph{world}   section  !  The   file  sample
\ref{sample:section_world:1}  stores  the   directives  that  must  be
written \textbf{after} the \TT{small\_red\_box} section :
\begin{sample}
\VerbatimInput[frame=single,
numbers=left,
numbersep=2pt,
firstline=37,
lastline=88,
fontsize=\footnotesize,
showspaces=false]{\codingpath/simple_world_1.geom}
\caption{The \emph{world}
  section of the  \TT{simple\_world\_1.geom} file.}
\label{sample:section_world:1}
\end{sample}

\pn It can  be seen here that this model needs  more information to be
properly described.  Not only it requires the dimensions, material and
visibility parameters,  but it  also needs some  geometry informations
for the placement  of the setup volume it  contains. The configuration
properties are rather self explanatory.

\pn  The  C++   program  \ref{program:section_world:1}  illustrates  a
minimal  use   of  the  \texttt{geomtools::model\_factory}   class  to
construct  the   virtual  geometry  model  that   corresponds  to  the
directives  stored in the  \TT{simple\_world\_1.geom} file.   Once the
factory object has loaded the file and has been locked, the transcient
geometry hierarchy model is built and the program prints its structure
on the terminal.
  

\begin{program}[h]
\VerbatimInput[frame=single,
numbers=left,
numbersep=2pt,
firstline=1,
%%lastline=88,
fontsize=\footnotesize,
showspaces=false]{\codingpath/simple_world_1.cxx}
\caption{A program for the  construction of the virtual geometry setup
  described in the \TT{simple\_world\_1.geom} file.}
\label{program:section_world:1}
\end{program}


\pn The following  C++ program \ref{program:section_world:2} shows how
to pass the geometry hierarchy model built by the \emph{geometry model
  factory} to some special driver : a Gnuplot renderer (visualization)
and  a GDML  filter.  Figure  \ref{fig:mf:simple_world_1:a}  shows the
Gnuplot  based 3D-display originated  from the  program. We  have been
able  to   achieve  the   goal  in  figure   \label{fig:mf:0}.  Sample
\ref{sample:gdml:1}  shows  the contents  of  the  GDML  file that  is
generated by the GDML export driver object.


\begin{program}[hp]
\VerbatimInput[frame=single,
numbers=left,
numbersep=2pt,
firstline=1,
%%lastline=88,
fontsize=\footnotesize,
showspaces=false]{\codingpath/simple_world_2.cxx}
\caption{A  program  that  extends  the  functionnalities  of  program
  \ref{program:section_world:1} and  display a  simple 3D view  of the
  geometry. It produces  also a GDML file that  describes the geometry
  setup.}
\label{program:section_world:2}
\end{program}

\begin{figure}[h]
\begin{center}
\includegraphics[width=0.75\linewidth]{\imagepath/simple_world_1.jpeg}
\end{center}
\caption{The Gnuplot  display of the simple  virtual world constructed
  from   the   file   \texttt{simple\_world\_1.geom}  built   by   the
  \ref{program:section_world:2}
  program.}\label{fig:mf:simple_world_1:a}
\end{figure}

\begin{sample}[hp]
\VerbatimInput[frame=single,
numbers=left,
numbersep=2pt,
firstline=1,
%%lastline=88,
fontsize=\tiny,
showspaces=false]{\codingpath/simple_world_2.gdml.save}
\caption{The  GDML  file  generated  by  \ref{program:section_world:2}
  program from  the setup described  in the \TT{simple\_world\_1.geom}
  file. Here a default list of  materials is added by the driver. In a
  practical  case, a  list of  materials  is inserted  by an  external
  software agent.}
\label{sample:gdml:1}
\end{sample}

\clearpage

\subsection{A virtual world with two identical boxes in it}

We now address  a somewhat more complex setup :  we want two identical
small  red boxes to  be placed  within the  world volume  at different
(no overlapping) positions and orientations.

We  thus create  a new  \TT{setup\_2.geom} file  to describe  this new
geometry  layout.  Here  the description  of  the \TT{small\_red\_box}
geometry model is unchanged with regards to the previous setup (sample
\ref{sample:section_srb:1}). But now, as  we want to place two objects
within   the   \emph{world}  volume,   we   cannot   use  the   former
\texttt{geomtools::simple\_world\_model} geoemtry model.  In place, we
will  use a  \texttt{geomtools::simple\_shaped\_model}  model (with  a
\TT{box}  shape) but  we will  add some  directives in  the \TT{world}
section  to  inform  the  \emph{world}  volume that  it  contains  two
boxes.  To  achieve  this   we  use  special  directives  prefixed  by
\texttt{internal\_item}.

\pn The file  sample \ref{sample:section_world:2} shows the directives
that   must   be  written   in   the   \TT{world}   section  of   file
\TT{setup\_2.geom} in order to describe the new world volume :
\begin{sample}
\VerbatimInput[frame=single,
numbers=left,
numbersep=2pt,
firstline=37,
lastline=68,
fontsize=\footnotesize,
showspaces=false]{\codingpath/setup_2.geom}
\caption{The \emph{world}
  section of the  \TT{setup\_2.geom} file.}
\label{sample:section_world:2}
\end{sample}

Using   the   \texttt{setup\_construct\_and\_view.cxx}  program   (see
program  \ref{program:scav:0}),  we   obtain  the  display  in  figure
\ref{fig:setup_2:0}. 

\begin{program}[hp]
\VerbatimInput[frame=single,
numbers=left,
numbersep=2pt,
firstline=1,
fontsize=\footnotesize,
showspaces=false]{\codingpath/setup_construct_and_view.cxx}
\caption{The \texttt{setup\_construct\_and\_view.cxx} program.}
\label{program:scav:0}
\end{program}

\begin{figure}[h]
\begin{center}
\includegraphics[width=0.75\linewidth]{\imagepath/setup_2.jpeg}
\end{center}
\caption{The Gnuplot  display of the  virtual world constructed
  from   the   file   \texttt{setup\_2.geom}  built   by   the
  \ref{program:scav:0} program.}\label{fig:setup_2:0}
\end{figure}


\clearpage

\subsection{A virtual world with several objects of different types\dots}

Adding more  boxes is trivial, we  just have to  extend the \TT{world}
section's   \texttt{internal\_item.labels}   list   of   labels   with
additional      names      and      provide     the      corresponding
\texttt{internal\_item.model.XXX}                                   and
\texttt{internal\_item.placement.XXX}  rules.  But we  can also  get a
mix  of different  kind of  objects as  daughters of  the \emph{world}
volume.  

In  a new  \TT{setup\_3.geom}  file, let's  introduce,  after the  now
well-known  \TT{small\_red\_box} section,  a new  geometry  model that
represents a  long blue cylinder.  This  is done with  the file sample
\ref{sample:setup_3:0} that shows the  parameters of a new section for
the cylindric object.

\begin{sample}[h]
\VerbatimInput[frame=single,
numbers=left,
numbersep=2pt,
firstline=26,
lastline=34,
fontsize=\footnotesize,
showspaces=false]{\codingpath/setup_3.geom}
\caption{The \emph{long blue cylinder}
  section of the  \TT{setup\_3.geom} file.}
\label{sample:setup_3:0}
\end{sample}

The \TT{world} section now write like in sample \ref{sample:setup_3:1}.
\begin{sample}[h]
\VerbatimInput[frame=single,
numbers=left,
numbersep=2pt,
firstline=41,
lastline=60,
fontsize=\footnotesize,
showspaces=false]{\codingpath/setup_3.geom}
\caption{The \emph{world} section of the \TT{setup\_3.geom} file.}
\label{sample:setup_3:1}
\end{sample}

Now the \texttt{setup\_construct\_and\_view.cxx} program (program source 
\ref{program:scav:0}) displays the figure \ref{fig:setup_3:0}.
Easy isn't it ?

\begin{figure}[h]
\begin{center}
\includegraphics[width=0.75\linewidth]{\imagepath/setup_3.jpeg}
\end{center}
\caption{The Gnuplot  display of the  virtual world constructed
  from   the   file   \texttt{setup\_3.geom}  built   by   the
  \ref{program:scav:0} program.}\label{fig:setup_3:0}
\end{figure}

\clearpage

\subsection{A setup with more hierarchy levels}

This section describes a more complex setup. Now we will implement a
world volume that contains not only a long blue cylinder but also
a huge magenta cube that contains in turn three small red boxes
with arbitrary placements. 

File samples \ref{sample:setup_4:1} and \ref{sample:setup_4:2}
show the associated sections of a new file \TT{setup\_4.geom}.
The display of this three-level hierarchy setup can be sen on figure \ref{fig:setup_4:0}.

\begin{sample}[h]
\VerbatimInput[frame=single,
numbers=left,
numbersep=2pt,
firstline=41,
lastline=58,
fontsize=\footnotesize,
showspaces=false]{\codingpath/setup_4.geom}
\caption{The \emph{huge magenta cube}
  section of the  \TT{setup\_4.geom} file.}
\label{sample:setup_4:1}
\end{sample}

\begin{sample}[h]
\VerbatimInput[frame=single,
numbers=left,
numbersep=2pt,
firstline=65,
lastline=82,
fontsize=\footnotesize,
showspaces=false]{\codingpath/setup_4.geom}
\caption{The \emph{world} section of the \TT{setup\_4.geom} file.}
\label{sample:setup_4:2}
\end{sample}

\begin{figure}[h]
\begin{center}
\includegraphics[width=0.75\linewidth]{\imagepath/setup_4.jpeg}
\end{center}
\caption{The Gnuplot  display of the  virtual world constructed
  from   the   file   \texttt{setup\_4.geom}.}\label{fig:setup_4:0}
\end{figure}

\clearpage


%% end of use_cases.tex
