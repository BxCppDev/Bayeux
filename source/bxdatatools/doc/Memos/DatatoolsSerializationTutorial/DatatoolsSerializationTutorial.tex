%% DatatoolsSerializationTutorial.tex
%%
%%
\documentclass[a4paper,12pt]{article}

\usepackage[T1]{fontenc} 
\usepackage{ucs} 
\usepackage[utf8x]{inputenc} 
%%french%%\usepackage[frenchb]{babel}
\usepackage{amsmath}
\usepackage{amssymb}
\usepackage{latexsym}
\usepackage{verbatim}
\usepackage{moreverb}
\usepackage{fancyvrb}
\usepackage{alltt} 
\usepackage{eurosym} 
\usepackage{hyperref}
\usepackage{colortbl}
\usepackage{epsfig}
\usepackage{graphicx}
\usepackage{pgf}
\usepackage{float}

\addtolength{\textwidth}{+2cm}
\addtolength{\textheight}{+3cm}
\addtolength{\topmargin}{-1.5cm}
\addtolength{\oddsidemargin}{-1cm}

\newcommand{\basepath}{.}
\newcommand{\imagepath}{\basepath/images}
\newcommand{\codingpath}{\basepath/coding}
\newcommand{\pdftextpath}{\basepath/pdftex_t}
\newcommand{\pdftexpath}{\basepath/pdftex}

%% declare_verbatim.tex
%% -*- mode: latex;-*-
%%
\DefineVerbatimEnvironment%
{ShellVerbatim}{Verbatim}%
{fontsize=\small,%
frame=single,%
framesep=2mm,%
framerule=0.25mm,%
labelposition=topline,%
numbers=none%
}

\DefineVerbatimEnvironment%
  {PathVerbatim}{Verbatim}%
{fontsize=\small,%
frame=single,%
framesep=1mm,%
framerule=0.15mm,%
numbers=none%
}

\DefineVerbatimEnvironment%
{CppVerbatim}{Verbatim}%
{commandchars=\\\{\},%
fontsize=\small,%
frame=single,%
framesep=2mm,%
framerule=0.25mm,%
labelposition=topline,
numbers=left,%
numbersep=2pt%
}

%% end of declare_verbatim.tex


%% Example :
%% \VerbatimInput[frame=single,
%% numbers=left,
%% numbersep=2pt,
%% firstline=1,
%% fontsize=\small,
%% showspaces=false]{\codingpath/code_snippet.cxx}
%% \caption{Some code snippet.}\label{fig:code_snippet}

\newcommand{\pn}{\par\noindent}
\newcommand{\TT}[1]{"\texttt{#1}"}

\title{Serialization with the \texttt{datatools} program library\\%
{\small{(Software/DatatoolsSerializationTutorial -- version 0.1)}}}
\author{F. Mauger <\texttt{mauger@lpccaen.in2p3.fr}>}
\date{2011-11-29}

%%%%%%%%%%%%%%%%%%%%%%%%%%%%%%%%%%%%%%%%%%%%%%%%%%%%%%%%%%%%%%%%
\begin{document}

\maketitle

\begin{abstract}
This note  explains how to implement class serialization functionnality
using the \texttt{datatools} program library serialization interface.
\end{abstract}

\section{Introduction}

\pn Serialization means  storing and/or loading data to  and/or from a
storage media. Typical storage media are files that we use everyday to
save some data  produced by some program $A$ then  reload it in memory
within program $B$ and so on and so on.

\pn The \texttt{datatools} program  library proposes some tools to add
serialization functionnalities  in (mostly)  any user classes.  As its
native I/O  system, it uses the serialization  concepts and techniques
from the Boost/Serialization library.


\vskip 5mm
\pn
Subversion repository:\\
\texttt{https://nemo.lpc-caen.in2p3.fr/svn/datatools/} \\
in the \texttt{doc/Memos/DatatoolsSerializationTutorial} subdirectory
\pn
DocDB reference: \texttt{NemoDocDB-doc-2003}
\pn
References: 
\begin{itemize}
\item The Boost/Serialization library documentation (\texttt{http://www.boost.org/}),
\item \textit{Using container objects in} \texttt{datatools}
(\texttt{NemoDocDB-doc-1997}).
\end{itemize}
\clearpage

\section{Basic concepts}

\subsection{Serialization with the Boost/Serialization  library}

The Boost/Serialization  library by Robert  Ramey is based  on elegant
and  powerful  concepts  and  thus  provides  smart  tools  to  enable
serialization  of arbitrary  objects.  The  core concept  is  that the
serialization  format  (ASCII,  XML,  binary\dots or  any  other  user
format) is  distinct from the  storage media (streams/files)  and thus
can be  redirect to  whatever media the  user chooses :  files, memory
region, containers of bytes\dots

Practically,  if  we  use  a  class, we  must  implement  the  special
\texttt{serialize} template  method that will  allow the serialization
(input  and output)  of all  (or  only part  of) data  members of  the
class. It is up to the user  to choose what he want to store/load from
the specialized serialize method.

The Boost/Serialization library has many  features that are out of the
scope of this document. However we can mention a few ones:
\begin{itemize}
  
\item support for versioning of serialization for each class,

\item support for inherited  serialization between mother and daughter
  classes,

\item support for  all basic types in C++  (for example using portable
  \emph{typedefs} from the \texttt{boost/cstdint.hpp} header file),

\item support for some basic STL containers (vector, list, map, string\dots),

\item  support   for  cross-references  through   pointers  within  an
  serializable object or between different serializable objects,

\item support  for standard  ASCII and XML  input and  output archives
  contribution external library  implementing support for binary input
  and output archives,

\item support for \emph{by-pointer} serialization of objects belonging to
  an inherited class hierarchy.

\end{itemize}

\pn  It should be  noticed that  all of  these features  are typically
implemented through one single template method per class and need only
a few C++ headers that must  be included at the beginning of the class
header file.

\pn The  code, based on templates,  is generated at  compile time with
strict  type checking. There  is no  need for  the generation  of some
complex class dictionary needed  by some introspection mechanism as it
is  done   by  tools  such   as  \texttt{rootcint}  within   the  ROOT
framework. In  this way,  serialization is made  very easy  using with
very limited intrusive  code in user's classes and  no external tools.
Note also  that, within an  archive, Boost serialization  implements a
memory  tracking   algorithm  that  enables   to  serialize  transient
references (pointers) between objects.

\subsection{A practical case}

The technique to add serialization  support to a given class is simple
but one must strictly respect a few rules, particularly concerning the
headers to  include. The sample  \ref{sample:bs:0} shows how to  add a
\emph{serialization} method to the  interface of a given class (header
file \texttt{data.h}).   It can be seen that  some special declaration
instructions  have  been added  in  the  private  part of  the  class'
interface. Particularly, the \texttt{serialize} template method is the
main  interface  for   Boost/Serialization.  Free  access  to  private
attributes of the class is  garanteed by the declaration of the friend
\texttt{boost::serialization::access} class.

Here the implementation of the class interface is splitted in two parts:
\begin{itemize}

\item   the  usual   implementation  of   the   methods  (constructor,
  getter/setter)   of    the   class   is   shown    in   the   sample
  \ref{sample:bs:1} (source file \texttt{data.cc}),

\item the definition of the \texttt{serialize}   template method is provided
  apart and shown    in   the   sample
  \ref{sample:bs:2} (definition file \texttt{data.ipp}).
\end{itemize}

This implementation  splitting approach is recommended  for it enables,
on   user  request,   to  use   an  alternative   definition   of  the
\texttt{serialize} template  method, without modification  of the core
class machinery.

\begin{sample}[h]
\VerbatimInput[frame=single,
numbers=left,
numbersep=2pt,
firstline=4,
lastline=55,
fontsize=\footnotesize,
showspaces=false]{\codingpath/data.h}
\caption{The interface of a \texttt{data} class enriched with
Boost/Serialization support.
}
\label{sample:bs:0}
\end{sample}

\begin{sample}[h]
\VerbatimInput[frame=single,
numbers=left,
numbersep=2pt,
firstline=1,
fontsize=\footnotesize,
showspaces=false]{\codingpath/data.cc}
\caption{The implementation of the \texttt{data} class. Here we do not
put any specific Boost/Serialization related material.
}
\label{sample:bs:1}
\end{sample}

\begin{sample}[h]
\VerbatimInput[frame=single,
numbers=left,
numbersep=2pt,
firstline=2,
fontsize=\footnotesize,
showspaces=false]{\codingpath/data.ipp}
\caption{The    definition   of   the    \texttt{serialize}   template
  method. Note the  use of the \texttt{\&} operator  which is the core
  bi-directionnal  I/O archive  operator in  Boost/Serialization. Also
  the    special     \emph{key-value}    pair    handling    construct
  \texttt{boost::serialization::make\_nvp("XXX",  attribute)}  is used
  to enable XML archive formatting. }
\label{sample:bs:2}
\end{sample}

%\clearpage

The program \ref{program:bs:0} first serialize (store) a  \texttt{data} object
within a Boost text archive. It then deserialize (load/restore) the object
from the  Boost text archive. The output file contents is shown in sample
\ref{sample:bs:0b}.

\begin{program}[h]
\VerbatimInput[frame=single,
numbers=left,
numbersep=2pt,
firstline=1,
%lastline=53,
fontsize=\footnotesize,
showspaces=false]{\codingpath/data_1.cxx}
\caption{A program that serializes and deserializes an 
object of \texttt{data} class using text archives.
}
\label{program:bs:0}
\end{program}

\begin{sample}[h]
\VerbatimInput[frame=single,
numbers=left,
numbersep=2pt,
firstline=1,
fontsize=\footnotesize,
showspaces=false]{\codingpath/stored_data.txt.out}
\caption{The output file with a Boost text archive produced by program
  \ref{program:bs:0}.      Here       the      leading      \texttt{22
    serialization::archive 9}  string is the  Boost archive identifier
  (archive's  header  with  version  9 in  Boost version 1.47.0),  following
  \texttt{0 0} are flags to identify the object type and ID within the
  archive  (this  could  be  used  in  the  case  of  cross-referenced
  objects),  then  \texttt{1  63  666...}  are  the  serialized  class
  attributes values, i.e. the data. }
\label{sample:bs:0b}
\end{sample}

The    program   \ref{program:bs:1}    behaves   like    the   program
\ref{program:bs:0}  but  uses Boost  XML  archives  in  place of  text
ones. The output file contents is shown in sample \ref{sample:bs:1b}.

\begin{program}[h]
\VerbatimInput[frame=single,
numbers=left,
numbersep=2pt,
firstline=1,
%lastline=53,
fontsize=\footnotesize,
showspaces=false]{\codingpath/data_2.cxx}
\caption{A program that serializes and deserializes an 
object of \texttt{data} class using XML archives.
}
\label{program:bs:1}
\end{program}

\begin{sample}[h]
\VerbatimInput[frame=single,
numbers=left,
numbersep=2pt,
firstline=1,
fontsize=\footnotesize,
showspaces=false]{\codingpath/stored_data.xml.out}
\caption{The output file with a  Boost XML archive produced by program
  \ref{program:bs:1}. The  internal composition  of the object  can be
  easily investigated  with this  human-friendly format which  is very
  useful for debugging  purpose. Note the XML format  use typical 5 to
  10 times more storage than the text format. It is also slower during
  I/O operations.  }
\label{sample:bs:1b}
\end{sample}

\clearpage

\subsection{Portability issues}

Portability  of serialized data  is not  easy to  achieve due  to many
sources   of   disagrement   between   many  software   and   hardware
architectures we may use.

Concerning    integral     types    (\texttt{bool},    \texttt{char},
\texttt{int},  \texttt{long}   and  their  respective  \emph{unsigned}
versions),  the problem  is  solved thanks  to  the use  of some  type
aliases  (typedefs) defined  in the  \texttt{boost/cstdint.hpp} header
and generalizing the standard  \texttt{stdint.h} header. It means that
the class members of whatever integral types to be serialized must use
such aliases as shown in the data sample class above.

For floating  values (float, double),  it is more  complex because
the storage  of values  in the computer  RAM depends on  the processor
architecture (little endian,  big endian, hybrid endianness\dots). More,
we must ensure that special  values like NaNs (not-a-number) and (plus
or  minus) infinity  (as  well as  denormalized  floating values)  are
supported. This is tricky.

The  Boost/Serialization  library implements  three  kinds of  archives
(input and output versions) :
\begin{itemize}

\item the text archives use an  ASCII format is not strictly portable
by default because it does not handle non-finite values for
 floating point numbers, however, for integer types, it is ok,
provided that the resources in \texttt{boost/cstdint.hpp} are used,

%% that can be made portable
%%   if some special care is taken :
%%   \begin{itemize}

%%   \item include  the \texttt{boost/cstdint.hpp}  header  and use  the
%%     \texttt{typedefs} for integer numbers that are declared in it,

%%   \item prepare  the associated I/O stream  (\texttt{imbue} method) in
%%     such a  way it support  the (de)serialization of NaN  and infinite
%%     floating point number values.

%%   \end{itemize}

\item the XML archives also uses an ASCII format and has the same
limitation as the text archives,

\item  the original  Boost  binary archives  is  portable for  integer
  numbers but does not support the (de)serialization of floating point
  numbers; we  do not recommend its  usage, which is  very limited and
  not adapted for scientific purpose.

\end{itemize}

Practically the text  and XML archives can be made  portable if we use
special features from the standard  iostream library in order to store
and  load  NaNs  and  infinite  values  correctly.   The  use  of  the
\texttt{boost/cstdint.hpp}  header  implies  portability  for  integer
numbers.  In  order to benefit  of the advantages of  binary formatted
archives (fastness and compactness  of storage), the portability issue
can  be  solved  using   some  contributed  library  from  Boost  (the
\emph{floating  point utilities}  by John  Rade) coupled  with special
implementation of a portable binary archive by Christian Pfligersdorffer.

 %%  This should be  extensively tested  to make  sure we
%% won't meet  problems. First encouraging  tests have been  performed to
%% check 32-bits versus 64-bits data exchange (using Linux). More work is
%% needed but things are on the right way.

The \texttt{datatools} library addresses these issues through some
special reader/writer classes that  are implemented in order 
to garantee portable serialization.

%%%%%%%%%%%%%%%%%%%%%%%%%%%%%%%%%%%%%

\section{Using serialization within  \texttt{datatools}}

\subsection{Reader and writer classes}

The  \texttt{datatools} implements  some generic  serialization reader
and writer classes :

\begin{itemize}

\item \texttt{datatools::serialization::data\_writer} : a general purpose serializer class (writer)  

\item \texttt{datatools::serialization::data\_reader} : a general purpose deserializer class (reader)  
  
\end{itemize}

\pn      These      classes      are      available      from      the
\texttt{datatools/serialization/io\_factory.h} header file.

\pn The  storage media consists in  a standard file  (I/O file stream)
which can be compressed (or not) using:

\begin{itemize}
\item no compression : no additional suffix,
\item GZIP compression : suffix is \TT{.gz},
\item BZIP2 compression : suffix is \TT{.bz2}.
\end{itemize}

\pn The  writer and reader classes support  serialization of different
kinds of archive:

\begin{itemize}  

\item    text archives (ASCII) from the Boost/Serialization library : file extension is \TT{.txt},

\item    XML archives (ASCII too) from the Boost/Serialization library : file extension is \TT{.xml},

\item    portable binary archives using the Boost/Serialization library interface
\footnote{a contribution by Christian Pfligersdorffer} : file extension is \TT{.data}. 

\end{itemize}

\pn  The filename's  suffix  determines automatically  the format  and
compression to be used. Examples:

\begin{itemize}
\item \texttt{my\_data.txt.gz} : corresponds to a GZIP compressed text
  archive,
\item  \texttt{my\_data.xml} :  corresponds  to a  not compressed  XML
  archive,
\item \texttt{my\_data.data.bz2}  : corresponds to  a BZIP2 compressed
  binary archive.
\end{itemize}


\pn Typical usage uses the simple interfaces from these classes:

\begin{itemize}    

\item the \texttt{data\_writer} class uses its template \texttt{store} 
  method to serialize a serializable object within a file,

\item the \texttt{data\_reader} class uses its template \texttt{load} 
  method to deserialize a serialized object from a file,
  
\end{itemize}

\pn In their basics implementation the \texttt{store} and \texttt{load} 
method requires :

\begin{itemize}

\item a reference to the object to be (de)serialized,

\item the object must fulfill the Boost serialization interface, i.e.
has a \texttt{serialize} template method,

\item a special character string, namely the \emph{record tag}, must be passed
to the store/load methods to associate the serialized object wih some 
unique identifier,

\end{itemize}

\subsection{Examples}

\pn The program \ref{program:bs:3} shows how to use the writer and reader classes
to serialize several object of some serializable class. The output file contents is shown in sample
\ref{sample:bs:3}.

\begin{program}[h]
\VerbatimInput[frame=single,
numbers=left,
numbersep=2pt,
firstline=1,
%lastline=53,
fontsize=\footnotesize,
showspaces=false]{\codingpath/data_3.cxx}
\caption{A program that serializes and deserializes an 
object of \texttt{data} class using XML archives.
}
\label{program:bs:3}
\end{program}

\begin{sample}[h]
\VerbatimInput[frame=single,
numbers=left,
numbersep=2pt,
firstline=1,
fontsize=\footnotesize,
showspaces=false]{\codingpath/data_3.txt.out}
\caption{The output file from a writer object produced by the program
  \ref{program:bs:3}. 
Here the writer has been configurated to \emph{encapsulate} each \texttt{data} object
within its own archive (see the \texttt{ds::using\_multi\_archives} flag at
 the writer's construction), thus we observe three serialization blocks. Note that each block
has an archive header (\texttt{2 serialization::archive 9}) 
immediately followed by the  \emph{record tag} (the \texttt{4 data} tokens serialize
the \TT{data} string) associated to the object which is streamed just after (\texttt{0 0 1 63\dots}).
}
\label{sample:bs:3}
\end{sample}
\clearpage

\subsection{Important remarks}

\subsubsection{Sequential access}

\pn  It  must be  noticed  that with  the  reader  and writer  classes
provided by \texttt{datatools}, serialized  objects can only be stored
in files sequentially and then reloaded  in the same way. This is thus
not  possible to implement  random access  to serialized  objects with
this   simple  technique.    Archive  files   generated   through  the
\texttt{datatools::serialization::data\_writer} may be compared with a
magnetic  tape.   Particularly,   if  the  serialization  flow  failed
somewhere  in the  middle of  the stream  media, the  stream  from the
failure point to its end is likely to be unusable.


\pn Note that  the reader class has some method to  check if some data
are   available  from   the  stream.    This  is   shown   in  program
\ref{program:bs:3X}.   The program serializes  two different  types of
objects in  some arbitrary order.  Then  we assume that  we don't know
the number of  serialized objects at load time and  also we don't know
their ordering in  the serialization stream.  The program  thus uses a
deserialization  loop   that  checks  the   stored  \emph{record  tag}
associated to the  next object from the archive  stream; this approach
allow to determine/guess which \emph{deserialization driver} has to be
used to load the next object from the stream.  Of course, one needs at
least to known in advance  the various types of the serialized objects
and provide the associated deserialization drivers.

\begin{program}[h]
\VerbatimInput[frame=single,
numbers=left,
numbersep=2pt,
firstline=1,
%lastline=53,
fontsize=\footnotesize,
showspaces=false]{\codingpath/data_3X.cxx}
\caption{Serialization and deserialization of two different  types of  
objects in some arbitrary order. 
}
\label{program:bs:3X}
\end{program}

%%%%%%%%%%%%%%%%
\subsubsection{Serialization strategy with \texttt{datatools} reader and writer}

\pn A \texttt{datatools} writer (and reader) object can be initialized with two different
strategies (\emph{modes}) :

\begin{itemize}

\item  One  single  archive  is  associated to  the  file  stream  and
  \emph{all}  serialized   objects  are  stored   within  this  unique
  archive. This leads to a  file with the following internal structure
  :
\begin{PathVerbatim}
<archive>
  <object #1 record tag>
  <object #1 data>
  <object #2 record tag>
  <object #2 data>
  <object #3 record tag>
  <object #4 data>
  ...
  <object #N record tag>
  <object #N data>
</archive>
\end{PathVerbatim}

\pn This behavior implies the following declaration syntax:
\begin{CppVerbatim}
namespace ds = datatools::serialization;
ds::data_writer the_writer ("data_3.txt", ds::using_single_archive);
ds::data_reader the_reader ("data_3.txt", ds::using_single_archive);
\end{CppVerbatim}

\pn Because of some internal functionnalities (\emph{memory tracking})
in  the Boost/Serialization  library, this  mode should  be  used with
care.

\item One archive is associated to  \emph{each} serialized
objects. This leads to a file with as many archives as serialized objects, thus 
the following internal
structure :
\begin{PathVerbatim}
<archive>
  <object #1 record tag>
  <object #1 data>
</archive>
<archive>
  <object #2 record tag>
  <object #2 data>
</archive>
<archive>
  <object #3 record tag>
  <object #4 data>
</archive>
  ...
<archive>
  <object #N record tag>
  <object #N data>
</archive>
\end{PathVerbatim}
\pn This behavior implies the following declaration syntax:
\begin{CppVerbatim}
namespace ds = datatools::serialization;
ds::data_writer the_writer ("data_3.txt", ds::using_multi_archives);
ds::data_reader the_reader ("data_3.txt", ds::using_multi_archives);
\end{CppVerbatim}

\pn The use of this mode is recommended, unless the user knows what he/she is doing.
\end{itemize}

\pn Note that if an archive file has been created with one of these modes,
it must be read with the same mode.

\clearpage
\section{The \texttt{datatools::serialization::i\_serializable} interface}

\pn We  have shown in the  previous sections that  developpers have to
provide the \texttt{serialize} template  method for each class that is
supposed  to be serialized  through Boost/Serialization.  However with
datatools serialization tools (reader and writer), a special interface
is provided  in order to equip  a given classes  with some dynamically
inherited   \emph{serializable}   trait.    On   this   purpose,   the
\texttt{datatools::serialization::i\_serializable}  abstract class has
been                implemented                (from               the
\texttt{datatools/serialization/i\_serializable.h} header file).

\subsection{The \emph{serializable} tag}

\pn    This    interface    class    simply   enforces    a    virtual
\texttt{get\_serial\_tag} method  to be  associated to the  class. The
signature of the method is :

\begin{CppVerbatim}
const std::string & get_serial_tag () const;
\end{CppVerbatim}

\pn The aim  of this method is to provide  an unique character string,
i.e. the \emph{serialization tag}, that can identify without ambiguity
the  type  of  an  object   stored  in  a  Boost  archive.  Using  the
\texttt{datatools} reader  and writer classes (see above),  it is thus
possible, for an object \texttt{obj1}  of which the class fulfills the
\texttt{i\_serializable} interface, to use the alternative methods :
\begin{CppVerbatim}
  ...
  the_writer.store (obj1); // The get_serial_tag () method is invoked 
                           // internally
  ...
  the_reader.load (obj1);  // The get_serial_tag () method is invoked 
                           // and checked internally
  ...
\end{CppVerbatim}

This   makes   life    easier   for   users   because   once   a
\emph{serialization tag} has been  fixed while designing a given class
and  implemented through  the \texttt{i\_serializable}  interface, one
does not have to remind it.

The   samples    \ref{sample:bs:4},    \ref{sample:bs:4b}   and
\ref{sample:bs:4c}  illustrates the case of a  \texttt{sdata} class  
that inherits
the \\
\texttt{datatools::serialization::i\_serializable} interface.
The program \ref{program:bs:4} shows how to use the writer and reader classes
to serialize several \texttt{sdata} objects. 

\begin{sample}[h]
\VerbatimInput[frame=single,
numbers=left,
numbersep=2pt,
firstline=4,
lastline=57,
fontsize=\footnotesize,
showspaces=false]{\codingpath/sdata.h}
\caption{The interface of a serializable \texttt{sdata} class.
}
\label{sample:bs:4}
\end{sample}

\begin{sample}[h]
\VerbatimInput[frame=single,
numbers=left,
numbersep=2pt,
firstline=1,
fontsize=\footnotesize,
showspaces=false]{\codingpath/sdata.cc}
\caption{The implementation of the \texttt{sdata} class.}
\label{sample:bs:4b}
\end{sample}

\begin{sample}[h]
\VerbatimInput[frame=single,
numbers=left,
numbersep=2pt,
firstline=2,
fontsize=\footnotesize,
showspaces=false]{\codingpath/sdata.ipp}
\caption{The definition of  the \texttt{serialize} template method for
  the serializable \texttt{sdata} class.  }
\label{sample:bs:4c}
\end{sample}

\begin{program}[h]
\VerbatimInput[frame=single,
numbers=left,
numbersep=2pt,
firstline=1,
fontsize=\footnotesize,
showspaces=false]{\codingpath/data_4.cxx}
\caption{A  program  that   serializes  and  deserializes  objects  of
  \texttt{data} class using \texttt{datatools} writer and reader.  }
\label{program:bs:4}
\end{program}
\clearpage

As  the syntax  needed to  implement such  functionnalities  is rather
complex, a  bunch of useful  pre-processor macros are  available.  The
samples  \ref{sample:bs:5}, \ref{sample:bs:5b}  and \ref{sample:bs:5c}
illustrates  the  use  of  these  macros to  define  the  serializable
\texttt{sdata} class.

\begin{sample}[h]
\VerbatimInput[frame=single,
numbers=left,
numbersep=2pt,
firstline=12,
lastline=31,
fontsize=\footnotesize,
showspaces=false]{\codingpath/sdata2.h}
\caption{The interface of a serializable \texttt{sdata} class.
}
\label{sample:bs:5}
\end{sample}

\begin{sample}[h]
\VerbatimInput[frame=single,
numbers=left,
numbersep=2pt,
firstline=3,
lastline=11,
fontsize=\footnotesize,
showspaces=false]{\codingpath/sdata2.cc}
\caption{The implementation of the \texttt{sdata} class (excerpt).}
\label{sample:bs:5b}
\end{sample}

\begin{sample}[h]
\VerbatimInput[frame=single,
numbers=left,
numbersep=2pt,
firstline=8,
fontsize=\footnotesize,
showspaces=false]{\codingpath/sdata2.ipp}
\caption{The definition of  the \texttt{serialize} template method for
  the serializable \texttt{sdata} class (this version uses a macro).  }
\label{sample:bs:5c}
\end{sample}

\clearpage

\section{Serialization through pointers to an abstract mother class}

The Boost/Serialization library  provides a mechanism to automatically
serialize objects  derived from a  polymorphic mother class.   In this
case serialization can be performed through pointers to the base class.

This  system  relies on  some  useful  macros  to automate  a  complex
mechanism to register  the mother/daughter class relationships between
classes and determine the actual derived type of a serialized object.

Practically,  the \texttt{datatools::utils::i\_serializable} interface
behaves like such a mother  class. Thus, each class that inherits from
it can be (de)serialized through this mechanism.

The    program    samples   \ref{sample:ps:6a},    \ref{sample:ps:6b},
\ref{sample:ps:6c}  and  \ref{sample:ps:6d}  illustrate  the  case  of
serializable  objects of  classes \texttt{sdata1}  and \texttt{sdata2}
derived from  \texttt{datatools::utils::i\_serializable}.  An array of
pointers  to  such \emph{serializable}  objects  is  filled with  such
randomly generated  objects.  The array  is then serialized  through a
\texttt{datatools} writer that uses a  text archive file.  In a second
step,   the  array   is   deserialized  from   the   file  through   a
\texttt{datatools} reader.

\begin{sample}[h]
\VerbatimInput[frame=single,
numbers=left,
numbersep=2pt,
firstline=1,
lastline=29,
fontsize=\footnotesize,
showspaces=false]{\codingpath/data_6.cxx}
\caption{The interface of classes  \texttt{sdata1}  and \texttt{sdata2}.  }
\label{sample:ps:6a}
\end{sample}

\begin{sample}[h]
\VerbatimInput[frame=single,
numbers=left,
numbersep=2pt,
firstline=31,
lastline=41,
fontsize=\footnotesize,
showspaces=false]{\codingpath/data_6.cxx}
\caption{The implementation code for classes  \texttt{sdata1}  and \texttt{sdata2}.  }
\label{sample:ps:6b}
\end{sample}


\begin{sample}[h]
\VerbatimInput[frame=single,
numbers=left,
numbersep=2pt,
firstline=43,
lastline=71,
fontsize=\footnotesize,
showspaces=false]{\codingpath/data_6.cxx}
\caption{Definition and automated instantiation of serialization and exporting code.  }
\label{sample:ps:6c}
\end{sample}

\begin{sample}[h]
\VerbatimInput[frame=single,
numbers=left,
numbersep=2pt,
firstline=74,
lastline=124,
fontsize=\footnotesize,
showspaces=false]{\codingpath/data_6.cxx}
\caption{Main program.  }
\label{sample:ps:6d}
\end{sample}

When  one creates some  new serializable  \texttt{foo} class  with this
technique, it  is strongly recommended to split  the interface (header
file   \texttt{foo.h}   as    in   sample   \ref{sample:ps:6a}),   the
implementation   (source    file   \texttt{foo.cc}   as    in   sample
\ref{sample:ps:6b})   and  the  code   specific  to   serialization  (
\texttt{foo.ipp} as in  sample \ref{sample:ps:6c}).  The good approach
is :
\begin{itemize}

\item to build an object file from the implementation file :
\begin{center}
\texttt{foo.h} +\texttt{foo.cc} $\rightarrow$  \texttt{foo.o}
\end{center}

\item  to  build  an   additional  object  file  from  the  associated
  serialization code only :
\begin{center}
\texttt{foo.h} + \texttt{foo.ipp} + INSTANTIATION/EXPORT macros $\rightarrow$  \texttt{foo\_bio.o}
\end{center}
\pn where the \TT{\_bio} suffix refers to the Boost I/O system.
\end{itemize}

\pn This allows:
\begin{itemize} 

\item to link easily some executable with both object files 
 \texttt{foo.o} and \texttt{foo\_bio.o},

\item but also, if we  don't need the serialization functionnalities, to
  link only with the \texttt{foo.o} object file,

\item  and finally,  to link  with \texttt{foo.o}  and  an alternative
  serialization  object file  (\texttt{alt\_foo\_bio.o}) that  must be
  provided in  place of the  default \texttt{foo\_bio.o} if  this last
  one  does not  suit the  serialization requirements  for  a specific
  application.
\end{itemize}

\clearpage

\section{Conclusion}

The     \texttt{datatools}     library    implements     serialization
functionnalities based on the Boost/Se\-rialization library. Most of the
code is handled through  templates and high-level preprocessing macros
are  provided to  ease  the implementation  of  serialization code  in
users' applications.

An  interface for  \emph{serializable} object  has been  designed. The
container   classes  provided   by  \texttt{datatools}   fulfill  this
interface. Particularly, the \texttt{datatools::utils::things} generic
container can store any of these \emph{serializable} objects.

In this note, some recommendations have been done in order to organize
the   implementation  of   the   code  :   splitting  the   interface,
implementation and  serialization layers.  This allows to  use a clear
and  standard procedure and  makes the  serialization functionnalities
optionnal, versatile and  extensible for all new classes  that enter a
project.   The  integration with  the  high-level  generic reader  and
writer classes provided by \texttt{datatools} is thus straightforward.

Three Boost  archive formats  are available from  \texttt{datatools} 
in both input and output versions :
ASCII text, XML  and binary. All are made portable  and can handle all
integral types  up the  64-bits as well  as floating point  numbers in
single  and  double precision  (IEEE754)  including non-finite  values
(NaNs, $\pm\infty$).


The Boost/Serialization library has many features and functionnalities
(class  versionning, memory  tracking,  load/save code  splitting\dots)
that  are out  of the  scope  of this  document and  thus not  covered
here.  The  reader  is   invited  to  read  the  online  do\-cumentation
at:\\ \texttt{http://www.boost.org/doc/libs/1\_48\_0/libs/serialization/doc/index.html}.


\end{document}
%%%%%%%%%%%%%%%%%%%%%%%%%%%%%%%%%%%%%%%%%%%%%%%%%%%%%%%%%%%%%%%%

%% end of DatatoolsSerializationTutorial.tex
