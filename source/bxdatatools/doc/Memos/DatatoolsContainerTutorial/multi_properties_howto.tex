%% multi_properties_howto.tex

\section{The \texttt{datatools::utils::multi\_properties} class}\label{sec:multi_properties}

\subsection{Introduction}

We    have    seen    in     the    previous    section    that    the
\texttt{datatools::utils::properties}  class  can  be  used  to  store
simple arbitrary parameters of simple types. It is particularly useful
to represent  and manipulate some list of  configuration parameters to
setup an application or the behaviour of an algorithm.

\pn However, in a complex  software environment, there are many coexisting
components that may need such configuration interface. Using an unique
properties container to manage all  the settings for all components of
a large  scale application  is not very  practical or  desirable.  For
example, several  individual components may  implement a configuration
interface based on a properties container and associated configuration
ASCII  files.   Integrating  such  components  in  a  larger  software
framework  should  not  enforce  developpers  to  rewrite  the  global
configuration interface  and the  users to change  their configuration
file formatting.  A better approach  is likely to reuse the individual
configuration interfaces  from each component,  eventually adding some
extension to  it to  support a higher  level of functionnality  at the
scale of the full application.

\pn The \texttt{datatools::utils::multi\_properties}  class is designed to
address  this use  case, among  others.  It  is a  meta-container that
contains  several  sections,  each  section  being  implemented  as  a
\texttt{datatools::utils::properties}   container.    Practically,   a
multi-properties  container is  a  dictionnary of  which each  element
(properties container) are stored  with an unique mandatory access key
(or  name).  The  naming policy  is  the same  than the  one used  for
property records in a properties container.


\subsection{Header file and instantiation}

\pn  In   order  to  use  \texttt{datatools::utils::multi\_properties}
objects, one must use the following include directive:
\begin{CppVerbatim} 
#include <datatools/utils/multi_properties.h>
\end{CppVerbatim}

\pn The  declaration of a \texttt{datatools::utils::multi\_properties}
instance can be simply done with:
\begin{CppVerbatim} 
datatools::utils::multi_properties my_multi_container;
\end{CppVerbatim}
\pn Note  the use  of the nested  \texttt{datatools::utils} namespace.
Alternatively one can use :
\begin{CppVerbatim} 
using namespace datatools::utils;
multi_properties my_multi_container;
\end{CppVerbatim}
\pn or:
\begin{CppVerbatim} 
namespace du = datatools::utils;
du::multi_properties my_multi_container;
\end{CppVerbatim}

\subsection{Design}

\subsubsection{Names and metas}

\pn As  mentionned above, a multi-properties container  behaves like a
dictionary  of  properties  containers.   A  given  stored  properties
container is addressed through its unique key (or \emph{name}).  It is
also  optionally associated to  a \emph{meta}  data string  that gives
some additional information about the properties container stored in a
given section of a multi-properties container.

\subsubsection{Basics}

\pn When instantiated, a multi-properties is given:
\begin{itemize}

\item          an         optional          description         string
  (methods:\texttt{set\_description}, \texttt{get\_description}),

\item a  mandatory string  to label the  \emph{key} in  human readable
  format  (default  is  \TT{name}, methods:  \texttt{set\_key\_label},
  \texttt{get\_key\_label} ),

\item  a  optional string  to  label  the  \emph{meta} data  in  human
  readable     format      (default     is     \TT{type},     methods:
  \texttt{set\_meta\_label},   \texttt{get\_meta\_label}).  This  meat
  label can be forced to the empty string.  In this case, no meta data
  is used by the container.

\end{itemize}

\pn The program source sample \ref{program:multi_properties:0} shows how to 
declare and initialize a multi-properties object.
\begin{program}[h]
\VerbatimInput[frame=single,
numbers=left,
numbersep=2pt,
firstline=1,
fontsize=\footnotesize,
showspaces=false]{\codingpath/multi_properties_0.cxx}
\caption{Declare and setup a
  \texttt{datatools::utils::multi\_properties}   object.    
}
\label{program:multi_properties:0}
\end{program}

\subsubsection{Container interface}

\pn The container interface of a multi-properties object enables to :
\begin{itemize}

\item  \texttt{has\_key}, \texttt{has\_section} :  check if  a section
  with a given key exists,

\item \texttt{get}  : provide access to  a section entry  with a given
  name,

\item \texttt{add}  : add a new  section with given  name and optional
  meta strings,

\item  \texttt{remove} :  remove an  existing new  section  with given
  name,

\item \texttt{size} : return the number of stored properties,

\item \texttt{clear} : remove all stored properties container items.

\end{itemize}

\pn The  program source sample  \ref{program:multi_properties:1} shows
how to add some empty sections in a multi-properties object. It prints
the  contents of  the container  in a  human readable  formaty (sample
\ref{sample:multi_properties:1}.

\begin{program}[h]
\VerbatimInput[frame=single,
numbers=left,
numbersep=2pt,
firstline=1,
fontsize=\footnotesize,
showspaces=false]{\codingpath/multi_properties_1.cxx}
\caption{Adding section in  a
  \texttt{datatools::utils::multi\_properties}   object.    
}
\label{program:multi_properties:1}
\end{program}


\begin{sample}[h]
\VerbatimInput[frame=single,
numbers=left,
numbersep=2pt,
firstline=1,
%lastline=3,
fontsize=\footnotesize,
showspaces=false]{\codingpath/multi_properties_1.out}
\caption{The  output of the  program \ref{program:multi_properties:1}.
  We  can  check that  the  three  sections  have been  stored.   They
  correspond to still empty  properties containers because no property
  have been stored so far.}
\label{sample:multi_properties:1}
\end{sample}

\pn It is  also possible to store a copy  of an independant properties
container  object  within a  multi-properties  container  as shown  in
sample   \ref{sample:multi_properties:2}.   This   technique  can   be
inefficient if the source properties container hosts a large number of
properties.


\begin{sample}[h]
\VerbatimInput[frame=single,
numbers=left,
numbersep=2pt,
firstline=18,
lastline=21,
fontsize=\footnotesize,
showspaces=false]{\codingpath/multi_properties_2.cxx}
\caption{Copy a standalone properties container object
within a multi-properties container.}
\label{sample:multi_properties:2}
\end{sample}

\pn  Accessing individual  properties containers  is  their respective
section          is          illustrated         with          program
sample        \label{sample:multi_properties:3}.        Here       the
\texttt{has\_key("<KEY>")} and chained \texttt{get("<KEY>").get\_properties()}
 methods are the basic tools.

\begin{sample}[h]
\VerbatimInput[frame=single,
numbers=left,
numbersep=2pt,
firstline=13,
lastline=31,
fontsize=\footnotesize,
showspaces=false]{\codingpath/multi_properties_3.cxx}
\caption{Manipulating  a  properties container  object  stored in  the
  section of a \texttt{datatools::utils::multi\_properties} object.  }
\label{sample:multi_properties:3}
\end{sample}

\pn A more efficient technique consists in the manipulation of the
stored properties container through a reference (mutable or const)
as shown on sample  \label{sample:multi_properties:4}.

\begin{sample}[h]
\VerbatimInput[frame=single,
numbers=left,
numbersep=2pt,
firstline=33,
lastline=48,
fontsize=\footnotesize,
showspaces=false]{\codingpath/multi_properties_3.cxx}
\caption{Manipulating  a  properties container  object  stored in  the
  section of a \texttt{datatools::utils::multi\_properties} object through
mutable and const references.  }
\label{sample:multi_properties:4}
\end{sample}

\clearpage
\subsubsection{Serialization and I/O features}

Like    the     \texttt{datatools::utils::properties}    class,    the
\texttt{datatools::utils::multi\_properties}     class     has     I/O
functionnalities :

\begin{itemize}

\item ASCII formatting in standard I/O streams.

\item High level  serialization based on the Boost/Serialization library
(not considered here).

\end{itemize} 


\pn The program source  \ref{program:multi_properties:5} shows how to use
this class to save a multi-properties object 
in an ASCII formated file (method: \texttt{write}) and to reload it
(method: \texttt{read}) . 
The contents of the saved file
is shown on sample \ref{sample:multi_properties:5}.


\begin{program}[h]
\VerbatimInput[frame=single,
numbers=left,
numbersep=2pt,
firstline=1,
%%lastline=88,
fontsize=\footnotesize,
showspaces=false]{\codingpath/multi_properties_4.cxx}
\caption{Save        and       load        functionnalities       with
  \texttt{datatools::utils::multi\_properties} objects.  Note that the
  optional description  of the multi-properties  container is recorded
  through a  meta-comment (line starting  with the \TT{\#@description}
  prefix).  The  mandatory key label  and meta label strings  are also
  printed using the \texttt{\#@key\_label} and \texttt{\#@meta\_label}
  meta-comments.  These three informations  must be printed before any
  sections.   Each  section  has   a  square  bracketed  header  which
  indicates the  key (\texttt{name="XXX"})  and the meta  string (with
  \texttt{type="YYY"}). The  internal of a section  conforms the ASCII
  formatting syntax for \texttt{datatools::utils::properties} objects.
}
\label{program:multi_properties:5}
\end{program}


\begin{sample}[h]
\VerbatimInput[frame=single,
numbers=left,
numbersep=2pt,
firstline=1,
%lastline=31,
fontsize=\footnotesize,
showspaces=false]{\codingpath/multi_properties_4.conf}
\caption{The ASCII file generated by program  \ref{program:multi_properties:5}.  }
\label{sample:multi_properties:5}
\end{sample}

\subsubsection{Advanced features}


\pn Having a look on the program output \ref{sample:multi_properties:1}, one
should notice that some informations about the order in which the different
sections have been stored is recorded inside the multi-properties container.

\pn  Such a feature  enables not  only to  benefit of  the dictionnary
interface   of  the   \\  \texttt{datatools::utils::multi\_properties}
class, but also to access the  items with respect to the order used to
store  them.  The  \texttt{ordered\_entries()}  and \texttt{entries()}
methods in  the class implements some non  mutable access respectively
to  the   ordered  collection  of   sections  (at  insertion   in  the
multi-properties container) and the non-ordered collection of sections
(using the  default order  provided by the  standard \texttt{std::map}
class).


%\subsection{Examples}


%% end of multi_properties_howto.tex
