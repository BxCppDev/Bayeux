%% things_howto_examples.tex

\subsection{Examples}

\subsubsection{Declare a \emph{things} container and add objects in it}

\pn  The sample program \ref{program:things:0}  shows how  to
declare a \emph{things} container object and add two \emph{properties}
objects in  it. It  prints the  contents of the  container in  a human
readable format (sample \ref{sample:things:0}.

\begin{program}[h]
\VerbatimInput[frame=single,
numbers=left,
numbersep=2pt,
firstline=1,
fontsize=\footnotesize,
showspaces=false]{\codingpath/things_0.cxx}
\caption{Adding     objects     in    a     \texttt{datatools::things}
  container. Note  that all  low-level memory allocation  operation is
  performed internally. The user does not have to care about it.  }
\label{program:things:0}
\end{program}



\begin{sample}[h]
\VerbatimInput[frame=single,
numbers=left,
numbersep=2pt,
firstline=1,
%lastline=3,
fontsize=\footnotesize,
showspaces=false]{\codingpath/things_0.out}
\caption{The  output of  the program  \ref{program:things:0}.   We can
  check  that  the  two banks  of  data  have  been stored.  Both  are
  \texttt{datatools::properties} objects, which are empty here.  }
\label{sample:things:0}
\end{sample}


\subsubsection{Instant manipulation of added objects through references}

\pn  The sample program  \ref{sample:things:1}  shows how  to
declare  a \emph{things} container  object, add  two \emph{properties}
objects in  it and make use  of the mutable reference  returned by the
template \texttt{add} methods.

\begin{sample}[h]
\VerbatimInput[frame=single,
numbers=left,
numbersep=2pt,
firstline=14,
lastline=24,
fontsize=\footnotesize,
showspaces=false]{\codingpath/things_1.cxx}
\caption{We use the mutable  references that are returned while adding
  objects  in  a  \texttt{datatools::things}  container  in  order  to
  manipulate the stored objects.  }
\label{sample:things:1}
\end{sample}


\subsubsection{Get references to manipulate stored objects}

\pn  The  program source  sample  \ref{sample:things:2}  shows how  to
obtain non-mutable  and mutable  references to an  object stored  in a
\emph{things} container object with a given name.  The \texttt{get<T>}
and \texttt{grab<T>} template  methods respectively return non-mutable
and   mutable  references  (here   class  \texttt{T}   corresponds  to
\texttt{datatools::properties}).   The  references  are then  used  to
manipulate the stored object through its own interface.

\begin{sample}[h]
\VerbatimInput[frame=single,
numbers=left,
numbersep=2pt,
firstline=26,
lastline=42,
fontsize=\footnotesize,
showspaces=false]{\codingpath/things_1.cxx}
\caption{We explicitely  initialize references to an  object stored in
  the \texttt{datatools::things} container  in order to manipulate the
  stored object.  }
\label{sample:things:2}
\end{sample}

\subsubsection{Create a new class to be stored in a \emph{things} container}

\pn     The     sample     programs    \ref{program:things:2a}     and
\ref{program:things:2b}  show how to  declare a new  serializable class
\texttt{storable\_type}  that   can  be  stored   in  a  \emph{things}
container object (output is shown on sample \ref{sample:things:2}).

\begin{program}[h]
\VerbatimInput[frame=single,
numbers=left,
numbersep=2pt,
firstline=1,
lastline=50,
fontsize=\footnotesize,
showspaces=false]{\codingpath/things_2.cxx}
\caption{Creation of a new storable class for
  \texttt{datatools::things} container.
}
\label{program:things:2a}
\end{program}

\begin{program}[h]
\VerbatimInput[frame=single,
numbers=left,
numbersep=2pt,
firstline=52,
fontsize=\footnotesize,
showspaces=false]{\codingpath/things_2.cxx}
\caption{Program  to   store  a   new  storable  class   (see  program
  \ref{program:things:2a})   in   a  \texttt{datatools::things}
  container.  }
\label{program:things:2b}
\end{program}


\begin{sample}[h]
\VerbatimInput[frame=single,
numbers=left,
numbersep=2pt,
firstline=1,
%lastline=3,
fontsize=\footnotesize,
showspaces=false]{\codingpath/things_2.out}
\caption{The          output          of          the          program
  \ref{program:things:2a}-\ref{program:things:2b}.  }
\label{sample:things:2}
\end{sample}

%% end of things_howto_examples.tex
