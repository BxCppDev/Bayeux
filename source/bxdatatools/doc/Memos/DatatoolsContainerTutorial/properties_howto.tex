%% properties_howto.tex

\section{The \texttt{datatools::properties} class}\label{sec:properties}

\subsection{Introduction}

The \texttt{datatools::properties}  class is designed to  be an useful
container of arbitrary  \emph{properties}.  In the \texttt{datatools}'
terminology, a  \emph{property} is a  variable of a simple  type which
has :
\begin{itemize}
\item  a \emph{name}  (or \emph{key})  which can  be used  to uniquely
  identify the property and thus access to its value,
\item a \emph{value} of a  certain \emph{type} which is supposed to be
  used by an user or a client application.
\end{itemize}

As  an example,  let's build  the virtual  representation of  a simple
geometry  3D-object, namely  a cube,  using \emph{properties}.  We may
choose the following list of \emph{properties} :

\begin{center}
\begin{tabular}{|c|c|c|}
\hline
Name    &    Value  & Type \\
\hline
\hline
\TT{dimension} & 28.7 cm   & real  \\
\hline
\TT{colour}    & \TT{blue} & string  \\
\hline
\TT{material}  & \TT{copper} & string  \\
\hline
\TT{price}     & 2.36 \euro & real \\
\hline
\TT{available} & true       & boolean \\
\hline
\TT{nb\_in\_stock} & 65     & integer \\
\hline
\TT{manufacturer}  & \TT{The ACME International Company}  & string \\
\hline
\TT{reference\_number}  & \TT{234/12.456}  & string \\
\hline
\end{tabular}
\end{center}

As it can be seen in the example above, this list of \emph{properties}
makes possible to figure  out non-ambiguously the main characteristics
of  an  object.   We   should  have  writen,  its  main  \emph{static}
characteristics, because the table above does not contain informations
about the  \emph{behavior} and \emph{dynamic}  characteristics of the
cube.


\subsection{Header file and instantiation}

\pn In order to use \texttt{datatools::properties} objects, one
must use the following include directive:
\begin{CppVerbatim}
#include <datatools/properties.h>
\end{CppVerbatim}

\pn The declaration of a  \texttt{datatools::properties} instance
can be simply done with:
\begin{CppVerbatim}
datatools::properties my_properties_container;
\end{CppVerbatim}
Alternatively one can use :
\begin{CppVerbatim}
using namespace datatools;
properties my_properties_container;
\end{CppVerbatim}

\subsection{Design}

\subsubsection{Naming scheme}

The \texttt{datatools::properties} class is implemented through
a \texttt{std::map}  container as a  dictionnary (or lookup  table) of
\emph{records}.  A  record   contains  all  informations  to  uniquely
identify  a  given  property :  name (key),  type,  value  and a  few  other
attributes. On the  point of view of the  user, implementation details
are not important.

First  of  all,   when  one  wants  to  store   a  property  within  a
\texttt{datatools::properties} container object, we must choose
it  a \emph{name}  (or \emph{key}).  This name  must be  unique  in this
container, so it must not be already used by another property record.

The name must fulfill some rules:
\begin{itemize}

\item it must not be the empty string (\verb+""+),

\item it must contains  only digits (0\dots9), alphabetical characters
  (\verb+'a'+  to \verb+'z'+  and  \verb+'A'+ to  \verb+'Z'+) and  the
  underscore (\verb+'_'+) or dot (\verb+'.'+) characters,

\item it must not start with a digit,

\item it must not end with a dot.

\end{itemize}

\pn Table \ref{tab::properties:keys:0}  shows some valid and invalid
choices   for   names   to   be   used   as   property   keys   in   a
\texttt{datatools::properties} object.

\begin{table}[h]
\begin{center}
\begin{tabular}{|c|c|}
\hline
Key      &   Validity \\
\hline
\hline
\TT{a}                 &     yes  \\
\hline
\TT{debug\_level}      &     yes  \\
\hline
\TT{\_hello\_}         &     yes  \\
\hline
\TT{FirstName}         &     yes  \\
\hline
\TT{\_\_status}        &     yes  \\
\hline
\TT{logging.filename}  &     yes  \\
\hline
\TT{b.}                &     no  \\
\hline
\TT{\$\{DOC\}}         &     no  \\
\hline
\TT{007}               &     no  \\
\hline
\end{tabular}
\end{center}
\caption{Example of names (keys) supported or not supported
by the \texttt{datatools::properties} class.}
\label{tab::properties:keys:0}
\end{table}

%% This default  behavior can be  changed by attaching  another \emph{key
%%   validator} object to the  container. However, this functionnality is
%% out of the scope of this tutorial.

\subsubsection{Supported types}

Now we have chosen the name  (key) for a property, we must indicate the
  type  of its value.  Only   four   types  are   supported  by   the
\texttt{datatools::properties}  class, both in  a \emph{scalar}
version (one  single value) or in a  \emph{array} version (implemented
as  a  \texttt{std::vector}).  These  types  are given  in  table
\ref{tab::properties:types:0}.

\begin{table}[h]
\begin{center}
\begin{tabular}{|c|c|l|}
\hline
Label      &   Implementation & Range \\
\hline
\hline
boolean (a.k.a. \emph{flag}) & \texttt{bool} & false or true (0 or 1)  \\
\hline
integer    & 32 bits signed integers & from $-2^{31}$ to $+2^{31}-1$ \\
\hline
real       & \texttt{double}  & IEEE-754 64 bit encoded double \\
           &                  & precision floating point numbers \\
\hline
string     &  \texttt{std::string} & any character string not \\
           &                       & containing the double quote \\
           &                       & character(\verb+'"'+) \\
\hline
\end{tabular}
\end{center}
\caption{Property         types        supported         by        the
  \texttt{datatools::properties} class.}
\label{tab::properties:types:0}
\end{table}



\subsubsection{Additional property's traits}

When one stores a new property in a \emph{properties} container, it is
possible to provide an arbitrary string that describes it : its \emph{description}.
It is a human-friendly character string, stored in one single line.

More it  is also possible to  mark  a property as  \emph{non-mutable}. It means
that it will not be possible to further modify its value. However, the
property can be erased.

It is also possible to use \emph{private} properties.  Conventionnaly,
private  properties  have  a  name/key starting  with  two  underscores
(example: \TT{\_\_secret)}. The \emph{private} trait can trigger different
I/O behaviors (see below).

\subsubsection{Vector properties}

As mentionned above, a property can store a single (scalar) typed value
as well as an array of values of the same type. The size of such an array
can be zero or whatever number of elements. It is only limited
by the available memory and possible limitations of the implementation
(here we use \texttt{std::vector} templatized containers).

\subsubsection{Meta informations}

Aside some internal data structures used to implement the dictionnary
of property records, a \texttt{datatools::properties} object
has a \emph{debug} flag which is not supposed to be activated in some production
program and a \emph{description} string which can be used to store
some  arbitrary text formated meta-information.


\subsubsection{Constructors}

The \texttt{datatools::properties} class provides a few useful
constructors.

The default  constructor initializes an empty  container of properties
and   fulfills   most   needs   in   real world   use   cases.   Sample
\ref{sample:properties:0}   shows  the  syntax   used  to   declare  a
properties container object.

\begin{sample}[h]
\VerbatimInput[frame=single,
numbers=left,
numbersep=2pt,
firstline=1,
%%lastline=9,
fontsize=\footnotesize,
showspaces=false]{\codingpath/properties_0.cxx}
\caption{The declaration of a default empty \emph{properties}
object and a documented empty \emph{properties}
object with its embedded description.}
\label{sample:properties:0}
\end{sample}


\subsubsection{Interface methods}

There     are     many     methods     available     to     manipulate
\texttt{datatools::properties} container objects.

\pn Non mutable methods :

\begin{itemize}

\item  \texttt{get\_description}   returns  the  embedded  description
  character string (if any),

\item \texttt{keys} builds a list of the keys of all objects stored in
  the container,

\item  \texttt{size} returns the  number of  properties stored  in the
  container,

\item \texttt{has\_key} informs if  the container contains a property
  with a given key/name,

\item  \texttt{is\_boolean},  \texttt{is\_integer}, \texttt{is\_real},
  \texttt{is\_string} inform  if a stored  property is of  some given
  type,

\item \texttt{is\_scalar}  informs if a  stored property is  scalar (a
  single value is stored) or not,

\item \texttt{is\_vector}  informs if a stored property  is vector (an
  array of values of the same type is stored) or not,

\item \texttt{has\_flag}  informs if a stored  boolean property exists
  and has the \TT{true} value,

\item \texttt{fetch\_boolean} returns the value of an existing boolean
  property (flag) stored in the container.

\item \texttt{fetch\_integer} returns the value of an existing integer
  property stored in the container.

\item  \texttt{fetch\_real}  returns the  value  of  an existing  real
  property stored in the container.

\item \texttt{fetch\_string}  returns the value of  an existing string
  property stored in the container.

\item  various  \texttt{fetch}  and  \texttt{fetch\_XXX\_YYY}  methods
  return  the  value  of  an  existing property  of  some  given  type
  \texttt{XXX}  (\texttt{boolean}, \texttt{integer},  \texttt{real} or
  \texttt{string})  and   given  traits  \texttt{YYY}  (\texttt{scalar},
  \texttt{vector})  stored in  the  container.  In  case  of a  vector
  property, the index of the value stored in the internal array can be passed.

\item \texttt{tree\_dump}  prints the content  of the container  in an
  ASCII readable format.

\end{itemize}

\pn A large set of mutable methods are available :

\begin{itemize}

\item \texttt{store\_flag}  and \texttt{set\_flag} adds  a new boolean
  property in the container with a given key (the key must not already
  exist), the value is set at \TT{true} by default

\item \texttt{store\_boolean} adds a new boolean property in the container
with a given key (the key must not already exist), the value is set on user
choice,

\item \texttt{store\_integer} adds a new integer property in the container
with a given key (the key must not already exist), the value is set on user
choice,

\item \texttt{store\_real} adds a new real property in the container
with a given key (the key must not already exist), the value is set on user
choice,

\item \texttt{store\_string} adds a new string property in the container
with a given key (the key must not already exist), the value is set on user
choice,

\item various overloaded \texttt{store}  methods add a new property of
  a given type (scalar, vector) in the container with a given key (the
  key must not already exist), the value is set at user choice,

\item  various \texttt{change\_XXX}  methods  modify the  value of  an
  existing   property    of   type   \texttt{XXX}   (\texttt{boolean},
  \texttt{integer},  \texttt{real} or \texttt{string}).  In case  of a
  vector  property,  the index  of  the value  to  be  changed in  the
  internal array can be passed.

\item various  overloaded \texttt{change} methods modify  the value of
  an existing  property of a given  type and size  (scalar, vector) in
  the container.  The value is set at user choice. In case of a vector
  property, the index of the value to be changed in the internal array
  can be passed.

\item \texttt{update\_flag} adds a new boolean true value using a given
key or update it at its true value is it already exists.

\item \texttt{update\_XXX}  methods update the  value or add it  if it
  does not exist yet. The value is set at user choice.

\item various overloaded \texttt{update} methods behave like
the one described above.

\item \texttt{erase}  erase the property with a given key

\item \texttt{erase\_all}  erase all stored properties,

\item \texttt{erase\_all\_starting\_with}  erase all stored properties
with a key that starts with a given prefix,

\item \texttt{erase\_all\_not\_starting\_with}  erase all stored properties
with a key that does not start with a given prefix,
,

\item \texttt{erase\_all\_ending\_with}  erase all stored properties
with a key that ends with a given prefix,

\item \texttt{erase\_all\_not\_ending\_with}  erase all stored properties
with a key that does not end with a given prefix,

\item \texttt{lock}  locks the container  for all mutable  methods but
  the \texttt{unlock} method,

\item  \texttt{unlock} unlocks the container if it was locked
  previously.

\item  \texttt{reset} and \texttt{clear} completely blank the contents
of the container (its description and its dictionnary of properties).

\end{itemize}

%\clearpage

\subsubsection{Exporting features}

The \texttt{datatools::properties} class  is copyable. It means
that it is possible to use the assignment operator (\verb+operator=+)
to assign the full content from one container property object to another one. A
copy   constructor   is    also   available.    The   program   source
\ref{program:properties:2} shows how two containers can be initialized
by copy  of an  original one.  The output printed  is shown  on sample
\ref{sample:properties:2}.


\begin{program}[h]
\VerbatimInput[frame=single,
numbers=left,
numbersep=2pt,
firstline=1,
%%lastline=88,
fontsize=\footnotesize,
showspaces=false]{\codingpath/properties_2.cxx}
\caption{A program that uses the assignment operator and copy constructor
for  \texttt{datatools::properties} objects.}
\label{program:properties:2}
\end{program}

\begin{sample}[h]
\VerbatimInput[frame=single,
numbers=left,
numbersep=2pt,
firstline=1,
%lastline=3,
fontsize=\footnotesize,
showspaces=false]{\codingpath/properties_2.out}
\caption{The output of the program \ref{program:properties:2}.}
\label{sample:properties:2}
\end{sample}

These plain copying functionalities are obviously extremely useful and
are implemented  by default in  the library. However, it  is sometimes
desirable  to export  only a  subset  of properties  from an  original
properties  container and \emph{install}  these subset  inside another
properties container : this  is called \emph{properties export}. It
realizes some kind of partial  copy of some original container (source)
to another one (target).

Several  export methods  are  implemented,  all need  to  be passed  a
mutable reference to  the target container and  some string that is  used as a
prefix for properties' keys:

\begin{itemize}

\item   \texttt{export\_starting\_with}   installs,   in   the   target
  container, a copy  of each property, of which the  key starts with a
  given prefix,

\item  \texttt{export\_not\_starting\_with}  installs,  in  the  target
  container, a copy of each property,  of which the key does not start
  with a given prefix,

\item  \texttt{export\_and\_rename\_starting\_with}  installs,  in  the
  target container, a  copy of each property, of  which the key starts
  with  a given  prefix;  copied  properties are  renamed  with a  new
  prefix,

\item \texttt{export\_if} (template) installs, in the target container,
  a  copy  of  each  property,  of  which the  key  fulfills  a  given
  predicate,

\item  \texttt{export\_not\_if}  (template)  installs,  in  the  target
  container,  a copy  of  each property,  of  which the  key does  not
  fulfill a given predicate.

\end{itemize}

The program source  \ref{program:properties:3} shows how properties
containers can be initialized by exporting original properties from
an source proerties container.
The  output   printed  is  shown   on  sample
\ref{sample:properties:3}.


\begin{program}[h]
\VerbatimInput[frame=single,
numbers=left,
numbersep=2pt,
firstline=1,
%%lastline=88,
fontsize=\footnotesize,
showspaces=false]{\codingpath/properties_3.cxx}
\caption{A  program  that  uses  \texttt{datatools::properties}
  class' exporting methods.}
\label{program:properties:3}
\end{program}

\begin{sample}[h]
\VerbatimInput[frame=single,
numbers=left,
numbersep=2pt,
firstline=1,
%lastline=3,
fontsize=\footnotesize,
showspaces=false]{\codingpath/properties_3.out}
\caption{The output of the program \ref{program:properties:3}.}
\label{sample:properties:3}
\end{sample}

%\clearpage

\subsubsection{Serialization and I/O features}

The \texttt{datatools::properties}  container class is equipped
with some I/O  functionnalities. There are two techniques  that can be
used to \emph{serialize} a properties container object :

\begin{itemize}

\item ASCII  formatting in standard I/O streams:  this technique makes
  possible  to store  and  load properties  objects  from simple  text
  files.  It uses an ASCII human readable format. The full dictionnary
  of  properties  is  thus  recorder  in an  non-ambiguous  way.  Such
  approach is extremely useful and preferred to implement configuration files.

\item High level  serialization using the Boost/Serialization library.
  Some  dedicated  template methods  are  implemented  to fulfill  the
  Boost/Serialization API. Boost text, XML and binary I/O archives are
  supported.  The  \texttt{datatools::properties} class  itself
  fulfills some  special serialization  interface that makes  it usable
  with    some     advanced    serialization    mechanism\footnote{The
    \texttt{datatools::i\_serializable} interface           class}.
  The description  of this feature is out  of the scope
  of this document.

\end{itemize}

\pn  Let's  concentrate on  the  ASCII  I/O  functionnalities: \\  the
\texttt{datatools::properties::config} class  is responsible to
write/read  to/from ASCII formated  standard streams.   Utility static
wrapper  methods are  provided  to perform  such  I/O operations  with
files.  By default  only  \emph{non-private} properties  are saved  or
loaded.

The program source  \ref{program:properties:4} shows how to use
this class to dump the ASCII format on the terminal or
a file. The  output   printed  is  shown   on  sample
\ref{sample:properties:4}, the contents of the saved file
is shown on sample \ref{sample:properties:4}.


\begin{program}[h]
\VerbatimInput[frame=single,
numbers=left,
numbersep=2pt,
firstline=1,
%%lastline=88,
fontsize=\footnotesize,
showspaces=false]{\codingpath/properties_4.cxx}
\caption{Save        and       load        functionnalities       with
  \texttt{datatools::properties}   objects.    Note  that   the
  description   of  the   container  itself   is  printed   through  a
  meta-comment (line starting  with the \TT{\#@config} prefix).  Also,
  some  properties  have  been  originaly stored  with  an  associated
  transcient description  string. This  information is also  saved for
  convenience  in the  ASCII file  using special  meta-comments (lines
  starting with \TT{\#@description} prefix).}
\label{program:properties:4}
\end{program}


\begin{sample}[h]
\VerbatimInput[frame=single,
numbers=left,
numbersep=2pt,
firstline=1,
%lastline=3,
fontsize=\footnotesize,
showspaces=false]{\codingpath/properties_4.out}
\caption{The output of the program \ref{program:properties:4}.}
\label{sample:properties:4}
\end{sample}

\begin{sample}[h]
\VerbatimInput[frame=single,
numbers=left,
numbersep=2pt,
firstline=1,
%lastline=3,
fontsize=\footnotesize,
showspaces=false]{\codingpath/properties_4.conf.save}
\caption{The   contents  of   the  file   generated  by   the  program
  \ref{program:properties:4}.  Note the  string property \TT{name} has
  no description.}
\label{sample:properties:4}
\end{sample}
%\clearpage

Of course,  it is  perfectly possible to  create a  configuration file
that  respects this  format using  your  favorite text  editor or  some
external  script.  It  is  then possible  to  ask a  given program  to
initialize a properties container  using this input file.  The program
source  \ref{program:properties:5}  shows   such  an  example  with  a
hand-edited  configuration  file  (sample  \ref{sample:properties:5}).
The    output     of    the    program    is     shown    in    sample
\ref{sample:properties:5out}.

\begin{sample}[h]
\VerbatimInput[frame=single,
numbers=left,
numbersep=2pt,
firstline=1,
fontsize=\footnotesize,
showspaces=false]{\codingpath/properties_5.conf}
\caption{A configuration file created \emph{by hand} to be used by the
  program  \ref{program:properties:5}.   Note   that  if  a  backslash,
  immediately followed by  a newline character, is added
  at the end  of a given line,  the parser assumes the line continues
  on the next line in the file.}
\label{sample:properties:5}
\end{sample}

\begin{program}[h]
\VerbatimInput[frame=single,
numbers=left,
numbersep=2pt,
firstline=1,
%%lastline=88,
fontsize=\footnotesize,
showspaces=false]{\codingpath/properties_5.cxx}
\caption{Initialization   of  a  \texttt{datatools::properties}
  container  through  the  reading  of the  ASCII  configuration  file
  (sample \ref{sample:properties:5})}.
\label{program:properties:5}
\end{program}

\begin{sample}[h]
\VerbatimInput[frame=single,
numbers=left,
numbersep=2pt,
firstline=1,
%lastline=3,
fontsize=\footnotesize,
showspaces=false]{\codingpath/properties_5.out}
\caption{The output of the program \ref{program:properties:5}.}
\label{sample:properties:5out}
\end{sample}

%% properties_howto_examples.tex

\subsection{Examples}

\subsubsection{Store and fetch boolean properties}

\pn  Sample code  for  adding  boolean properties  and/or  flags in  a
properties  container (by  convention a  flag is  an  existing boolean
property with value \emph{true}) :

\pn
\VerbatimInput[frame=single,
numbers=left,
numbersep=2pt,
firstline=7,
lastline=14,
fontsize=\footnotesize,
showspaces=false]{\codingpath/properties_6.cxx}

\pn Sample code for checking  the existence of some boolean properties
(flags) in a properties container and fetch the associated value :

\pn
\VerbatimInput[frame=single,
numbers=left,
numbersep=2pt,
firstline=16,
lastline=35,
fontsize=\footnotesize,
showspaces=false]{\codingpath/properties_6.cxx}

\pn Sample code to store a vector of boolean values :

\pn
\VerbatimInput[frame=single,
numbers=left,
numbersep=2pt,
firstline=38,
lastline=41,
fontsize=\footnotesize,
showspaces=false]{\codingpath/properties_6.cxx}

\pn Sample code to fetch a vector of boolean values :
\pn
\VerbatimInput[frame=single,
numbers=left,
numbersep=2pt,
firstline=45,
lastline=48,
fontsize=\footnotesize,
showspaces=false]{\codingpath/properties_6.cxx}


\subsubsection{Store and fetch integer properties}


\pn  Sample  code  for  adding  integer  properties  in  a  properties
container :

\pn
\VerbatimInput[frame=single,
numbers=left,
numbersep=2pt,
firstline=10,
lastline=11,
fontsize=\footnotesize,
showspaces=false]{\codingpath/properties_7.cxx}

\pn Sample code for checking  the existence of some integer properties
in a properties container and fetch the associated value :

\pn
\VerbatimInput[frame=single,
numbers=left,
numbersep=2pt,
firstline=13,
lastline=23,
fontsize=\footnotesize,
showspaces=false]{\codingpath/properties_7.cxx}

%% XXX
%\end{document}

\pn Sample code to store a vector of integer values :

\pn
\VerbatimInput[frame=single,
numbers=left,
numbersep=2pt,
firstline=26,
lastline=31,
fontsize=\footnotesize,
showspaces=false]{\codingpath/properties_7.cxx}

\pn Sample code to fetch a vector of integer values :
\pn
\VerbatimInput[frame=single,
numbers=left,
numbersep=2pt,
firstline=35,
lastline=40,
fontsize=\footnotesize,
showspaces=false]{\codingpath/properties_7.cxx}

\pn Sample code to get the size of a vector of integer values :
\pn
\VerbatimInput[frame=single,
numbers=left,
numbersep=2pt,
firstline=46,
lastline=47,
fontsize=\footnotesize,
showspaces=false]{\codingpath/properties_7.cxx}

\pn Sample code to fetch a value at given index of a vector of integer values :
\pn
\VerbatimInput[frame=single,
numbers=left,
numbersep=2pt,
firstline=49,
lastline=50,
fontsize=\footnotesize,
showspaces=false]{\codingpath/properties_7.cxx}

\pn Sample code to change a scalar integer value. The \texttt{change} methods
request that the property exists and has the right type :
\pn
\VerbatimInput[frame=single,
numbers=left,
numbersep=2pt,
firstline=52,
lastline=52,
fontsize=\footnotesize,
showspaces=false]{\codingpath/properties_7.cxx}

\pn Sample code to change a value at given index from a vector of integers :
\pn
\VerbatimInput[frame=single,
numbers=left,
numbersep=2pt,
firstline=53,
lastline=53,
fontsize=\footnotesize,
showspaces=false]{\codingpath/properties_7.cxx}

\pn Sample code to update some integer scalar values. The \texttt{update}
methods request that the property has the right type is it already exists,
otherwise it is created :
\pn
\VerbatimInput[frame=single,
numbers=left,
numbersep=2pt,
firstline=58,
lastline=59,
fontsize=\footnotesize,
showspaces=false]{\codingpath/properties_7.cxx}

\subsubsection{Store and fetch real properties}

The real properties are handled  is the same way the integer properties
are.   The  \texttt{store},  \texttt{store\_real},  \texttt{is\_real},
\texttt{change},        \texttt{change\_real},        \texttt{update},
\texttt{update\_real},  \texttt{fetch},  \texttt{fetch\_real},
\texttt{fetch\_real\_scalar},
\texttt{fetch\_real\_vector}  methods
are implemented.

\subsubsection{Store and fetch string properties}


The  string  properties  are  handled  is the  same  way  the  integer
properties    are.    The    \texttt{store},   \texttt{store\_string},
\texttt{is\_string},     \texttt{change},     \texttt{change\_string},
\texttt{update},        \texttt{update\_string},       \texttt{fetch},
\texttt{fetch\_string},
\texttt{fetch\_string\_scalar},
\texttt{fetch\_string\_vector} methods are implemented.

\pn  Reminder:  a  string  value  cannot  contents  the  double  quote
character, because it is used as the string delimiter character.

%% end of properties_howto_examples.tex


%% end of properties_howto.tex
